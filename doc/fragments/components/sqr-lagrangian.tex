\providecommand{\docroot}{../..}
\documentclass[\docroot/reports/draft/report.tex]{subfiles}

\begin{document}

    Поскольку гравитационные волны очень слабы, для их описания фактически достаточно рассматривать лишь малые отклонения метрики пространства $g_{ij}$ от некоторой фоновой метрики $\gamma_{ij}$ (Минковского, Шварцшильда). Как предложил еще Эйнштейн \cite{einstein_grav_waves}, запишем метрику в виде
    %
    \begin{equation}\label{eq:gab}
        {}^{(4)}g_{ij} = {}^{(4)}\gamma_{ij} + {}^{(4)}h_{ij} .
    \end{equation}
    %
    Здесь ${}^{(4)}\gamma_{ij}$~--- четырехмерная метрика, $h_{ij}$~--- малые поправки к ней. Аппарат биметрического формализма и некоторые его следствия описаны, например, в \cite{burlankov_jetp_covar_cosn}. Важную роль в них играет операция ковариантного дифференцирования в пространстве сравнения, ${}^{(\gamma)}\nabla_i$, определяемая через связности ${}^{(\gamma)}\Gamma^i_{jk}$ пространства сравнения (а не исследуемого пространства). Это позволяет, в частности, упростить математический аппарат исследования. Здесь и далее запись ${}^{(\gamma)}\nabla_i$ будет сокращаться просто до \enquote{${}_{;i}$}, если это не приводит к неоднозначности.

    В теории глобального времени все мыслимые метрики должны иметь вид:
    %
    \begin{equation}
        {}^{(4)}g_{ij} = \begin{pmatrix}
            1 - {}^{(3)}g_{ij} V^i V^j   & - {}^{(3)}g_{ij} V^i \\
            - {}^{(3)}g_{ij} V^j         & {}^{(3)}g_{ij}
        \end{pmatrix} , \qquad
        {}^{(4)}g^{ij} = \begin{pmatrix}
            - 1   & - V^j                           \\
            - V^i & {}^{(3)}g^{ij} + V^i V^j
        \end{pmatrix} ,
    \end{equation}
    %
    где $V^i$~--- некоторое поле скоростей. Следовательно, всякая фоновая метрика  ${}^{(4)}\gamma_{ij}$ должна предварительно быть приведена к глобальному времени. Метрика Минковского уже является приведенной, в то время как метрика Шварцшильда должна быть заменена соответствующей ей метрикой Пенлеве. (Напомним, что ${}^{(3)}g^{ij} \neq {}^{(4)}g^{ij}$ при $i,j \neq 0$.)

    Равным образом представляется и тензор возмущений:
    %
    \begin{equation}
        {}^{(4)}h_{ij} = \begin{pmatrix}
            0    & - v_j          \\
            -v_i & {}^{(3)}h_{ij}
        \end{pmatrix} .
    \end{equation}
    %
    Временн\'{а}я компонента фоновой метрики оставляется невозмущенной: время в ТГВ полагается глобальным.

    Для описания слабых гравитационных волн достаточно линеаризованных уравнений динамики, которые получаются из квадратичной по возмущениям $h_{ij}$ части лагранжиана Гильберта (\autoref{eq:hilbert-lagr}). В работе \cite{burlankov_grav_waves} квадратичная \enquote{вырезка} из лагранжиана приводится к следующему трехмерному виду:
    %
    \begin{equation}\label{eq:lagr2}
        \begin{gathered}
            \mathcal{L}^{(2)} = \frac{1}{8} \qty(
                \gamma^{ij} \gamma^{kl} \qty(
                    E_{ik} E_{jl} - E_{ij} E_{kl}
                ) - B^i_j B^j_i
            ), \qquad \mathfrak{L}^{(2)} = \mathcal{L}\sqrt{\gamma} ,\\
            E_{ij} = D_t{h}_{ij} + v_{i;j} + v_{j;i}, \quad
            B^i_j = \Opsr(h)^i_j .
        \end{gathered}
    \end{equation}
    %
    (Снова напомним, что в записи $h_{ij;k}$ дифференцирование по $k$ производится в пространстве сравнения.) Второй член здесь не зависит от времени, потому является потенциальной энергией. Под $D_t{h}_{ij}$ понимается инвариантная производная по времени ($\underaccent{V}{\delta} h$~--- тензорная Ли-вариация):
    %
    \begin{gather}
        D_t{h}_{ij} =
            \dot{h}_{ij} - \underaccent{V}{\delta} h_{ij}\\
        -\underaccent{V}{\delta} h_{ij} =
            V^s h_{ij,s} + V^s_{,i} h_{sj} + V^s_{,j} h_{is} =
            V^s h_{ij;s} + V^s_{;i} h_{sj} + V^s_{;j} h_{is} .
    \end{gather}

\end{document}
