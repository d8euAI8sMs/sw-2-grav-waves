\providecommand{\docroot}{../..}
\documentclass[\docroot/reports/draft/report.tex]{subfiles}

\begin{document}

    Четные моды можно представить в следующем виде:
    %
    \begin{equation}
        h^{(e)} = \begin{pmatrix}-2h_e&h_{12}&0\\h_{12}&h_e-h_o&0\\0&0&h_e+h_o\end{pmatrix} .
    \end{equation}
    %
    В отношении операторов повышения и понижения компонента $h_e$ ведет себя как скаляр, пара $(h_{12},h_{13})$~--- как вектор, другая пара $(h_{23},h_o)$~--- как тензор.

    Исходя из вида основного вариационного уравнения можно заключить, что отыскание полного решения можно свести к отысканию лишь нечетного решения с последующей \textit{генерацией} четного решения применением полуротора.

    Предметом изучения данной работы являются нечетные моды. Далее мы приступим к их подробному исследованию.

\end{document}
