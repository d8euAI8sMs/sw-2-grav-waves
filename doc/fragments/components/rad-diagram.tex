\def\docroot{../..}
\documentclass[\docroot/reports/draft/report.tex]{subfiles}

\begin{document}

    Еще одной показательной характеристикой излучения наряду с плотностью и потоком энергии является его диаграмма направленности, которая в случае гравитационного излучения определяется следующим образом:
    %
    \begin{equation}\label{eq:radp}
        J(\theta,\varphi;r) = h^{ij}h_{ij} = \gamma^{ik}\gamma^{jl}h_{ij}h_{kl}.
    \end{equation}
    %
    В отличие от электромагнитного поля, в котором параметр $r$ определяет лишь масштаб диаграммы направленности, в гравитационном поле $r$ существенно влияет на общий вид $J(\theta,\varphi;r)$. Можно говорить о том, что $r$ определяет зону, в которой анализируется диаграмма направленности: ближнюю ($r \to 0$), дальнюю ($r \to \infty$) или среднюю (переходную).

    В силу того, что зависимость от $\varphi$ вносится в сферические моды лишь множителем $\exp(im\varphi)$, усредненная по времени диаграмма направленности, как и средняя энергия, не зависит от $\varphi$. Наиболее показательной, однако, является мгновенная диаграмма направленности, взятая в произвольный момент времени, который естественно выбрать нулевым. Поскольку нас интересует действительная величина диаграммы направленности, в \autoref{eq:radp} в качестве $h_{ij}$ следует брать только действительную его часть.

\end{document}
