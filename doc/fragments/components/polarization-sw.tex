\def\docroot{../..}
\documentclass[\docroot/reports/draft/report.tex]{subfiles}

\begin{document}

    Четные моды можно далее представить в виде:
    %
    \begin{equation}
        h^{(e)} = \begin{pmatrix}h_{11}&h_{12}&0\\h_{12}&h_e-h_o&0\\0&0&h_e+h_o\end{pmatrix} .
    \end{equation}
    %
    В отношении операторов повышения и понижения компоненты $(h_{11},h_e)$ ведут себя как скаляры, пара $(h_{12},h_{13})$~--- как вектор, другая пара $(h_{23},h_o)$~--- как тензор.

    В то время как нечетные моды имеют нулевой след, четные в общем случае могут иметь более богатую структуру. Оператор $\Opsr$ устраняет след, так что на самом деле из нечетной моды можно получить лишь моду, у которой $h_{11} = -2 h_e$. Если $h_{ij}$ ищется в виде симтензора, т.е. его след и симдивергенция изначально полагаются равными нулю, четное решение можно генерировать из нечетного применением операции полуротора. В этом случае записывать вариационные уравнения для четной моды излишне.

\end{document}
