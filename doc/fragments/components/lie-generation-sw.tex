\def\docroot{../..}
\documentclass[\docroot/reports/draft/report.tex]{subfiles}

\begin{document}\label{sec:lie-generation}

    Основное вариационное уравнение (\autoref{eq:lagr2var}) имеет довольно громоздкий вид. Будучи переписанным относительно $h_{ij}$, оно существенно усложняется:
    %
    \begin{equation}\label{eq:lagr2var-hij}
        \begin{gathered}
            \gamma_{\alpha i}\gamma_{\beta j} D_t\qty(\qty[\gamma^{ik}\gamma^{jl} - \gamma^{ij}\gamma^{kl}] \qty(D_t{h_{kl}} + v_{(k;l)})) + \\ + \qty(D_t{h_{\alpha\beta}} + v_{(\alpha;\beta)} - \gamma_{\alpha\beta}\gamma^{ij}\qty(D_t{h_{ij}} + v_{(i;j)})) \Opdiv{V} + \Opbr(h)_{\alpha\beta} = 0 ,
        \end{gathered}
    \end{equation}
    %
    где за $v_{(i;j)}$ обозначена симметричная комбинация $v_{i;j} + v_{j;i}$. При зависимости от времени $h\sim\exp(-i\omega t)$ данное уравнение может быть переформулировано в виде уравнения на собственные функции некоторого оператора $\hat{\Box}$:
    %
    \begin{equation*}
        \lambda(\omega) h = \hat{\Box}\, h ,
    \end{equation*}
    %
    где $\lambda(\omega)$~--- некоторая константа (собственное значение). Оператор $\Opbr$ входит линейно в $\hat{\Box}$. Остальную часть $\hat{\Box}$ составляют различные комбинации $h_{ij}$ и его пространственных производных с полями скоростей $V^i$ и $v^i$ и их производными, причем всюду, естественно, тензор возмущений входит линейно.

    Центральной теоремой, обеспечивающей применение метода Ли-генерации к сферическим гармоникам гравитационного поля, является
    %
    \begin{theorem}
        Операторы Киллинга коммутируют с оператором $\hat{\Box}$.
    \end{theorem}

    Таким образом, нахождение полного решения сводится к получению угловой зависимости для базовой моды из \autoref{eq:killeq-hl0}, радиальной зависимости из \autoref{eq:lagr2var} и применению необходимое количество раз операторов повышения и понижения. Временн\'{а}я зависимость дается множителем $\exp(-i \omega t)$.

\end{document}
