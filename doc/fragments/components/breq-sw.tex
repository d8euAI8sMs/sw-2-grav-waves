\def\docroot{../..}
\documentclass[\docroot/reports/draft/report.tex]{subfiles}

\begin{document}

    Оказывается, что вариация исходного квадратичного лагранжиана приводит к следующим уравнениям динамики:
    %
    \begin{equation}\begin{aligned}\label{eq:lagr2var}
        -4\fdv{\mathfrak{L}^{(2)}}{h_{ij}} = D_t{e^{ij}} + e^{ij} \Opdiv{V} + \Opsr(B)^{ij} = 0, \qquad e^{ij} = E^{ij} - \gamma^{ij} \tr E .
    \end{aligned}\end{equation}
    %
    Будем называть полученное соотношение \textit{основным вариационным уравнением} или \textit{основным динамическим уравнением}. Вывод данного соотношения является чисто механистическим, хотя и весьма громоздким. Достаточно записать уравнения Эйлера-Лагранжа для квадратичного лагранжиана и соответствующим образом привести их (существенно проще это выглядит в ковариантной форме уравнений).

\end{document}
