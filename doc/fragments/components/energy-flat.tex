\providecommand{\docroot}{../..}
\documentclass[\docroot/reports/draft/report.tex]{subfiles}

\begin{document}

    Плотность энергии отличается от квадратичного лагранжиана знаком перед потенциальной энергией:
    %
    \begin{equation}\label{eq:eps}
        \varepsilon = \pdv{\mathfrak{L}^{(2)}}{\dot{h}_{ij}} \dot{h}_{ij} - \mathfrak{L}^{(2)}
                    = \frac{1}{8} \qty(
            \gamma^{ij} \gamma^{kl} \qty(
                E_{ik} E_{jl} - E_{ij} E_{kl}
            ) + B^i_j B^j_i
        ) \sqrt{\gamma}
    \end{equation}

    Поток энергии (вектор Умова-Пойнтинга) описывается выражением:
    %
    \begin{equation}\label{eq:Uil}
        U^i = \pdv{\mathfrak{L}^{(2)}}{h_{kl;i}} \dot{h}_{kl}
            = -\frac{1}{4}\varepsilon^{ijk} E_{lk} B^l_j
    \end{equation}

    Если нас интересуют средние по времени величины плотности энергии и потока энергии, а метрика $h_{ij}$ представлена в виде
    %
    \begin{equation*}
        h_{ij}(t,r,\theta,\varphi) = \Re{h_{ij}(r,\theta,\varphi)\exp(-i \omega t)},
    \end{equation*}
    %
    где $h_{ij}(r,\theta,\varphi)$~--- не зависящая от времени комплексная величина (комплексная амплитуда), их можно вычислить из следующих соображений. Каждое произведение двух комплексных амплитуд $h_{ij} h_{kl}$ заменяется на $\flatfrac{1}{2} \Re{h_{ij} h^*_{kl}}$. Выражения для средней плотности энергии и среднего потока энергии примут вид:
    %
    \begin{equation}\label{eq:epsm}
        \overline{\varepsilon} = \frac{1}{16} \Re{
            \omega^2 \gamma^{ij} \gamma^{kl} \qty(
                h_{ij} h^*_{kl} - h_{ik} h^*_{jl}
            ) + B^i_j \qty(B^j_i)^*
        } \sqrt{\gamma},
    \end{equation}
    %
    \begin{equation}\label{eq:Uilm}
        \overline{U}^i = \frac{1}{8} \omega \Im{\varepsilon^{ijk} h_{lk} (B^l_j)^*} .
    \end{equation}

\end{document}
