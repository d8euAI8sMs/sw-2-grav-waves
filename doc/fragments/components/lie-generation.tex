\providecommand{\docroot}{../..}
\documentclass[\docroot/reports/draft/report.tex]{subfiles}

\begin{document}

    Для базовой моды при зависимости от времени $\sim\exp(-i\omega t)$ основное вариационное уравнение (\autoref{eq:lagr2var-br}) примет вид:
    %
    \begin{equation}\label{eq:lagr2var-br-hl0}
        \Opbr(h_{l0}) = - \omega^2 h_{l0} .
    \end{equation}

    Центральной теоремой, обеспечивающей применение метода Ли-генерации к сферическим гармоникам гравитационного поля, является
    %
    \begin{theorem}
        Операторы Киллинга коммутируют с оператором биротор $\Opbr$.
    \end{theorem}

    Таким образом, нахождение полного решения сводится к получению угловой зависимости для базовой моды из \autoref{eq:killeq-hl0}, радиальной зависимости из \autoref{eq:lagr2var-br-hl0} и применению необходимое количество раз операторов повышения и понижения. Временн\'{а}я зависимость дается множителем $\exp(-i \omega t)$.

\end{document}
