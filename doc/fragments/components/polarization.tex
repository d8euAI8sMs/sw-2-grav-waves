\providecommand{\docroot}{../..}
\documentclass[\docroot/reports/draft/report.tex]{subfiles}

\begin{document}

    Можно показать, что вариационные уравнения в конечном итоге распадаются на две независимые группы уравнений, определяющих две поляризации тензорных мод:
    %
    \begin{equation}
        h^{(o)} = \begin{pmatrix}0&0&h_{13}\\0&0&h_{23}\\h_{13}&h_{23}&0\end{pmatrix} \quad\text{и}\quad
        h^{(e)} = \begin{pmatrix}h_{11}&h_{12}&0\\h_{12}&h_{22}&0\\0&0&h_{33}\end{pmatrix} .
    \end{equation}

    Будем называть $h^{(o)}$ \textit{нечетными}, а $h^{(e)}$~--- \textit{четными} модами.

    Оператор полуротора формально переводит моды одной поляризации в моды другой поляризации:
    %
    \begin{equation}\label{eq:sr-he-ho}
        \Opsr(h_{l0}^{(e)}) = k h_{l0}^{(o)}, \quad \Opsr(h_{l0}^{(o)}) = k^{-1} h_{l0}^{(e)}, \quad k = (l-1)(l+2) .
    \end{equation}
    %
    О некотором операторе $A$, формально переводящем четные гармоники в четные, а нечетные~--- в нечетные, можно говорить как о сохраняющем четность. Очевидно, что биротор является оператором, сохраняющим четность.

\end{document}
