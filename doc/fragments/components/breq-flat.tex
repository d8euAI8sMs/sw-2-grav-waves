\providecommand{\docroot}{../..}
\documentclass[\docroot/reports/draft/report.tex]{subfiles}

\begin{document}

    Если фоновая метрика ${}^{(4)}\gamma_{ij}$ представляет собой метрику Минковского, то поле скоростей $V^i = 0$. Если же при этом также воспользоваться глобальной калибровкой (см. далее \autoref{sec:gauge-invariance}) и обратить $v^i$ в нуль, все дальнейшие соотношения существенно упростятся. Например, существенно упрощается $E_{ij}$:
    %
    \begin{equation*}
        E_{ij} = \dot{h}_{ij} .
    \end{equation*}
    %
    Далее будем полагать, что фоновая метрика является плоской, а все вычисления производятся в глобальной калибровке.

    Замечательным свойством лагранжиана в форме \autoref{eq:lagr2} при указанных условиях является то, что уравнения динамики имеют простой бироторный вид:
    %
    \begin{equation}\begin{aligned}\label{eq:lagr2var}
        4\fdv{\mathfrak{L}^{(2)}}{h_{ij}} &= \Opbr(h)^{ij} -
        \qty(\gamma^{ki}\gamma^{lj} - \gamma^{kl}\gamma^{ij}) \dot{E}_{kl} \\
                   &= \Opbr(h)^{ij} - \qty(\dot{E}^{ij} - \gamma^{ij} \trace{\dot{E}}) = 0.
    \end{aligned}\end{equation}

    Оказывается, что лишь только свойств оператора $\Opsr$ достаточно, чтобы указать, что $E_{ij}$, $B_{ij}$, равно как и $h_{ij}$ должны являться симтензорами. Из этого, в частности, следует, что $\tr E = 0$. Окончательно \textit{основное вариационное уравнение} для плоского пространства в глобальной калибровке формулируется в виде:
    %
    \begin{equation}\label{eq:lagr2var-br}
        \Opbr(h) = \ddot{h} .
    \end{equation}

\end{document}
