\def\docroot{../..}
\documentclass[\docroot/reports/draft/report.tex]{subfiles}

\begin{document}

    Определим операцию \textit{полуротора}:
    %
    \begin{equation}\label{eq:sr}
        \Opsr(h)^i_m = \varepsilon^{ikl} h_{mk;l} .
    \end{equation}
    %
    Здесь $\varepsilon_{ikl} = 0, \pm \sqrt{\gamma}$~--- тензор Леви-Чивиты.

    \textit{Биротор} получается двухкратным применением полуротора:
    %
    \begin{equation}\label{eq:br}
        \Opbr(h) = \Opsr(\Opsr(h)) .
    \end{equation}

    Операция \textit{симдивергенции} переводит тензор в вектор:
    %
    \begin{equation}\label{eq:sdiv}
        \Opsdiv(h)_i = h^j_{i;j} .
    \end{equation}

    Операция \textit{биградиента} переводит векторное поле в тензор:
    %
    \begin{equation}\label{eq:bigr}
        \Opbigr(\xi)_{ij} = \xi_{i;j} + \xi_{j;i}.
    \end{equation}

    Для введенных операций сохраняются тождества, справедливые для аналогичных векторных операций, например, симдивергенция биротора и биротор биградиента равны нулю тождественно:
    %
    \begin{equation}\label{eq:br-of-bigr}
        \Opbr(\Opbigr(\xi)) = 0,
    \end{equation}
    %
    \begin{equation}\label{eq:sdiv-of-br}
        \Opsdiv(\Opbr(h)) = 0.
    \end{equation}

    \textit{Симтензором} будем называть тензор $h$, удовлетворяющий трем свойствам:
    %
    \begin{equation*}
        h_{ij} = h_{ji}, \qquad
        \tr h = h^i_i = 0 , \qquad
        \Opsdiv(h) = 0 .
    \end{equation*}
    %
    Введенные выше операции тесно связаны с симтензорами. Так, $\Opbigr(\xi)$ на самом деле симтензор. Можно показать, что $\Opsr(h)$ будет являться симтензором при условии, что и $h$ является симтензором (или, по крайней мере, имеет нулевую симдивергенцию).

\end{document}
