\providecommand{\docroot}{..}
\documentclass[\docroot/reports/draft/report.tex]{subfiles}

\begin{document}

\onlyinsubfile{\tableofcontents}

1. В работе описан подход к построению теории гравитации на принципах глобального времени. Отказ от общековариантного подхода и переход к число лагранжеву формализму позволяют упростить получение линеаризованных уравнений на компоненты тензора возмущений, а также выражений для различных энергетических характеристик гравитационного поля. Сферическая симметрия задачи и явно выделенное глобальное время позволяют эффективно применить метод Ли-генерации сферических гармоник.

2. Данная работа является продолжением трудов \cite{burlankov_space_dynamics} и \cite{burlankov_grav_waves}. Здесь впервые найдены аналитические решения для радиальных частей нечетных мод.

3. Уделено внимание диаграммам направленности излучения в ближней и дальней зонах. Показано, что в ближней зоне на диаграмме направленности сказываются исчезающие на бесконечности компоненты тензора возмущений. Их влияние в на картину излучения в ближней зоне существенно и способно полностью видоизменить диаграмму направленности. Это отличает гравитационные волны от электромагнитных, в которых диаграмма направленности от расстояния до источника излучения не зависит.

4. Плотность и поток энергии также ведут себя довольно специфично. В ближней зоне снова сказывается многокомпонентность метрики, которая приводит к возникновению реверсивных потоков энергии. Показано, что в дальней зоне остается лишь постоянный радиальный поток энергии, зависящий лишь от одного угла $\theta$.

\onlyinsubfile{
    \clearpage
    \phantomsection
    \addcontentsline{toc}{section}{Список литературы}
    \bibliographystyle{\docroot/../lib/doc/bib/utf8gosttu}
    \bibliography{\docroot/../lib/doc/bib/math,\docroot/../lib/doc/bib/physics}
}

\end{document}
