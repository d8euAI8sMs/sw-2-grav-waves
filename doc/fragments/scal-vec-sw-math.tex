\def\docroot{..}
\documentclass[\docroot/reports/draft/report.tex]{subfiles}

\begin{document}

\onlyinsubfile{\tableofcontents}

\subsection{Лагранжев подход к описанию динамики}

    Классический лагранжев подход к описанию динамики некоторого поля $f(t,x)$ заключается в сведении физической задачи к задаче минимизации некоторого функционала \textit{действия} \cite{landau_v1}
    %
    \begin{equation}
        S\qty[f] = \int\limits_{(1)}^{(2)} L(t, x, f(t,x), f'(t,x)) \dd{x}\dd{t}.
    \end{equation}
    %
    Функция $L$, зависящая, помимо совокупности координат $x$, от искомого поля и его производных, называется \textit{плотностью функций Лагранжа}, которую далее мы будем называть просто функций лагранжа (строго говоря, функция Лагранжа~--- интеграл по объему от ее плотности).

    Принцип наименьшего действия подсказывает, что искомое поле дается минимизацией функционала действия на функции $f$. Минимум действия $S$ при фиксированных граничных условиях $(1)$ и $(2)$ достигается в том случае, если $f$ получается из \textit{уравнения Эйлера-Лагранжа} (\textit{вариационного уравнения}):
    %
    \begin{equation}
        \pdv{L}{f} - \sum\limits_{i=0}^{n} \dv{x^i}\pdv{L}{f_{,i}} = 0 .
    \end{equation}
    %
    Здесь $f_{,i}$~--- сокращенная запись для $\pdv*{f}{x^i}$, а для большей компактности положено $x^0 \equiv t$ (такая мнемоника вполне оправдана и упростит дальнейшее обобщение сказанного на четырехмерную нотацию).

    Если поле $f$ многокомпонентно, лагранжиан $L$ приобретает вид
    %
    \begin{equation}
        L = L(x, \dots, f_i(x), \dots, f'_i(x), \dots),
    \end{equation}
    %
    где индекс $i$ следует понимать как совокупный (например, если $f$~--- тензор второго ранга, то $f_i$ может означать $f_{mn}$, $f^{pq}$ и т.д.).

    Уравнения Эйлера-Лагранжа тривиально обобщаются на многокомпонентные поля:
    %
    \begin{equation}
        \pdv{L}{f_j} - \sum\limits_{i=0}^{n} \dv{x^i}\pdv{L}{f_{j,i}} = 0 .
    \end{equation}

    Энергетические характеристики поля определяются его (четырехмерным) тензором энергии-импульса \cite{landau_v1,landau_v2}:
    %
    \begin{equation}
        T_i^k = \sum_s f_{s,i} \pdv{L}{f_{s,k}} - \delta_i^k L
    \end{equation}
    %
    Тензор энергии-импульса в общем случае не симметричный. Однако неоднозначность его определения позволяет получить каноническую симметричную форму прибавлением некоторой калибровочной тензорной величины $\sum\limits_l \psi^{ikl}_l$, не меняющей свойств исходного тензора ($\psi^{ikl}$ должен быть антисимметричным по последним двум индексам). В дальнейшем мы будем полагать тензор энергии-импульса симметричным (симметризованным должным образом).

    Компоненты симметричного тензора энергии-импульса определяют скаляр плотности энергии $\epsilon$ и вектор плотности потока энергии $\vb{S}$ поля $f$:
    %
    \begin{equation}
        T^{ik} = \begin{pmatrix}
            \epsilon & S^k / c \\
            S^i / c  & \sigma^{ik}
        \end{pmatrix} .
    \end{equation}
    %
    Тензор $\sigma^{ij}$ называется тензором напряжений.

    Плотность и поток энергии сами по себе интересны, но мало показательны (особенно если зависимость их от времени гармоническая). Средние по периоду плотность и поток энергии наиболее значимы, потому целесообразно рассматривать именно их. Среднее по периоду некоторой энергетической функции $\mathscr{E}(f_j)$ может быть найдено интегрированием по периоду:
    %
    \begin{equation}
        \bar{\mathscr{E}} = \frac{1}{T} \int\limits_0^T \mathscr{E} \dd{t} .
    \end{equation}
    %
    Более простой в некотором смысле способ применительно к комплексному представлению поля $f$ в случае квадратичного лагранжиана заключается в замене всех произведений $f^{(k_1)}_i f^{(k_2)}_j$ на $1/2 \mathrm{Re}\qty{f^{(k_1)}_i f^{(k_2)*}_j}$.

    Далее мы будем оперировать лишь средними по периоду энергетическими величинами.

\subsubsection{Лагранжев подход в искривленном пространстве}

    В пространстве Минковского с координатами $\qty{x^0,\,x^1,\,x^2,\,x^3} = \qty{ct,\,x,\,y,\,z}$ вид действия некоторого поля $f$ (скалярного, векторного, тензорного), а также уравнений Эйлера-Лагранжа несколько упрощается в записи с применением канонической для римановой геометрии нотации \cite{landau_v1,riemannian_geometry_and_tensor_analysis}:
    %
    \begin{gather}
        S\qty[f] = \int\limits_{(1)}^{(2)} L(x^i, f_j(x^i), f'_j(x^i)) \dd[4]{x}; \\
        \pdv{L}{f_j} - \qty(\pdv{L}{f_{j,i}})_{,i} = 0 .
    \end{gather}

    В четырехмерном пространстве с произвольной метрикой $g$ в лагранжиане появляется множитель $\sqrt{-g}$~--- корень из детерминанта метрики (якобиан системы координат). Согласно введенной в \autoref{seq:notation} нотации,
    %
    \begin{equation}
        S = \int \mathcal{L} \sqrt{-g}\dd[4]{x}, \qquad L = \mathcal{L} \sqrt{-g} .
    \end{equation}

    Последнее выражение включает в себя результат, полученный строчкой выше. Действительно, это видно из того, что детерминант метрики Минковского $g = -1$.

    Разделение $L$ и $\mathcal{L}$ оправдано практическими соображениями: вариационные уравнения возможно переформулировать в ковариантном виде (через ковариантные производные), что оказывается весьма удобным. Следующая ковариантная формулировка уравнения Эйлера-Лагранжа справедлива для произвольных скалярного, векторного и тензорного полей:
    %
    \begin{equation*}
        \pdv{L}{f_{\alpha}} - \qty(\pdv{L}{f_{\alpha,\chi}})_{,\chi} = 0 \quad\iff\quad
        \pdv{\mathcal{L}}{f_{\alpha}} - \qty(\pdv{\mathcal{L}}{f_{\alpha;\chi}})_{;\chi} = 0 .
    \end{equation*}

    Тензор энергии-импульса $T_i^k$, как и лагранжиан, можно представить в виде
    %
    \begin{equation}
        T_i^k = \mathcal{T}_i^k \sqrt{-g}, \qquad \mathcal{T}_i^k = f_{s,i} \pdv{\mathcal{L}}{f_{s,k}} - \delta_i^k \mathcal{L},
    \end{equation}
    %
    однако механическая замена здесь обычных производных на ковариантные возможна лишь для скалярного поля.

\subsection{Метрика Шварцшильда}

    Метрика Шварцшильда~--- одна из наиболее фундаментальных метрик в ОТО. Она возникает в результате решения уравнений Эйнштейна для свободного от тяготеющих тел сферически-симметричного пространства \cite{schwarzschild_free_space_rus,mtw_v2}. Также она является внешней метрикой массивной сферической звезды \textit{гравитационного радиуса} $r_g$. Выраженная в сферических координатах $\qty{ct,\,r,\,\theta,\,\phi}$, это диагональная метрика вида
    %
    \begin{equation}
        g_{\alpha\beta} = \begin{pmatrix}
            -\qty(1 - \frac{r_g}{r}) & 0 & 0 & 0 \\
            0 & \qty(1 - \frac{r_g}{r})^{-1} & 0 & 0 \\
            0 & 0 & r^2 & 0 \\
            0 & 0 & 0 & r^2 \sin^2(\theta)
        \end{pmatrix} .
    \end{equation}
    %
    Ее детерминант $g = -r^4 \sin^4\theta$, т.е. $\sqrt{-g} = r^2 \sin^2\theta$, как и для обычной метрики сферической системы координат.

    Четырехмерный интервал $\dd{s} = g_{\alpha\beta} \dd{x^\alpha}\dd{x^\beta}$ для даной метрики принимает вид:
    %
    \begin{equation}
        \dd{s}^2 = -\qty(1 - \frac{r_g}{r}) c^2 \dd{t}^2
            + \frac{\dd{r}^2}{\qty(1 - \frac{r_g}{r})}
            + r^2 \qty(\dd{\theta}^2 + \sin^2(\theta) \dd{\phi}^2) .
    \end{equation}

    Метрика Шварцшильда диагональна. Ее \enquote{угловая} часть в точности соответствует таковой в метрике сферической системы координат плоского пространства (пространства Минковского). В силу этого, пространство Шварцшильда обладает теми же (трехмерными) движениями (теми же симметриями), что и пространство Минковского. Справедливо и более общее утверждение: любое сферически симметричной пространство обладает теми же (трехмерными) движениями.

\subsection{Особенности дифференциальных уравнений}\label{sec:deq_sing}

    Рассмотрим линейное дифференциальное уравнение второго порядка общего вида относительно функции $u(x)$ с в общем случае комплексными коэффициентами $P(x)$ и $Q(x)$:
    %
    \begin{equation}\label{eq:deq_init}
        u''(x) + P(x) u'(x) + Q(x) = 0.
    \end{equation}
    %
    Точки сингулярности коэффициентов $P(x)$ и $Q(x)$ уравнения называются \textit{особыми точками} дифференциального уравнения.

    Пусть $x_0$~--- особая точка. Если $x_0$~--- полюс не выше первого порядка коэффициента $P(x)$ и не выше второго порядка коэффициента $Q(x)$, то $x_0$ называется \textit{регулярной особой точкой} (также обыкновенной, правильной; подробнее см. \cite{whittaker_watson_1} или \cite{fedoryuk_de}).

    Представим (\ref{eq:deq_init}) в виде
    %
    \begin{equation}\label{eq:deq_init}
        (x - x_0)^2 u''(x) + (x - x_0) p(x) u'(x) + q(x) = 0.
    \end{equation}
    %
    Тогда точка $x = x_0$ является регулярной тогда и только тогда, когда оба коэффициента $p(x)$ и $q(x)$ являются аналитическими в окрестности $x_0$.

    Регулярные особые точки очень полезны в том плане, что решение уравнения в окрестности особой точки может быть выражено через коэффициенты $p(x)$ и $q(x)$ следующим образом. Если
    %
    \begin{equation}
        p(x) = p_0 + \sum\limits_{n=1}^\infty p_n (x - x_0)^n, \qquad
        q(x) = q_0 + \sum\limits_{n=1}^\infty q_n (x - x_0)^n,
    \end{equation}
    %
    то $u(x)$ можно представить в виде ряда
    %
    \begin{equation}
        u(x) = (x - x_0)^\lambda \qty(1 + \sum\limits_{n=1}^\infty u_n (x - x_0)^n),
    \end{equation}
    %
    где два \textit{показателя особой точки} $\lambda_{1,2}$ (в общем случае комплексных) определяются из \textit{характеристического уравнения}
    %
    \begin{equation}
        \lambda^2 + (p_0 - 1) \lambda + q_0 = 0.
    \end{equation}
    %
    Коэффициенты $u_n$ определяются последовательно:
    %
    \begin{equation}\label{eq:deq_coefs}
        u_n \qty((\lambda + n)^2 + (p_0 - 1) (\lambda + n) + q_0) +
            \sum\limits_{m = 1}^{n - 1} u_{n - m} \qty(p_m (\lambda + n - m) + q_m) = 0 .
    \end{equation}

    При совпадении показателей $\lambda_{1,2}$ или при их различии на целое число второе решение может не существовать. В таком случае недостающее решение можно определить из соотношения
    %
    \begin{equation}\label{eq:deq_alt_sln}
        w(r) = u(r) \qty(A + B \int\limits^r u^{-2}(r) \exp(- \int\limits^r r^{-1} p(r) \dd{r}) \dd{r})
    \end{equation}

\onlyinsubfile{
    \clearpage
    \phantomsection
    \addcontentsline{toc}{section}{Список литературы}
    \bibliographystyle{\docroot/../lib/doc/bib/utf8gosttu}
    \bibliography{\docroot/../lib/doc/bib/math,\docroot/../lib/doc/bib/physics}
}

\end{document}
