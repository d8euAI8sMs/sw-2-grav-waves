\providecommand{\docroot}{..}
\documentclass[\docroot/reports/draft/report.tex]{subfiles}

\begin{document}

\onlyinsubfile{\tableofcontents}

\subsection{Решение Шварцшильда для свободного пространства}

    К.~Шварцшильдом в 1916 г. \cite{schwarzschild_free_space_rus}\footnotemark{} было найдено статическое решение уравнений Эйнштейна для сферически симметричного пространства. Далее мы получим эквивалентное решение, пользуясь лагранжевым подходом.

    \footnotetext{
        В оригинальной статье используется другая метрика (конформные координаты). На наш взгляд ставшая уже канонической (см. напр. \cite{delgaty_lake_solution_list,cambridge_exact_solutions,boonserm_exact_solutions}) предлагаемая далее метрика существенно проще.
    }

    Основные предположения шварцшильдовой модели следующие \cite{schwarzschild_free_space_rus}:
    %
    \begin{enumerate}
        \item Стационарность метрики: $\pdv*{g_{\mu\nu}}{t} = 0$. Инвариантность метрики относительно обращения времени.
        \item Свободное пространство (вакуум, отсутствие материи).
        \item Сферическая симметрия пространства. Инвариантность метрики относительно поворотов и отражений.
    \end{enumerate}
    %
    Эти положения приводят к наиболее общему виду статической сферически симметричной метрики:
    %
    \begin{equation*}
        g_{\mu\nu} = \text{diag}_{\mu\nu}\ \{\ -e^{A(r)},\ e^{B(r)},\ r^2,\ r^2 \sin\theta \ \}
    \end{equation*}
    %
    Компоненты $g_{\theta\theta}$ и $g_{\phi\phi}$ являются компонентами обычной сферической метрики.

    В выбранной метрике плотность действия Гильберта $L = \flatfrac{1}{2} R \sqrt{-g}$ (после исключения высших производных) запишется в виде:
    %
    \begin{equation*}
        L = e^{\frac{1}{2} (A - B)} \qty(e^B + r B' - 1) .
    \end{equation*}

    Вариации лагранжиана по $A$ и $B$ дают, соответственно,
    %
    \begin{equation*}
        e^B + r B' - 1 = 0 , \qquad e^B - r A' - 1 = 0 .
    \end{equation*}
    %
    Первое уравнение решается аналитически, для решения второго достаточно вычесть одно уравнение из другого. В итоге имеем пару уравнений,
    %
    \begin{equation*}
        \qquad A' + B' = 0 , \qquad e^B + r B' - 1 = 0
    \end{equation*}
    %
    решение которых:
    %
    \begin{equation*}
        A = C_1 - B , \qquad B = - \log{\frac{e^{C_2}}{r} + 1} .
    \end{equation*}

    Подставляя полученное решение в обратную метрику $g^{\mu\nu}$, получим решение с пока еще не определенными константами:
    %
    \begin{equation*}
        g^{\mu\nu} = \text{diag}^{\mu\nu}\ \{\
            -\frac{e^{-C_1} r}{e^{C_2} + r},\
            \frac{e^{C_2} + r}{r},\
            r^{-2},\
            \qty(r^2 \sin^2\theta)^{-1}
        \ \} .
    \end{equation*}
    %
    Константа $e^{C_1} = 1$ определяется из требования $g^{\mu\nu} \to \gamma^{\mu\nu}$ при $r \to \infty$, где $\gamma^{\mu\nu}$~--- метрика Минковского. Теперь можно записать окончательный вид $g_{\mu\nu}$:
    %
    \begin{equation*}
        g_{\mu\nu} = \text{diag}_{\mu\nu}\ \{\
            - \qty(1 - \frac{r_s}{r}),\
            \qty(1 - \frac{r_s}{r})^{-1},\
            r^2,\
            r^2 \sin^2\theta
        \ \} , \qquad r_s = - e^{C_2} = \frac{2 G m}{c^2} = \frac{m}{4 \pi} .
    \end{equation*}
    %
    Здесь $r_s$~--- шварцшильдов радиус~--- определяется из приближения слабого гравитационного поля.

\subsection{Внутреннее решение Шварцшильда для жидкой сферы}

    В том же году К.~Шварцшильдом \cite{schwarzschild_fluid} было найдено еще одно решение уравнений эйнштейна внутри сферы из идеальной несжимаемой жидкости. В оригинальной работе использовались конформные координаты, которые, как замечалось ранее, не слишком удобны. В современных обозначениях внутреннее решение Шварцшильда встречается в различных более или менее удобных формулировках (см. напр. \cite{delgaty_lake_solution_list,cambridge_exact_solutions,boonserm_exact_solutions}), отличающихся друг от друга по форме. Здесь мы получим наиболее известную из них.

    Проделаем аналогичные предыдущему пункту шаги, однако теперь лагранжиан $\mathcal{L} = \mathcal{L}_g + \mathcal{L}_f$ состоит из двух слагаемых: лагранжиана пространства $\mathcal{L}_g = \flatfrac{1}{2} R$ и лагранжиана идеальной жидкости $\mathcal{L}_f = p$.

    Ранее был получен краткий вид лагранжиана пространства в метрике Шварцшильда:
    %
    \begin{equation*}
        L_g = e^{\frac{1}{2} (A - B)} \qty(e^B + r B' - 1) .
    \end{equation*}
    %
    Лагранжиан жидкости получим из уравнения динамики жидкости
    %
    \begin{equation*}
        \vb{u} \nabla_u p + (\varepsilon + p) \nabla_u \vb{u} + \nabla p = 0 .
    \end{equation*}
    %
    Здесь мы, следуя Шварцшильду, полагаем жидкость несжимаемой, следовательно $\varepsilon = \rho c^2 = const$. Давление $p$, однако, является функцией $r$ (зависимость от угловых координат исключается сферической симметрией).

    Получим выражение для 4-вектора скорости, учитывая симметрию и статичность задачи.
    %
    \begin{equation*}
        x^\mu = \{\ t,\ r,\ \theta,\ \phi \ \}^\mu , \quad
        u^\mu = \dv{x^\mu}{\tau} , \quad
        \dd{\tau} = \sqrt{ - g_{\alpha\beta} \dd{x^\alpha} \dd{x^\beta} }
                  = \sqrt{ - g_{\alpha\beta} \pdv{x^\alpha}{t} \pdv{x^\beta}{t} } \dd{t} .
    \end{equation*}
    %
    В силу сферической симметрии задачи от $t$ может зависеть только $r$. Однако, поскольку задача статическая, $\pdv*{r}{t}$ также обращается в нуль. Получаем
    %
    \begin{equation*}
        \dd{\tau} = \sqrt{ - g_{00} } \dd{t} , \quad
        u^\mu = \dv{x^\mu}{\tau}
              = \frac{\dd{x^\mu}}{\sqrt{- g_{00}} \dd{t}}
              = \frac{1}{\sqrt{- g_{00}}} \delta^\mu_0 .
    \end{equation*}
    %
    Отсюда
    %
    \begin{equation*}
        u_\nu = g_{\mu\nu} u^\mu = \frac{1}{\sqrt{- g_{00}}} g_{\nu 0}
    \end{equation*}

    Подставляя в уравнение динамики выражение для вектора скорости, получим дифференциальное уравнение
    %
    \begin{equation*}
        \frac{1}{2} (\varepsilon + p) A' + p' = 0 ,
    \end{equation*}
    %
    решение которого
    %
    \begin{equation*}
        p = \gamma e^{-\frac{A}{2}} - \varepsilon \equiv \mathcal{L}_f
    \end{equation*}
    %
    и является искомым лагранжианом жидкости.

    Вариации $L$ по неизвестным функциям $A$ и $B$ дают два уравнения, первое из которых,
    %
    \begin{equation*}
        e^B (1 - \varepsilon r^2) + r B' - 1 = 0 ,
    \end{equation*}
    %
    решается независимо:
    %
    \begin{equation*}
        B = - \log\qty(1 - \frac{1}{3} \varepsilon r^2 - \frac{const}{r}) .
    \end{equation*}
    %
    Выбор константы $const = 0$ следует из того, что при $r \to 0$ сингулярность в решении должна отсутствовать. Окончательно,
    %
    \begin{equation*}
        B = - \log\qty(1 - \frac{1}{3} \varepsilon r^2) .
    \end{equation*}

    Второе вариационное уравнение можно упростить, переформулировав его относительно $p$. Применив замену
    %
    \begin{equation*}
        A = 2 \log{\frac{\gamma}{\varepsilon + p}} ,
    \end{equation*}
    %
    получим уравнение
    %
    \begin{equation*}
        r (\varepsilon + p) (\varepsilon + 3 p) - 2 (\varepsilon r^2 - 3) p' = 0 ,
    \end{equation*}
    %
    которое также решается аналитически:
    %
    \begin{equation*}
        p = \frac{\varepsilon - \varepsilon e^{2 \varepsilon C} \sqrt{\varepsilon r^2 - 3}}{e^{2 \varepsilon C} \sqrt{\varepsilon r^2 - 3} - 3} .
    \end{equation*}

    Константа $C$ в последнем уравнении определяется из граничного условия $p(R) = 0$. Окончательно,
    %
    \begin{equation}
        p = \frac{\varepsilon \sqrt{\varepsilon R^2 - 3} - \sqrt{\varepsilon r^2 - 3}}{
            \sqrt{\varepsilon r^2 - 3} - 3 \sqrt{\varepsilon R^2 - 3}} .
    \end{equation}

    Константа $\gamma$ должна определиться из условий сшивки с метрикой свободного пространства на границе сферы (т.е. при $r = R$). Масса сферы определяется интегралом
    %
    \begin{equation*}
        m = \int\limits_0^R 4 \pi r^2 \rho \dd{r} = \qty( \rho = const ) = \frac{4}{3} \pi R^3 \rho .
    \end{equation*}
    %
    Сама же плотность жидкости $\rho = \flatfrac{\varepsilon}{c^2} = \varepsilon$.

    Отсюда заключаем, что
    %
    \begin{equation}
        \gamma = \frac{1}{3} \varepsilon^2 (3 - \varepsilon R^2) .
    \end{equation}

    Окончательно имеем:
    %
    \begin{equation}
        g_{\mu\nu} = \text{diag}_{\mu\nu}\ \{\
            \frac{1}{12} \qty(\sqrt{\varepsilon r^2 - 3} - 3 \sqrt{\varepsilon R^2 - 3})^2,\
            \frac{3}{3 - \varepsilon r^2},\
            r^2,\
            r^2 \sin\theta
        \ \}
    \end{equation}

    В таком виде решение не является, однако, математически однородным. Вынося $i$ из под корня, пользуясь определением шварцшильдова радиуса $r_s$, массы $m$, плотности жидкости $\rho$ и вводя параметр
    %
    \begin{equation*}
        \mathcal{R} = \frac{R^3}{r_s} ,
    \end{equation*}
    %
    можно придти к более компактному результату:
    %
    \begin{equation}
        g_{\mu\nu} = \text{diag}_{\mu\nu}\ \{\
            - \frac{1}{4} \qty(\sqrt{1 - \frac{r^2}{\mathcal{R}^2}} - 3 \sqrt{1 - \frac{R^2}{\mathcal{R}^2}})^2,\
            \qty(1 - \frac{r^2}{\mathcal{R}^2})^{-1},\
            r^2,\
            r^2 \sin\theta
        \ \}
    \end{equation}

\onlyinsubfile{
    \clearpage
    \phantomsection
    \addcontentsline{toc}{section}{Список литературы}
    \bibliographystyle{\docroot/../lib/doc/bib/utf8gosttu}
    \bibliography{\docroot/../lib/doc/bib/math,\docroot/../lib/doc/bib/physics}
}

\end{document}
