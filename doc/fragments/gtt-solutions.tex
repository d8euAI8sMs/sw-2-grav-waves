\providecommand{\docroot}{..}
\documentclass[\docroot/reports/draft/report.tex]{subfiles}

\begin{document}

\onlyinsubfile{\tableofcontents}

\subsection{Статическое сферическое решение}

    Наиболее общий вид статической сферически симметричной метрики в ТГВ \cite{burlankov_space_dynamics,burlankov_new_phys}:
    %
    \begin{equation*}
        g_{ij} = \text{diag} \begin{pmatrix}B^{-1}, & r^2, & r^2 \sin\theta\end{pmatrix} , \quad
        V^i = \begin{pmatrix}V(r), & 0, & 0\end{pmatrix} .
    \end{equation*}
    %
    При этом 4-интервал имеет вид (знак \enquote{-} обусловлен выбранной сигнатурой 4-метрики)
    %
    \begin{equation*}
        - \dd{s}^2 = \qty(1 - \frac{V^2}{B}) \dd{t}^2 + 2 \frac{V}{B} \dd{t} \dd{r} - \frac{1}{B^2} \dd{r}^2 - r^2 \dd{\Omega}^2 , \quad \dd{\Omega} = \dd{\theta}^2 + \sin^2{\theta} \dd{\phi}^2 .
    \end{equation*}

    Вид интервала математически существенно неоднороден. Также усложняется вид лагранжиана и вариационных уравнений. Можно привести интервал к симметричному виду
    %
    \begin{equation*}
        - \dd{s}^2 = \qty(1 - V^2) \dd{t}^2 + 2 \frac{V}{B} \dd{t} \dd{r} - \frac{1}{B^2} \dd{r}^2 - r^2 \dd{\Omega}^2
    \end{equation*}
    %
    с метрикой
    %
    \begin{equation*}
        g_{ij} = \text{diag} \begin{pmatrix}B^{-2}, & r^2, & r^2 \sin\theta\end{pmatrix} , \quad
        V^i = \begin{pmatrix}B(r)V(r), & 0, & 0\end{pmatrix} .
    \end{equation*}

    Введем замену переменных
    %
    \begin{equation*}
        V = \sqrt{1 - U} .
    \end{equation*}
    %
    После соответствующей замены и устранения не влияющих членов лагранжиан пространства $\mathfrak{L}_g = \flatfrac{1}{2} R^{(4)} \sqrt{-g}$ примет вид
    %
    \begin{equation*}
        \mathfrak{L}_g = \frac{1}{B} - r U B' .
    \end{equation*}

    Вариации лагранжиана по $B$ и $U$ дают уравнения:
    %
    \begin{equation*}
        B' = 0 , \quad B^2 (U + r U') = 1 .
    \end{equation*}
    %
    Из первого уравнения следует, что $B = const$. Второе дает:
    %
    \begin{equation*}
        U = \frac{C}{r} + \frac{1}{B^2} .
    \end{equation*}

    Константы $B$ и $C$ определятся из условий на бесконечности и условий сшивки с классической теорией. Первое заключается в том, что при $r \to \infty$ должно выполняться $g \to \gamma$, где $\gamma$~--- метрика Минковского. Отсюда получаем, что $B = 1$. Второе можно получить, усматривая эквивалентность движения тела в поле скоростей и его свободного падения в гравитационном потенциале. В обоих случаях движение тела будет одинаково, поэтому можно записать равенство кинетической и потенциальной энергий:
    %
    \begin{equation*}
        \frac{\qty|V|^2}{2} = \frac{G M}{r} ,
    \end{equation*}
    %
    $M$~--- масса источника поля, $G = \flatfrac{1}{8\pi}$~--- гравитационная постоянная Ньютона. Следует отметить, что под $V$ здесь понимается модуль вектора $\vb{V}$, именно $|V|^2 = V^i V_i$. Отсюда $C = \sqrt{2 G M}$.

    Окончательно имеем:
    %
    \begin{equation*}
        B = 1, \quad V^2 = \frac{2 G M}{r}
    \end{equation*}
    %
    \begin{equation*}
        g_{\mu\nu} = \begin{pmatrix}
            \frac{2 G M}{r} - 1 & -\sqrt{\frac{2 G M}{r}} & 0 & 0 \\
            -\sqrt{\frac{2 G M}{r}} & 1 & 0 & 0 \\
            0 & 0 & r^2 & 0 \\
            0 & 0 & 0 & r^2 \sin\theta \\
        \end{pmatrix} .
    \end{equation*}

    Полученное решение~--- частный случай более общей метрики Пенлеве \cite{burlankov_new_phys}.

\subsection{Внутренне решение для шварцшильдовой жидкой сферы}

    Выберем метрику ТГВ в том же виде, что и ранее. Получим внутреннее решение для шварцшильдовой жидкой сферы: будем полагать жидкость несжимаемой, т.е. $\varepsilon = const$, в то время как все еще $p = p(r)$.

    Выражение для коэффициентов метрики:
    %
    \begin{equation*}
        g_{ij} = \text{diag} \begin{pmatrix}B^{-2}, & r^2, & r^2 \sin\theta\end{pmatrix} , \quad
        V^i = \begin{pmatrix}B(r)V(r), & 0, & 0\end{pmatrix} , \quad
        V(r) = \sqrt{1 - U(r)} .
    \end{equation*}

    Лагранжиан пространства $\mathcal{L}_g$ останется тем же. Лагранжиан жидкости $\mathcal{L}_f$ получим из уравнения динамики жидкости:
    %
    \begin{equation*}
        \vb{u} \nabla_u p + (\varepsilon + p) \nabla_u \vb{u} + \nabla p = 0 .
    \end{equation*}
    %
    откуда
    %
    \begin{equation*}
        p = \frac{\gamma}{\sqrt{V}} - \varepsilon \equiv \mathcal{L}_f .
    \end{equation*}

    Вариации суммарного лагранжиана дают два уравнения:
    %
    \begin{gather*}
        \varepsilon r - \frac{1}{r} - \frac{\gamma r}{\sqrt{V}} + \frac{B^2 V}{r} + B^2 V' = 0 , \\
        \frac{\gamma r}{2 B V \sqrt{V}} + B' = 0 .
    \end{gather*}

    Уравнения достаточно сложно завязаны. Связность их можно уменьшить с помощью замены переменных:
    %
    \begin{equation*}
        V = \frac{1}{r w} , \qquad B = \sqrt{b w} .
    \end{equation*}
    %
    Применяя замену в отношении обратной метрики $g^{\mu\nu}$, из необходимости $g^{rr} > 0$ можно усмотреть необходимость $b(r) > 0$. Требование действительности коэффициентов метрики также определяет $w(r) > 0$.

    Вариационные уравнения с применением указанной замены по отдельности остаются завязанными, однако их разность дает простое уравнение:
    %
    \begin{equation*}
        b' = 1 - \varepsilon r^2 ,
    \end{equation*}
    %
    откуда сразу же
    %
    \begin{equation*}
        b = r \qty(1 - \frac{\varepsilon r^2}{3}) + const .
    \end{equation*}
    %
    Константа $const = 0$, поскольку иначе нарушается требование $g^{rr} \neq \infty$ в центре сферы. Из необходимости $b(r) > 0$ вытекает важное условие на $r$:
    %
    \begin{equation*}
        r < \frac{\varepsilon}{\sqrt{3}} ,
    \end{equation*}
    %
    которое определяет при $r \to R$ критический радиус при заданной массе звезды (поскольку $\varepsilon = \flatfrac{\rho}{c^2} \sim M$).

    Подстановка решения $b(r)$ в вариационные уравнения дает уравнение на $w(r)$:
    %
    \begin{equation*}
        (1 - \varepsilon r^2) w + \gamma r^\frac{5}{2} w^\frac{3}{2} + \frac{r}{3} (3 - \varepsilon r^2) w' = 0 .
    \end{equation*}
    %
    Его решение:
    %
    \begin{equation*}
        w(r) = - \frac{4 \varepsilon^2}{
            r \qty(
                12 \sqrt{C} \varepsilon \gamma \sqrt{3 - \varepsilon r^2} - 4 C \varepsilon^2 (3 - \varepsilon r^2) - 9 \gamma^2
            )
        } .
    \end{equation*}
    %
    Константа $C$ сугубо положительна.

    Давление $p$ определяется лишь функцией $w(r)$. Имеем:
    %
    \begin{equation*}
        p(r) = \gamma \sqrt{r w} - \varepsilon .
    \end{equation*}
    %
    Условие $p(R) = 0$ определяет константу $\gamma$:
    %
    \begin{equation*}
        \gamma = 2 \sqrt{C} \varepsilon \sqrt{3 - \varepsilon R^2} .
    \end{equation*}
    %
    (Второе решение $\gamma_1 = \flatfrac{\gamma}{5}$ приводит к отрицательному давлению, его следует отбросить.)

    Дальнейшие выкладки существенно упрощаются, если ввести
    %
    \begin{align*}
        3 - \varepsilon r^2 &= 3 \sinh^2\chi , \\
        3 - \varepsilon R^2 &= 3 \sinh^2\xi .
    \end{align*}
    %
    Тогда
    %
    \begin{equation*}
        \gamma = 2 \sqrt{3} \sqrt{C} \varepsilon \sinh\xi ,
    \end{equation*}
    %
    а компонента $g^{rr}$~--- просто $\sinh^2\chi$.

    Константа $C$ определится из условий сшивки с метрикой свободного пространства: $g^{\mu\nu} = \gamma^{\mu\nu}$ на границе сферы. Компонента $\gamma^{0r}$ в новых переменных выглядит просто:
    %
    \begin{equation*}
        \gamma^{0r} = - \sqrt{1 - \sinh\chi} .
    \end{equation*}
    %
    Компонента $g^{0r}$:
    %
    \begin{equation*}
        g^{0r} = - \sqrt{
            \frac{1}{3 C (1 - 3 \cosh\chi \sinh\xi)^2} - \sinh\chi^2
        } .
    \end{equation*}
    %
    Сшивка дает $C = \flatfrac{1}{12}$. Тогда $\gamma = e \sinh\xi$.

    Учет констант определяет давление:
    %
    \begin{equation*}
        p(r) = \varepsilon \frac{\sinh\chi - \sinh\xi}{3\sinh\xi - \sinh\chi} .
    \end{equation*}
    %
    Следует отметить, что и числитель, и знаменатель полученного выражения всегда положительны, что обеспечивает положительность давления при $r < R$. Применяя обратную замену $(\chi,\xi) \mapsto (r,R)$, получаем
    %
    \begin{equation*}
        p(r) = \varepsilon \frac{\sqrt{3 - \varepsilon r^2} - \sqrt{3 - \varepsilon R^2}}{3 \sqrt{3 - \varepsilon R^2} - \sqrt{3 - \varepsilon R^2}} .
    \end{equation*}

    Мы пришли к такому же результату, что и при решении данной задачи в метрике Шварцшильда. Отличен лишь вид метрики, но не выражение для давления жидкости.

\subsection{Внутренне решение для сжимаемой жидкой сферы}

    Будем рассматривать метрику в той же форме. При этом будем учитывать взаимодействие гравитационного поля с идеальной \textit{сжимаемой} жидкостью с уравнением состояния $\varepsilon = q p + \varepsilon_0$. Внешним решением будет выступать ранее полученное решение для свободного пространства. Внутреннее нам предстоит получить. Данная задача решалась также в \cite{burlankov_new_phys}.

    Из уравнения состояния жидкости,
    %
    \begin{equation*}
        \vb{u} \nabla \cdot \vb{u} - \frac{\chi}{1 - \chi} \nabla_u \vb{u} = \nabla \ln p^{\chi} ,
    \end{equation*}
    %
    можно получить выражение для $p$ через коэффициенты метрики:
    %
    \begin{equation*}
        \frac{1}{2 (1 - \chi)} \frac{\dd{U}}{U} + \frac{\dd{p}}{p} = 0 ,
    \end{equation*}
    %
    решение которого при $q = 3$, что соответствует случаю ультрарелятивистского предела, имеет вид:
    %
    \begin{equation*}
        p = \frac{c_0}{U^2} .
    \end{equation*}
    %
    Ранее обсуждалось, что для учета ненулевой плотности энергии $\varepsilon_0 = (q+1) \epsilon$ границы сферы достаточно заменить $p \mapsto P \equiv (p + \epsilon)$ в уравнении состояния, поэтому окончательно имеем
    %
    \begin{equation*}
        \mathcal{L}_f \equiv p = \frac{c_0}{U^2} - \epsilon .
    \end{equation*}

    Выбор $q = 3$ кажется наиболее реалистичным, поскольку при высоких давлениях (в нейтронных звездах и других массивных объектах) вещество должно переходить в ультрарелятивистский газ \cite{oppenheimer_volkoff,burlankov_new_phys} с $q = 3$ и $\epsilon = 0$.

    Неизвестные функции $U(r)$ и $B(r)$ определяются из вариационных уравнений:
    %
    \begin{gather*}
        r \epsilon - \frac{1}{r} - \frac{r c_0}{U^2} + \frac{B^2 U}{r} + B^2 U' = 0 \\
        \frac{2 r c_0}{B U^3} + B' = 0
    \end{gather*}

    Коэффициент метрики $g^{rr} = B^2 U$. В то же время в метрике свободного пространства $\gamma^{rr} = 1 - \flatfrac{2 G M}{r}$. Это подталкивает нас к замене $1 - \flatfrac{m(r)}{r} = B^2 U$, где $m(r)$~--- некоторая функция, значение которой при $r = R$ пропорционально массе жидкой сферы.

    Введением функций
    %
    \begin{equation*}
        m = r (1 - B^2 U) , \quad w = r^2 U^2
    \end{equation*}
    %
    полученные уравнения можно переформулировать:
    %
    \begin{align*}
        m'(r) &= r^2 \epsilon + \frac{3 r^4 c_0}{w(r)} , \\
        \frac{w'(r)}{2} &= \frac{r^4 c_0 + (1 - r^2 \epsilon) w(r)}{r - m(r)} .
    \end{align*}

    Проведем некоторый анализ полученного решения. Во-первых, обратная метрика в неизвестных $(B,m)$ имеет вид:
    %
    \begin{equation*}
        g^{0r} = - \sqrt{B^2(r) - 1 + \frac{m(r)}{r}} , \qquad
        g^{rr} = 1 - \frac{m(r)}{r} .
    \end{equation*}
    %
    Условия сшивки непосредственно определяют $B(R) = 1$.

    Во-вторых, условие обращения в нуль давления жидкости на границе сферы $p(R) = 0$ дает также
    %
    \begin{equation*}
        U(R) = \sqrt{\frac{c_0}{\epsilon}} .
    \end{equation*}
    %
    Отсюда, в частности, видно, что случай $\epsilon = 0$ несостоятелен: требования конечности массы $m(R) \sim M$, конечности $B(R) = 1$ и обращения в нуль давления $p(R) = 0$ (что влечет $U(R) \to \infty$) не могут быть удовлетворены одновременно. Таким образом, статическое решение для ультрарелятивисткого газа в принятой модели существовать не может.

\onlyinsubfile{
    \clearpage
    \phantomsection
    \addcontentsline{toc}{section}{Список литературы}
    \bibliographystyle{\docroot/../lib/doc/bib/utf8gosttu}
    \bibliography{\docroot/../lib/doc/bib/math,\docroot/../lib/doc/bib/physics}
}

\end{document}
