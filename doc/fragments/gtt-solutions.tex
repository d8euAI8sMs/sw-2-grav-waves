\providecommand{\docroot}{..}
\documentclass[\docroot/reports/draft/report.tex]{subfiles}

\begin{document}

\onlyinsubfile{\tableofcontents}

\subsection{Статическое сферическое решение}

    Наиболее общий вид статической сферически симметричной метрики в ТГВ \cite{burlankov_space_dynamics,burlankov_new_phys}:
    %
    \begin{equation*}
        g_{ij} = \text{diag} \begin{pmatrix}B^{-1}, & r^2, & r^2 \sin\theta\end{pmatrix} , \quad
        V^i = \begin{pmatrix}V(r), & 0, & 0\end{pmatrix} .
    \end{equation*}
    %
    При этом 4-интервал имеет вид (знак \enquote{-} обусловлен выбранной сигнатурой 4-метрики)
    %
    \begin{equation*}
        - \dd{s}^2 = \qty(1 - \frac{V^2}{B}) \dd{t}^2 + 2 \frac{V}{B} \dd{t} \dd{r} - \frac{1}{B^2} \dd{r}^2 - r^2 \dd{\Omega}^2 , \quad \dd{\Omega} = \dd{\theta}^2 + \sin^2{\theta} \dd{\phi}^2 .
    \end{equation*}

    Вид интервала математически существенно неоднороден. Также усложняется вид лагранжиана и вариационных уравнений. Можно привести интервал к симметричному виду
    %
    \begin{equation*}
        - \dd{s}^2 = \qty(1 - V^2) \dd{t}^2 + 2 \frac{V}{B} \dd{t} \dd{r} - \frac{1}{B^2} \dd{r}^2 - r^2 \dd{\Omega}^2
    \end{equation*}
    %
    с метрикой в виде
    %
    \begin{equation*}
        g_{ij} = \text{diag} \begin{pmatrix}B^{-2}, & r^2, & r^2 \sin\theta\end{pmatrix} , \quad
        V^i = \begin{pmatrix}B(r)V(r), & 0, & 0\end{pmatrix} .
    \end{equation*}

    Введем замену переменных
    %
    \begin{equation*}
        V \mapsto \sqrt{1 - U} .
    \end{equation*}
    %
    После соответствующей замены и устранения не влияющих членов лагранжиан пространства $L_g = \flatfrac{1}{2} R^{(4)} \sqrt{-g}$ примет вид
    %
    \begin{equation*}
        L_g = \frac{1}{B} - r U B' .
    \end{equation*}

    Вариации лагранжиана по $B$ и $U$ дают уравнения:
    %
    \begin{equation*}
        B' = 0 , \quad B^2 (U + r U') = 1 .
    \end{equation*}
    %
    Из первого уравнения следует, что $B = const$. Второе дает:
    %
    \begin{equation*}
        U = \frac{C}{r} + \frac{1}{B^2} .
    \end{equation*}

    Константы $B$ и $C$ определятся из условий на бесконечности и условий сшивки с классической теорией. Первое заключается в том, что при $r \to \infty$ должно выполняться $g \to \gamma$, где $\gamma$~--- метрика Минковского. Отсюда получаем, что $B = 1$. Второе можно получить, усматривая эквивалентность движения тела в поле скоростей и его свободного падения в гравитационном потенциале. В обоих случаях движение тела будет одинаково, поэтому можно записать равенство кинетической и потенциальной энергий:
    %
    \begin{equation*}
        \frac{\qty|V|^2}{2} = \frac{G M}{r} ,
    \end{equation*}
    %
    $M$~--- масса источника поля, $G = \flatfrac{1}{8\pi}$~--- гравитационная постоянная Ньютона. Следует отметить, что под $V$ здесь понимается модуль вектора $\vb{V}$, именно $|V|^2 = V^i V_i$. Отсюда $C = \sqrt{2 G M}$.

    Окончательно имеем:
    %
    \begin{equation*}
        B = 1, \quad V^2 = \frac{2 G M}{r}
    \end{equation*}
    %
    \begin{equation*}
        g_{\mu\nu} = \begin{pmatrix}
            \frac{2 G M}{r} - 1 & -\sqrt{\frac{2 G M}{r}} & 0 & 0 \\
            -\sqrt{\frac{2 G M}{r}} & 1 & 0 & 0 \\
            0 & 0 & r^2 & 0 \\
            0 & 0 & 0 & r^2 \sin\theta \\
        \end{pmatrix} .
    \end{equation*}

    Полученное решение~--- частный случай более общей метрики Пенлеве \cite{burlankov_new_phys}.

\subsection{Внутренне решение для идеальной жидкой самогравитирующей сферы}

    Будем рассматривать метрику в той же форме. При этом будем учитывать взаимодействие гравитационного поля с идеальной жидкостью. Внешним решением будет выступать ранее полученное решение для свободного пространства. Внутреннее нам предстоит получить. Данная задача решалась также в \cite{burlankov_new_phys}.

    Выражение для коэффициентов метрики:
    %
    \begin{equation*}
        g_{ij} = \text{diag} \begin{pmatrix}B^{-2}, & r^2, & r^2 \sin\theta\end{pmatrix} , \quad
        V^i = \begin{pmatrix}B(r)V(r), & 0, & 0\end{pmatrix} , \quad
        V(r) = \sqrt{1 - U(r)} .
    \end{equation*}

    Суммарный лагранжиан системы $\mathcal{L}$ будет складываться из ранее полученного лагранжиана пространства-времени $\mathcal{L}_g$ и лагранжиана идеальной жидкости $\mathcal{L}_f = p$, где $p$~--- давление жидкости.

    Из уравнения состояния жидкости,
    %
    \begin{equation*}
        \vb{u} \nabla \cdot \vb{u} - \frac{\chi}{1 - \chi} \nabla_u \vb{u} = \nabla \ln p^{\chi} ,
    \end{equation*}
    %
    где $\vb{u}$~--- 4-вектор скорости, $\chi = \flatfrac{q}{(q+1)}$, $q = \flatfrac{v^2}{c^2}$~--- скоростной фактор жидкости, можно получить выражение для $p$ через коэффициенты метрики.

    В силу статичности задачи и ее сферической симметрии $\vb{u} = \begin{pmatrix}u^0,&0,&0,&0\end{pmatrix}$:
    %
    \begin{gather*}
        \dd{\tau}^2 = - g_{\mu\nu} \dd{x^\mu}\dd{x^\nu} = - g_{\mu\nu} \dv{x^\mu}{t} \dv{x^\nu}{t} \dd{t}^2 = - g_{\mu\nu} \delta^\mu_0 \delta^\nu_0 \dd{t}^2 = - g_{00} \dd{t}^2 , \\
        u^\alpha = \dv{x^\alpha}{\tau} = \dv{x^\alpha}{t} \dv{t}{\tau} = \delta^\alpha_0 \dv{t}{\tau} = \frac{1}{\sqrt{- g_{00}}} \delta^\alpha_0 , \qquad - g_{00} = U .
    \end{gather*}

    Учитывая вышесказанное, получаем из уравнения состояния дифференциальное уравнение на $p$:
    %
    \begin{equation*}
        \frac{1}{2 (1 - \chi)} \frac{\dd{U}}{U} + \frac{\dd{p}}{p} = 0 ,
    \end{equation*}
    %
    решение которого при $q = 3$, что соответствует случаю ультрарелятивистского предела, имеет вид:
    %
    \begin{equation*}
        p = \frac{c_0}{U^2} .
    \end{equation*}
    %
    Ранее обсуждалось, что для учета ненулевой плотности энергии $\varepsilon_0 = (q+1) \epsilon$ границы сферы достаточно заменить $p \mapsto P \equiv (p + \epsilon)$ в уравнении состояния, поэтому окончательно имеем
    %
    \begin{equation*}
        \mathcal{L}_f \equiv p = \frac{c_0}{U^2} - \epsilon .
    \end{equation*}

    Выбор $q = 3$ кажется наиболее реалистичным, поскольку при высоких давлениях (в нейтронных звездах и других массивных объектах) вещество должно переходить в ультрарелятивистский газ \cite{oppenheimer_volkoff,burlankov_new_phys} с $q = 3$ и $\epsilon = 0$.

    Неизвестные функции $U(r)$ и $B(r)$ определяются из вариационных уравнений:
    %
    \begin{align*}
        \fdv{(\mathcal{L}\sqrt{-g})}{B} = 0 \quad\Longrightarrow\quad&
            r \epsilon - \frac{1}{r} - \frac{r c_0}{U^2} + \frac{B^2 U}{r} + B^2 U' = 0 \\
        \fdv{(\mathcal{L}\sqrt{-g})}{U} = 0 \quad\Longrightarrow\quad&
            \frac{2 r c_0}{B U^3} + B' = 0
    \end{align*}

    Введением функций
    %
    \begin{equation*}
        m = r (1 - B^2 U) , \quad w = r^2 U^2
    \end{equation*}
    %
    полученные уравнения можно переформулировать:
    %
    \begin{align*}
        m'(r) &= r^2 \epsilon + \frac{3 r^4 c_0}{w(r)} , \\
        \frac{w'(r)}{2} &= \frac{r^4 c_0 + (1 - r^2 \epsilon) w(r)}{r - m(r)} .
    \end{align*}
    %
    Для ультрарелятивистского газа ($\epsilon = 0$) они существенно упрощаются:
    %
    \begin{align*}
        m'(r) &= \frac{3 r^4 c_0}{w(r)} , \\
        \frac{w'(r)}{2} &= \frac{r^4 c_0 + w(r)}{r - m(r)} .
    \end{align*}

\onlyinsubfile{
    \clearpage
    \phantomsection
    \addcontentsline{toc}{section}{Список литературы}
    \bibliographystyle{\docroot/../lib/doc/bib/utf8gosttu}
    \bibliography{\docroot/../lib/doc/bib/math,\docroot/../lib/doc/bib/physics}
}

\end{document}
