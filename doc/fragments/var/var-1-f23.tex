\providecommand{\docroot}{../..}
\documentclass[\docroot/report.tex]{subfiles}

\begin{document}

Уравнение на $f_{23}$ имеет вид
%
\begin{equation}\label{eq:df23}
    f''(r) - l_1 \frac{ 2 r f'(r) + (r^2 - l_2) f(r) }{ r^2 (r^2 - l1) } + f(r) = 0 ,
\end{equation}
%
где $l_1 = (l-1)(l+2)$, $l_2 = l(l+1)$. Отсюда видно, что точки $r^2 = 0$ и $r^2 = l_2$ являются особыми точками. Исследуем поведение функции $f_{23}$ в окрестности $r^2 = l_2$.

%
%
%
%%%%%%%%%%%%%%%%%%%%%%%%%%%%%%%%%%%%%%%%%%%%%%%%%%%%%%%%%%%%%%%%%%%%%%%
%                        SUBSECTION                                   %
%%%%%%%%%%%%%%%%%%%%%%%%%%%%%%%%%%%%%%%%%%%%%%%%%%%%%%%%%%%%%%%%%%%%%%%
%
%
%

\subsection{Окрестность особой точки $r = r_0 = \sqrt{(l-1)(l+2)}$}

Как уравнение класса Фукса, \autoref{eq:df23} имеет $\rho = 0$ и $\rho = 2$ в исследуемых точках. Нам в первую очередь интересно первое решение (с $\rho = 0$) как физически реализуемое (второе решение, содержащее логарифм, отвечает нефизичным источникам). Оно имеет простой вид:
%
\begin{equation}\label{eq:f23-series}
    f(r - r_0) = \sum\limits_{n=0}^\infty b_n (r - r_0)^n .
\end{equation}
%
Подставляя его в уравнение \autoref{eq:df23}, получим рекуррентные соотношения для коэффициентов $b_n$:
%
\begin{equation}\begin{gathered}\label{eq:f23-series-coefs}
    b_0 = \alpha, \quad
    b_1 = \flatfrac{\alpha}{r_0}, \quad
    b_2 = \beta, \\
    b_3 = - \flatfrac{2}{3 r_0} (\alpha + 2 \beta), \quad
    b_4 = \flatfrac{1}{12 r_0^2} (7 \alpha + 20 \beta), \quad
    b_5 = \flatfrac{1}{60 r_0^3} (-41 \alpha + 8 (15 + r_0^2) \beta), \\
    b_n = - \frac{1}{2(n-2)n r_0^3} \\ (b_{n-5} + 4 r_0 b_{n-4}
          + (4-7n+n^2+4r_1^2) b_{n-3} \\ + 4r_0(n-3)(n-2) b_{n-2} + r_0^2(5n-7)(n-2) b_{n-1}),
\end{gathered}\end{equation}
%
где $r_0 = \sqrt{(l-1)(l+2)}$, $r_1 = \sqrt{l(l+1)}$, $\alpha$, $\beta$~--- параметры.

С увеличением числа коэффициентов радиус точного представления функции ее рядом увеличивается слабо. Для получения решения в большей окрестности гораздо более эффективным является численное решение уравнения \autoref{eq:df23} с заданными начальными условиями $f(r_0) = \alpha$, $f''(r_0) = 2\beta$ (двойка появляется из сличения ряда \label{eq:f23-series} и ряда Тейлора для $f(r)$ в окрестности $r_0$).

\end{document}
