\providecommand{\docroot}{../..}
\documentclass[\docroot/report.tex]{subfiles}

\begin{document}

Компонента $f_{13}$ выражается через $f_{23}$:
%
\begin{equation}\label{eq:f13-through-f23}
    f_{13} = r_0^2 \frac{ f_{23} - r f'_{23} }{ r^2 - r_0^2 } ,
\end{equation}
%
где $r_0^2 = (l-1)(l+2)$.

Точка $r^2 = r_0^2$ является особой точкой для дифференциального уравнения \autoref{eq:df23}. Однако в этой точке $f_{23}$ является вполне определенной функцией (выражается рядом по положительным целым степеням $(r-r_0)$). Из рекуррентных соотношений коэффициентов (\autoref{eq:f23-series-coefs}) можно видеть, что в окрестности особой точки $f_{13}$ будет ограничена, если $\alpha =0$. Действительно, в этом случае минимальная члена $(r-r_0)$ в разложении $f_{23}$ равна двум (а, соответственно, в разложении $f'_{23}$~--- единице).

\end{document}
