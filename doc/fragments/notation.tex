\def\docroot{..}
\documentclass[\docroot/reports/draft/report.tex]{subfiles}

\begin{document}

    В работе применяется следующая система единиц: скорость света $c = 1$, ньютоновская гравитационная постоянная $G = \flatfrac{1}{(8 \pi)}$. В частности, в этих обозначениях коэффициент $\kappa$, входящий в уравнения Эйнштейна и в лагранжиан Гильберта, становится равным единице: $\kappa = 1$.

    Метрика четырехмерного пространства выбирается с отрицательной сигнатурой $(\mathrel{-}\mathrel{+}\mathrel{+}\mathrel{+})$. Так, квадрат дифференциала собственного времени записывается в форме
    %
    \begin{equation*}
        \dd{\tau}^2 = - \dd{s}^2 = - g_{\mu\nu} \dd{x}^\mu \dd{x}^\nu >= 0 .
    \end{equation*}
    %
    Аналогично, для 4-скорости справедливо $u_s u^s = -1$.

    Лагранжиан без учета множителя $\sqrt{-g}$ обозначается стилизованным курсивным шрифтом, с учетом указанного множителя~--- стилизованным готическим шрифтом:
    %
    \begin{equation*}
        S_g = \int \mathfrak{L}_g \dd[4]{x} = \int \mathcal{L}_g \sqrt{-g} \dd[4]{x} .
    \end{equation*}
    %
    Те же соглашения применяются и к тензору энергии-импульса.

\onlyinsubfile{
    \clearpage
    \nocite{*}
    \phantomsection
    \addcontentsline{toc}{section}{Список литературы}
    \bibliographystyle{\docroot/../lib/doc/bib/utf8gosttu}
    \bibliography{\docroot/../lib/doc/bib/math,\docroot/../lib/doc/bib/physics}
}

\end{document}
