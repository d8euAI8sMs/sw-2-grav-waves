\providecommand{\docroot}{..}
\documentclass[\docroot/reports/draft/report.tex]{subfiles}

\begin{document}

    В работе \enquote{Проект обобщенной теории относительности и теории тяготения} \cite{einstein_grossman_grav_waves} от 1913 года Эйнштейн совместно с Гроссманом пишет:

    \enquote{\textit{Нам пришлось ввести некоторые далеко не очевидные, хотя и вероятные допущения. Искомые уравнения, по всей вероятности, должны иметь вид: $k \Theta_{\mu\nu} = \Gamma_{\mu\nu}$, где $k$~--- постоянная, $\Gamma_{\mu\nu}$~--- тензор второго ранга, образованный из производных фундаментального тензора $g_{\mu\nu}$.}}

    К концу 1915 года был найден наиболее общий вид тензора $\Gamma_{\mu\nu}$, который и лег в основу уравнений Эйнштейна в современной формулировке:
    %
    \begin{equation}\label{eq:einstein}
        G_{\mu\nu} + \Lambda g_{\mu\nu} = \kappa T_{\mu\nu}, \qquad
        G_{\mu\nu} = R_{\mu\nu} - \frac{R}{2} g_{\mu\nu} , \qquad
        \kappa = \frac{8 \pi G}{c^4} ,
    \end{equation}
    %
    $R$~--- скалярная кривизна пространства, $R_{\mu\nu}$~--- тензор Риччи, $T_{\mu\nu}$~--- тензор энергии-импульса материи, $g_{\mu\nu}$~--- метрический тензор, $c$~--- скорость света, $G$~--- ньютоновская гравитационная постоянная, $\Lambda$~--- космологическая постоянная. В реальных приложениях ввиду колоссальной малости $\Lambda$ членом уравнения $\Lambda g_{\mu\nu}$ часто пренебрегают.

    Параллельно с Эйнштейном математик Д.~Гильберт разрабатывал математический аппарат теории гравитации на основе принципа наименьшего действия. Результаты своей работы он доложил в Геттингенском университете 20 ноября 1915 года \cite{gilbert_phys}. Он доказал общую теорему.
    %
    \begin{theorem*}
        Вариация лагранжиана $\mathfrak{L}_f$ любого поля $f$ по компонентам метрического тензора определяет тензор энергии-импульса этого поля:
        %
        \begin{equation*}
            T^f_{ab} = 2 \fdv{\mathfrak{L}_f}{g^{ab}} .
        \end{equation*}
    \end{theorem*}
    %
    Он также показал, что действие Гильберта
    %
    \begin{equation*}
        S_g = \int \mathfrak{L}_g \dd[4]{x} , \qquad \mathfrak{L}_g = \mathcal{L}_g \sqrt{-g} , \qquad \mathcal{L}_g = \flatfrac{1}{2 \kappa} R
    \end{equation*}
    %
    с высокой степенью достоверности является действием самого пространства-времени, а вариация его лагранжиана по метрическому тензору определяет тензор Эйнштейна:
    %
    \begin{equation*}
        2\fdv{\mathfrak{L}_g}{g^{ab}} = T^g_{ab} = - \kappa^{-1} G_{ab} .
    \end{equation*}
    %
    Главным достижением работы Гильберта явился вывод уравнений Эйнштейна из вариационного принципа:
    %
    \begin{equation*}
        2 \fdv{\mathfrak{L}_\Sigma}{g^{ab}} =
        2 \fdv{(\mathfrak{L}_f + \mathfrak{L}_g)}{g^{ab}} =
        T^f_{ab} + T^g_{ab} =
        T^f_{ab} - \kappa^{-1} G_{ab} =
        0 .
    \end{equation*}
    %
    Уравнения Эйнштейна получаются вариацией суммарного действия по всем десяти независимым компонентам метрики.

\onlyinsubfile{
    \clearpage
    \phantomsection
    \addcontentsline{toc}{section}{Список литературы}
    \bibliographystyle{\docroot/../lib/doc/bib/utf8gosttu}
    \bibliography{\docroot/../lib/doc/bib/math,\docroot/../lib/doc/bib/physics}
}

\end{document}
