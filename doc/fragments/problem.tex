\providecommand{\docroot}{..}
\documentclass[\docroot/reports/draft/report.tex]{subfiles}

\begin{document}

\onlyinsubfile{\tableofcontents}

\subsection{Теория глобального времени\label{sec:tgt}}

    \subfile{\docroot/fragments/problem-tgt.tex}

    Уравнения Эйнштейна выражают принцип общей ковариантности: временн\'{а}я и пространственные переменные считаются равноправными. Основная проблема такого подхода заключается в том, что отличить временн\'{у}ю эволюцию системы от простой замены координат становится проблематично.

    Теория глобального времени отказывается от принципа общей ковариантности, считая время выделенной переменной, а связанную с ним компоненту метрики заданной: $g_{00} = 1$. Это допущение оказывается довольно плодотворным. Показывается \cite{burlankov_space_dynamics,burlankov_grav_waves}, что во многих задачах решения, полученные в рамках ТГВ, оказываются эквивалентными решениям через уравнения Эйнштейна. При этом ТГВ существенно облегчает решение некоторых задач, в частности связанных с анализом различного рода гравитационных волн \cite{burlankov_grav_waves}.

    Метрика пространства в ТГВ записывается следующим образом:
    %
    \begin{equation*}
        g^{(4)} = \begin{pmatrix}
            1       & -\vb{V} \\
            -\vb{V} & g^{(3)}
        \end{pmatrix},
    \end{equation*}
    %
    где $\vb{V}$~--- поле скоростей. Показывается, что поле скоростей носит лишь калибровочный характер. В частности, можно обратить $\vb{V}$ в нуль, что оказывается наиболее удобным при изучении гравитационных волн.

\subsection{Постановка задачи}

    В работе \cite{burlankov_grav_waves} описывается применение ТГВ к нахождению мод сферических гравитационных волн. В данной работе более глубоко изучаются нечетные сферические моды. Анализ волн от точечных источников может дать подходы к решению проблемы детектирования гравитационных волн.

\onlyinsubfile{
    \clearpage
    \phantomsection
    \addcontentsline{toc}{section}{Список литературы}
    \bibliographystyle{\docroot/../lib/doc/bib/utf8gosttu}
    \bibliography{\docroot/../lib/doc/bib/math,\docroot/../lib/doc/bib/physics}
}

\end{document}
