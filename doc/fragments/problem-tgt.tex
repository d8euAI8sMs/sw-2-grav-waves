\providecommand{\docroot}{..}
\documentclass[\docroot/reports/draft/report.tex]{subfiles}

\begin{document}

    В современной теории гравитации уравнения Эйнштейна без учета космологической постоянной записывают в следующей форме:
    %
    \begin{equation}\label{eq:einstein}
        G_{ab} = R_{ab} - \frac{1}{2} R g_{ab} = q T_{ab}, \quad
        q = \frac{8 \pi G}{c^4},
    \end{equation}
    %
    где $R$~--- скалярная кривизна пространства, $R_{ab}$~--- тензор Риччи, $T_{ab}$~--- тензор энергии-импульса материи, $g_{ab}$~--- метрический тензор, $c$~--- скорость света, $G$~--- ньютоновская гравитационная постоянная.

    К такой форме уравнений Эйнштейн пришел не сразу. В работе \enquote{Проект обобщенной теории относительности и теории тяготения} \cite{einstein_grossman_grav_waves} от 1913 года Эйнштейн совместно с Гроссманом пишет:

    \enquote{\textit{%
        Нам пришлось ввести некоторые далеко не очевидные, хотя
        и вероятные допущения. Искомые уравнения, по всей вероятности, должны иметь вид:
        %
        \begin{equation*}
            k \Theta_{\mu\nu} = \Gamma_{\mu\nu},
        \end{equation*}
        %
        где k~--- постоянная, $\Gamma_{\mu\nu}$~--- тензор второго ранга, образованный из производных фундаментального тензора $g_{\mu\nu}$.%
    }}

    К началу 1915 года Эйнштейном был представлен первый вариант своих уравнений:
    %
    \begin{equation*}
        R_{ab} = q T_{ab},
    \end{equation*}
    %
    отличающийся от \autoref{eq:einstein} отсутствием члена со скалярной кривизной. Решив эти уравнения для малых отклонений от метрики Минковского, он нашел, что планеты движутся по медленно вращающимся эллипсам, объяснив обнаруженное Леверье вращение перигелия Меркурия. В 1919 году было замерено предсказанное уравнениями отклонение света от прямолинейного распространения, величина которого вдвое превысила результаты классических расчетов.

    Однако вскоре Гильберт показал, что в общем случае уравнения Эйнштейна несовместны и занялся разработкой динамики пространства на основе принципа наименьшего действия.

    К концу 1915 года Эйнштейн получил совместную систему уравнений (\autoref{eq:einstein}).

    Результаты своей работы Гильберт доложил в Геттингенском университете 20 ноября 1915 года \cite{gilbert_phys}. Он доказал общую теорему.
    %
    \begin{theorem*}
        Вариация лагранжиана $\mathcal{L}_f$ любого поля $f$ по компонентам метрического тензора определяет тензор энергии-импульса этого поля:
        %
        \begin{equation*}
            T^f_{ab} = 2 \fdv{\mathcal{L}_f}{g^{ab}} .
        \end{equation*}
    \end{theorem*}
    %
    Он также показал, что действие Гильберта
    %
    \begin{equation*}
        S_g = k \int \mathcal{L}_g \dd[4]{x} , \quad \mathcal{L}_g = R \sqrt g
    \end{equation*}
    %
    с высокой степенью достоверности является действием самого пространства-времени, а вариация его плотности по метрическому тензору определяет тензор Эйнштейна:
    %
    \begin{equation*}
        2\fdv{\mathcal{L}_g}{g^{ab}} = T^g_{ab} = - \frac{1}{q} G_{ab} .
    \end{equation*}
    %
    Главным достижением работы Гильберта явился вывод уравнений Эйнштейна из вариационного принципа:
    %
    \begin{equation*}
        2 \fdv{\mathcal{L}_\Sigma}{g^{ab}} =
        2 \fdv{(\mathcal{L}_f + \mathcal{L}_g)}{g^{ab}} =
        T^f_{ab} + T^g_{ab} =
        T^f_{ab} - \frac{1}{q} G_{ab} =
        0 .
    \end{equation*}
    %
    Уравнения Эйнштейна получаются вариацией суммарного действия по всем десяти независимым компонентам метрики.

\onlyinsubfile{
    \clearpage
    \phantomsection
    \addcontentsline{toc}{section}{Список литературы}
    \bibliographystyle{\docroot/../lib/doc/bib/utf8gosttu}
    \bibliography{\docroot/../lib/doc/bib/math,\docroot/../lib/doc/bib/physics}
}

\end{document}
