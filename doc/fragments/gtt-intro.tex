\providecommand{\docroot}{..}
\documentclass[\docroot/reports/draft/report.tex]{subfiles}

\begin{document}

    Уравнения Эйнштейна выражают принцип общей ковариантности: временн\'{а}я и пространственные переменные считаются равноправными. Основная проблема такого подхода заключается в том, что отличить временн\'{у}ю эволюцию системы от простой замены координат становится проблематично.

    Проблема также выражается в невозможности прямого применения классических подходов анализа динамических систем (гамильтонов формализм). Тождественное равенство нулю тензора энергии-импульса гравитации и прочей материи (в т.ч. плотности и потока энергии) приводит к невозможности построения квантовой теории гравитации.

    Для решения части указанных проблем применяется подход расщепления четырехмерного пространства-времени на пространственную и временную составляющие (т.н. $(3+1)$-формализм) \cite{mtw_v2}. Наиболее известная и далеко идущая реализация этой техники принадлежит Арновитту, Дезеру и Мизнеру (АДМ). Ими было получено (АДМ-)разложение лагранжиана Гильберта, а позднее описан подход к построению гамильтониана \cite{adm_gr_dyn}.

    Метрика пространства в АДМ-разложении имеет вид:
    %
    \begin{equation*}
        \begin{pmatrix}
            {}^{(4)}g_{00} & {}^{(4)}g_{0k} \\
            {}^{(4)}g_{i0} & {}^{(4)}g_{ik} \\
        \end{pmatrix} =
        \begin{pmatrix}
            N_s N^s - N^2  & N_k    \\
            N_i            & g_{ik} \\
        \end{pmatrix} .
    \end{equation*}
    %
    Индексы трехмерного пространства $i$, $k$ изменяются от 1 до 3. Скалярная функция $N$ называется функцией хода. Трехмерный вектор $N^i$~--- вектором сдвига. Трехмерные индексы опускаются и поднимаются с использованием 3-метрики, например: $N_i = g_{is} N^s$.

    Обратная 4-метрика получается обращением прямой 4-метрики:
    %
    \begin{equation*}
        \begin{pmatrix}
            {}^{(4)}g^{00} & {}^{(4)}g^{0k} \\
            {}^{(4)}g^{i0} & {}^{(4)}g^{ik} \\
        \end{pmatrix} =
        \begin{pmatrix}
            - \qty(\flatfrac{1}{N^2})  & \flatfrac{N^k}{N^2} \\
            \flatfrac{N_i}{N^2}        & \qty(g^{ik} - \flatfrac{N^i N^k}{N^2}) \\
        \end{pmatrix} .
    \end{equation*}
    %
    При этом, как видно, ${}^{(4)}g^{ik} \neq g^{ik}$.

    Таким образом, 10 компонент 4-метрики переходят в 6 компонент 3-метрики, 3 компоненты вектора сдвига и одну скалярную функцию сдвига.

    АДМ-формализм, однако, по-прежнему не решает проблему равенства нулю плотности энергии $\varepsilon = T_{00}$ в общей теории относительности. Корнем проблемы является первое из уравнений Эйнштейна, получаемое приравниванием нулю вариации действия по компоненте метрики $g^{00}$. Отказ от этой вариации и, собственно, от одного из десяти уравнений Эйнштейна, приводит к решениям с в общем случае отличной от нуля плотностью энергии.

    Теория глобального времени исходит из этого принципа. Основываясь на представлении АДМ, она полагает функцию хода $N$ всюду равной единице. Время становится выделенной переменной. Это допущение оказывается довольно плодотворным. Показывается \cite{burlankov_space_dynamics,burlankov_grav_waves}, что во многих задачах решения, полученные в рамках ТГВ, эквивалентны классическим решениям. При этом ТГВ существенно облегчает решение некоторых задач, в частности связанных с анализом гравитационных волн \cite{burlankov_grav_waves}.

    Метрика пространства в ТГВ записывается следующим образом:
    %
    \begin{equation*}
        \begin{pmatrix}
            {}^{(4)}g^{00} & {}^{(4)}g^{0k} \\
            {}^{(4)}g^{i0} & {}^{(4)}g^{ik} \\
        \end{pmatrix} =
        \begin{pmatrix}
            - 1   & - V^k \\
            - V^i & g^{ik} - V^i V^k
        \end{pmatrix} .
    \end{equation*}
    %
    Векторное поле $V^i$ называется полем скоростей.

\onlyinsubfile{
    \clearpage
    \phantomsection
    \addcontentsline{toc}{section}{Список литературы}
    \bibliographystyle{\docroot/../lib/doc/bib/utf8gosttu}
    \bibliography{\docroot/../lib/doc/bib/math,\docroot/../lib/doc/bib/physics}
}

\end{document}
