\providecommand{\docroot}{..}
\documentclass[\docroot/reports/draft/report.tex]{subfiles}

\begin{document}

\onlyinsubfile{\tableofcontents}

\subsection{Симтензорный анализ}

    Здесь мы введем необходимый для дальнейшего изложения набор операций над симметричными тензорами второго ранга (\textit{симтензорами}).

    Определим операцию \textit{полуротора} так:
    %
    \begin{equation}\label{eq:sr}
        \Opsr(h)^i_m = \varepsilon^{ikl} h_{mk;l} .
    \end{equation}
    %
    Здесь тензор $\varepsilon_{ikl}$ определяется с учетом метрики (множителя $\sqrt{g}$).

    \textit{Биротор} получается двухкратным применением полуротора:
    %
    \begin{equation}\label{eq:br}
        \Opbr(h) = \Opsr(\Opsr(h)) .
    \end{equation}

    Операция \textit{симдивергенции} переводит симтензор в вектор:
    %
    \begin{equation}\label{eq:sdiv}
        \Opsdiv(h)_i = h^j_{i;j} .
    \end{equation}

    Операция \textit{биградиента} переводит векторное поле в симтензор:
    %
    \begin{equation}\label{eq:bigr}
        \Opbigr(\xi)_{ij} = \xi_{i;j} + \xi_{j;i}.
    \end{equation}

    Для введенных операций сохраняются тождества, справедливые для аналогичных векторных операций, например, симдивергенция биротора и биротор биградиента равны нулю тождественно:
    %
    \begin{equation}\label{eq:br-of-bigr}
        \Opbr(\Opbigr(\xi)) = 0,
    \end{equation}
    %
    \begin{equation}\label{eq:sdiv-of-br}
        \Opsdiv(\Opbr(h)) = 0.
    \end{equation}

\subsection{Квадратичный лагранжиан}

    Поскольку гравитационные волны очень слабы, фактически достаточно рассматривать лишь малые отклонения метрики пространства от метрики Минковского. Как предложил еще Эйнштейн\todo{источник}, запишем метрику в виде
    %
    \begin{equation}\label{eq:gab}
        g_{ab} = \gamma_{ab} + h_{ab} .
    \end{equation}
    %
    Здесь $\gamma_{ab}$~--- фоновая метрика Минковского, $h_{ab}$~--- малые поправки к ней.

    Для описания слабых гравитационных волн достаточно линеаризованных уравнений. Эти уравнения можно получить из квадратичного лагранжиана Гильберта. В работе \todo{гравитационные волны} показывается, что лагранжиан Гильберта в квадратичном приближении в планковской системе единиц имеет вид:
    %
    \begin{equation}\label{eq:lagrc2}
        \begin{gathered}
            \mathcal{L}^{(2)} = \frac{1}{8} \qty(
                \frac{S_1 - S_2}{2} - (S_3 - S_4)
            ) \sqrt{\gamma}, \\
            \begin{aligned}
                S_1 &= \gamma^{il}\gamma^{jm}\gamma^{kn} h_{ij;k} h_{lm;n} \\
                S_2 &= \gamma^{ij}\gamma^{lm}\gamma^{kn} h_{ij;k} h_{lm;n} \\
                S_3 &= \gamma^{in}\gamma^{jm}\gamma^{kl} h_{ij;k} h_{lm;n} \\
                S_4 &= \gamma^{in}\gamma^{jk}\gamma^{lm} h_{ij;k} h_{lm;n} .
            \end{aligned}
        \end{gathered}
    \end{equation}
    %
    С точки зрения ТГВ с ним не очень удобно работать, т.к. время не выделено явно, поэтому \autoref{eq:lagrc2} приводится к виду:
    %
    \begin{equation}\label{eq:lagr2}
        \begin{gathered}
            \mathcal{L}^{(2)} = \frac{1}{8} \qty(
                \gamma^{ij} \gamma^{kl} \qty(
                    E_{ik} E_{jl} - E_{ij} E_{kl}
                ) - B^i_j B^j_i
            ) \sqrt{\gamma}, \\
            E_{ij} = \dot{h}_{ij} + V_{i;j} + V_{j;i}, \quad
            B^i_j = \Opsr(h)^i_j .
        \end{gathered}
    \end{equation}
    %
    Несложно убедиться, что $\dot{h}_{ij} \equiv h_{ij;0}$. Второй член здесь не зависит от времени, потому является потенциальной энергией.

    Замечательным свойством лагранжиана в форме \autoref{eq:lagr2} является то, что уравнения динамики имеют простой бироторный вид:
    %
    \begin{equation}\begin{aligned}\label{eq:lagr2var}
        4\fdv{\mathcal{L}^{(2)}}{h_{ij}} &= \Opbr(h)^{ij} -
        \qty(\gamma^{ki}\gamma^{lj} - \gamma^{kl}\gamma^{ij}) \dot{E}_{kl} \\
                   &= \Opbr(h)^{ij} - \qty(\dot{E}^{ij} - \gamma^{ij} \trace{\dot{E}}) = 0.
    \end{aligned}\end{equation}

    Дальнейшие соотношения рассмотрим в глобальной калибровке.

    Бироторный вид уравнений (\autoref{eq:lagr2var}) удобен, однако в глобальной калибровке удобнее ввести пространственно-временной оператор
    %
    \begin{equation}\label{eq:brtr}
        \Opbrstar(h)^{ij} = \Opbr(h)^{ij} + \gamma^{ij} \trace{\ddot{h}} .
    \end{equation}
    %
    Тогда \autoref{eq:lagr2var} перепишется в виде
    %
    \begin{equation}\label{eq:lagr2var-brstar}
        \Opbrstar(h) = \ddot{h} .
    \end{equation}
    %
    Соотношение \ref{eq:lagr2var-brstar} назовем \textit{основным вариационным уравнением}.

    Плотность энергии отличается от квадратичного лагранжиана знаком перед потенциальной энергией:
    %
    \begin{equation}\label{eq:eps}
        \varepsilon = \pdv{\mathcal{L}^{(2)}}{\dot{h}_{ij}} \dot{h}_{ij} - \mathcal{L}^{(2)}
                    = \frac{1}{8} \qty(
            \gamma^{ij} \gamma^{kl} \qty(
                E_{ik} E_{jl} - E_{ij} E_{kl}
            ) + B^i_j B^j_i
        ) \sqrt{\gamma}
    \end{equation}

    Поток энергии (вектор Умова-Пойнтинга) описывается выражением:
    %
    \begin{equation}\label{eq:Uil}
        U^i = \pdv{\mathcal{L}^{(2)}}{h_{kl;i}} \dot{h}_{kl}
            = -\frac{1}{4}\varepsilon^{ijk} E_{lk} B^l_j
    \end{equation}

    Если нас интересуют средние по времени величины плотности энергии и потока энергии, а метрика $h_{ij}$ представлена в виде
    %
    \begin{equation*}
        h_{ij}(t,r,\theta,\varphi) = \Re{h_{ij}(r,\theta,\varphi)\exp(-i \omega t)},
    \end{equation*}
    %
    где $h_{ij}(r,\theta,\varphi)$~--- не зависящая от времени комплексная величина (комплексная амплитуда), их можно вычислить из следующих соображений. Каждое произведение двух комплексных амплитуд $h_{ij} h_{kl}$ заменяется на $\flatfrac{1}{2} \Re{h_{ij} h^*_{kl}}$. Выражения для средней плотности энергии и среднего потока энергии примут вид:
    %
    \begin{equation}\label{eq:epsm}
        \overline{\varepsilon} = \frac{1}{16} \Re{
            \omega^2 \gamma^{ij} \gamma^{kl} \qty(
                h_{ij} h^*_{kl} - h_{ik} h^*_{jl}
            ) + B^i_j \qty(B^j_i)^*
        } \sqrt{\gamma},
    \end{equation}
    %
    \begin{equation}\label{eq:Uilm}
        \overline{U}^i = \frac{1}{8} \omega \Im{\varepsilon^{ijk} h_{lk} (B^l_j)^*} .
    \end{equation}

\subsection{Калибровочная инвариантность}

    Тождество \ref{eq:br-of-bigr} определяет калибровочную инвариантность уравнений \ref{eq:lagr2var}: добавление биградиента произвольного поля $\xi$ к метрике $h$ при соответствующем изменении поля скоростей не меняет уравнений динамики:
    %
    \begin{equation}\begin{aligned}\label{eq:calibr-inv}
        \tilde{h}_{ij} &= h_{ij} + \xi_{i;j} + \xi_{j;i}, \\
        \tilde{V}_i    &= V_i - \dot{\xi}_i, \\
        \tilde{E}_{ij} &= E_{ij} + \dot{\xi}_{i;j} + \dot{\xi}_{j;i} \\
                       &= \dot{h}_{ij} + \qty(\tilde{V}_{i;j} + \dot{\xi}_{i;j}) + \qty(\tilde{V}_{j;i} +  \dot{\xi}_{j;i}), \\
                       &= \dot{h}_{ij} + V_{i;j} + V_{j;i} \\
                       &= E_{ij} .
    \end{aligned}\end{equation}
    %
    Правая часть \autoref{eq:lagr2var} не меняется в силу \autoref{eq:br-of-bigr}, а левая~--- в силу инвариантности $E_{ij}$ (\autoref{eq:calibr-inv}) относительно калибровочных преобразований.

    Наиболее простой является калибровка $\tilde{V}_i = 0$. Будем называть ее \textit{глобальной}. При этом:
    %
    \begin{equation}\label{eq:calibr-glob}
        \dot{\xi}_i = V_i, \quad \tilde{E}_{ij} = \dot{h}_{ij} + V_{i;j} + V_{j;i} = \dot{\tilde{h}}_{ij} .
    \end{equation}

\end{document}
