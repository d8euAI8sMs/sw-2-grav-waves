\providecommand{\docroot}{..}
\documentclass[\docroot/reports/draft/report.tex]{subfiles}

\begin{document}

\onlyinsubfile{\tableofcontents}

\subsection{Лагранжиан идеальной жидкости}

    Классический подход к изучению термодинамики идеальной жидкости связан с тензором энергии-импульса жидкости \cite{tolman}:
    %
    \begin{equation*}
        T^f_{ab} = (\varepsilon + p) u_a u_b + p g_{ab} .
    \end{equation*}
    %
    Здесь $\varepsilon$~--- плотность энергии жидкости, $p$~--- ее давление, $u^a$~--- 4-вектор скорости. В данной работе при формулировке динамики пространства мы исходим из действия Гильберта, потому не можем в явном виде применить полученный результат. Лагранжев аппарат описания динамики жидкости построен Б.~Шютцем в работе \cite{schutz_vel_pot}, которая базируется на представлении четырехмерной скорости $u^a$ шестью скалярными потенциалами: $(\mu,\phi,\alpha,\beta,\theta,S)$. Уравнения динамики жидкости получаются вариацией лагранжиана по этим потенциалам.

    Не все потенциалы, однако, являются существенными при изучении простейших моделей жидких сфер~--- потенциалы, связанные с тепловыми явлениями ($\theta$, $S$), а также потенциалы, описывающие вихри поля скоростей ($\alpha$, $\beta$), могут быть исключены из рассмотрения. Так, из оригинальных шести потенциалов следует оставить два: энтальпию $\mu$ и скалярный потенциал $\phi$. В итоге 4-скорость $u^a$ в своих ковариантных компонентах представляется в виде
    %
    \begin{equation*}
        u_a = \mu^{-1} \phi_{,a} = \mu^{-1} v_a .
    \end{equation*}
    %
    Векторное поле $v^a$ введенно для удобства. Шютц \cite{schutz_vel_pot} и цитированные там работы дают этому вектору также и физическую интерпретацию.

    Согласно Шютцу, основой действия жидкости является ее давление в параметрах $(\mu,S)$. Имея в виду введенные ранее границы модели, запишем:
    %
    \begin{equation}
        S_f = \int \mathcal{L}_f \sqrt{-g} \dd[4]{x} , \quad
        \mathcal{L}_f = p(\mu) .
    \end{equation}

    Энтальпия $\mu$ задается соотношением
    %
    \begin{equation*}
        \mu\rho = \varepsilon + p ,
    \end{equation*}
    %
    где $\rho$~--- плотность вещества жидкости. Показывается, что
    %
    \begin{equation*}
        \rho = p' \equiv \dv{p}{\mu} .
    \end{equation*}

    Из соотношения $u^a u_a = -1$ следует выражение для $\mu$:
    %
    \begin{equation*}
        \mu = \sqrt{-g^{ab} v_a v_b} .
    \end{equation*}

    Вариация лагранжиана по компонентам метрики $g^{ab}$ действительно дает тензор энергии-импульса жидкости:
    %
    \begin{equation}\begin{aligned}
        T^f_{ab} = \frac{-2}{\sqrt{-g}} \fdv{\qty(\sqrt{-g} \mathcal{L}_f)}{g^{ab}} &=
            \frac{-2}{\sqrt{-g}} \pdv{\qty(\sqrt{-g} \mathcal{L}_f)}{g^{ab}} \\ &=
            \frac{-2}{\sqrt{-g}} \qty(
                - \frac{1}{2 \mu} \sqrt{-g} p'(\mu) v_a v_b - \frac{1}{2} g_{ab} \sqrt{-g} p(\mu)
            ) \\ &=
            \frac{1}{\mu} p'(\mu) v_a v_b + p(\mu) g_{ab} \\ &=
            (p + \varepsilon) u_a u_b + p g_{ab} .
    \end{aligned}\end{equation}
    %
    Здесь использовано равенство
    %
    \begin{equation*}
        \fdv{\sqrt{-g}}{g^{ab}} = - \frac{1}{2} g_{ab} \sqrt{-g} .
    \end{equation*}

\subsection{Вариационные уравнения}

    Для любого тензора энергии-импульса справедливо тождество Гильберта:
    %
    \begin{equation}
        T^a_{b;a} = 0 ,
    \end{equation}
    %
    которое применительно к $T^f$ записывается в виде:
    %
    \begin{equation*}
        \qty(T^f)^a_{b;a} = t_{,a} u^a u_b + t (u^a_{;a} u_b + u^a u_{b;a}) + p_{,a} \delta^a_b = 0,
    \end{equation*}
    %
    где для краткости введено $t = p + \varepsilon$. Этот результат можно существенно упростить.

    Возьмем параллельную и перпендикулярную проекции полученного выражения на $u^b$, имея в виду тождества:
    %
    \begin{gather*}
        u_b u^b = -1 , \\
        0 = \qty(u_b u^b)_{;a} = u_{b;a} u^b + u_b u^b_{;a}
          = u_{b;a} u_s g^{sb} + u_b u^b_{;a}
          = u^s_{;a} u_s + u_b u^b_{;a}
          = 2 u_b u^b_{;a}
          = 0 .
    \end{gather*}
    %
    Для параллельной получаем
    %
    \begin{equation*}\begin{aligned}
        u^b \qty(T^f)^a_{b;a} &= - t_{,a} u^a - t u^a_{;a} + p_{,a} u^a \\
                              &= - \varepsilon_{,a} u^a - t u^a_{;a} = 0 .
    \end{aligned}\end{equation*}
    %
    Отсюда заключаем, что имеет место равенство
    %
    \begin{equation}
        \nabla_u \varepsilon + t\ \nabla \cdot \vb{u} = 0 .
    \end{equation}
    %
    Перпендикулярная строится с помощью тензора проектирования $P^a_b = \delta^a_b + u^a u_b$ (несложно убедиться, что $u^b P^a_b = 0$):
    %
    \begin{equation*}\begin{aligned}
        P^s_b \qty(T^f)^a_{s;a} = p_{,a} u^a u_b + t u^a u_{b;a} + p_{,b} = 0 ,
    \end{aligned}\end{equation*}
    %
    или,
    %
    \begin{equation}
        \vb{u} \nabla_u p + t \nabla_u \vb{u} + \nabla p = 0 .
    \end{equation}

    Еще одно уравнение получим вариацией действия по потенциалу $\sigma$:
    %
    \begin{equation*}
        0 = \fdv{\qty(\sqrt{-g} \mathcal{L}_f)}{\sigma} =
            \pdv{\qty(\sqrt{-g} \mathcal{L}_f)}{\sigma} -
            \qty(\pdv{\qty(\sqrt{-g} \mathcal{L}_f)}{v_k})_{,k} .
    \end{equation*}
    %
    Первое слагаемое обращается в нуль. Второе дает:
    %
    \begin{equation*}\begin{aligned}
        \qty(\pdv{\qty(\sqrt{-g} \mathcal{L}_f)}{v_k})_{,k} &=
        \qty(\sqrt{-g} \pdv{\mathcal{L}_f}{v_k})_{,k} =
        \qty(\sqrt{-g} \pdv{p(\mu)}{v_k})_{,k} =
        \sqrt{-g}^{-1} \qty(\pdv{p(\mu)}{v_k})_{;k} = 0
    \end{aligned}\end{equation*}
    %
    Производная $p$ по $v_k$ дает
    %
    \begin{equation*}
        \pdv{p(\mu)}{v_k} = p' \frac{1}{2 \mu} \qty(- 2 g^{k\alpha} v_\alpha) = - \rho u^k ,
    \end{equation*}
    %
    где использовано $\dv*{p}{\mu} = \rho$ и $v^k = \mu u^k$. Окончательно имеем
    %
    \begin{equation*}
        \qty(\rho u^k)_{;k} = 0 ,
    \end{equation*}
    %
    или
    %
    \begin{equation*}
        \rho_{,k} u^k + \rho u^k_{;k} = 0 .
    \end{equation*}
    %
    В векторной нотации
    %
    \begin{equation}
        \nabla_u \rho + \rho\ \nabla \cdot \vb{u} = 0 .
    \end{equation}

    Последнее полученное уравнение есть не что иное как уравнение неразрывности потока жидкости.

\subsection{Модель сжимаемой жидкости}

    Локальная скорость звука $a$ определяется зависимостью $p(\varepsilon)$:
    %
    \begin{equation*}
        \frac{a^2}{c^2} \equiv q^{-1} = \dv{p}{\varepsilon} .
    \end{equation*}
    %
    Случай $q = 0$ соответствует несжимаемой жидкости. При этом скорость звука $a = \infty$, плотность энергии $\varepsilon = const$ и плотность жидкости $\rho = \flatfrac{\varepsilon}{c^2} = const$. Модель несжимаемой жидкости имеет право на существование, однако она становится очень грубой при применении к массивным телам. Более реалистичным является приближение $q = const$: в массивных объектах при высоких давлениях вещество должно переходить в ультрарелятивистский газ \cite{oppenheimer_volkoff,burlankov_new_phys,landau_v5} с $q = 3$. Полагая $q = const$, имеем:
    %
    \begin{equation*}
        \varepsilon = q p + \varepsilon_0 ,
    \end{equation*}
    %
    где за $\varepsilon_0$ обозначена плотность энергии на границе жидкой сферы (сказанное справедливо в виду того, что на границе $p = 0$). Случай $\varepsilon_0 = 0$ в общем случае соответствует газу~--- резкая граница между газовой сферой и внешним пространством, таким образом, отсутствует.

    Дальнейший вывод серьезно упрощается, если полностью пренебречь $\varepsilon_0$. Полагая $q = const$, из выражения $\varepsilon = \mu\rho - p$ можно получить соотношение, связывающее $p$, $\mu$ и $\rho$:
    %
    \begin{equation}
        p = \frac{\mu\rho}{q + 1} .
    \end{equation}

    Приступим к формулировке уравнения динамики сжимаемой жидкости. Выпишем все уравнения для жидкости, которые получены на данный момент:
    %
    \begin{gather}
        \rho = \dv{p}{\mu} \\
        p = \frac{\mu\rho}{q + 1} , \quad q = const \\
        \nabla_u \rho + \rho\ \nabla \cdot \vb{u} = 0 \\
        \vb{u} \nabla_u p + (p + \varepsilon)\ \nabla_u \vb{u} + \nabla p = 0 \\
        \nabla_u \varepsilon + (p + \varepsilon)\ \nabla \cdot \vb{u} = 0 .
    \end{gather}

    Из первых двух уравнений можно выразить $\rho$ и $\mu$ как функции $p$. Действительно:
    %
    \begin{equation*}
        \dv{p}{\mu} = \rho = (q + 1) \frac{p}{\mu} \quad\Longleftrightarrow\quad
        \frac{1}{q + 1} \frac{\dd{p}}{p} = \frac{\dd{\mu}}{\mu} ,
    \end{equation*}
    %
    откуда
    %
    \begin{equation*}
        \mu = p^{\frac{1}{q + 1}} , \quad \rho = (q + 1) p^{\chi} , \quad \chi = \frac{q}{q + 1} .
    \end{equation*}

    Остальные уравнения переформулируем относительно $p$:
    %
    \begin{align*}
        \nabla_u \rho + \rho\ \nabla \cdot \vb{u} = 0 \quad&\Longleftrightarrow\quad
        \frac{\nabla_u p}{p} + \frac{1}{\chi} \nabla \cdot \vb{u} = 0 \\
        \vb{u} \nabla_u p + (p + \varepsilon)\ \nabla_u \vb{u} + \nabla p = 0 \quad&\Longleftrightarrow\quad
        \vb{u} \nabla_u p + \frac{1}{1-\chi} p\ \nabla_u \vb{u} + \nabla p = 0 \\
        \nabla_u \varepsilon + (p + \varepsilon)\ \nabla \cdot \vb{u} = 0 \quad&\Longleftrightarrow\quad
        \frac{\nabla_u p}{p} + \frac{1}{\chi} \nabla \cdot \vb{u} = 0
    \end{align*}
    %
    Отсюда видно, что два из оставшихся уравнений совпадают.

    Выразим $\nabla_u p$ из скалярного уравнения и подставим в векторное. После некоторых преобразований получим окончательно:
    %
    \begin{equation}
        \vb{u} \nabla \cdot \vb{u} - \frac{\chi}{1 - \chi} \nabla_u \vb{u} = \nabla \ln p^{\chi} .
    \end{equation}

    Таким образом, в сжимаемой жидкости давление связано с 4-скоростью одним векторным уравнением, а все остальные величины можно выразить через давление~--- лагранжиан жидкости.

\subsection{Учет резкой границы сферы}

    Уравнение связи в форме $\varepsilon = q p$ означает, что на границе жидкой сферы, где обязательно $p = 0$, $\varepsilon$ также необходимо обращается в нуль. Если при интегрировании исходного дифференциального уравнения связи учесть константу, т.е. считать $\varepsilon = q p + \varepsilon_0$, требование равенства нулю плотность энергии на границе сферы можно исключить.

    Проследим, к чему приведет учет $\varepsilon_0 \neq 0$. Во все уравнения давление входит либо в виде суммы $(\varepsilon + p)$, либо под знаком дифференциала (производной). Распишем $\varepsilon + p = (q + 1) p + \varepsilon_0 = (q + 1) (p + \epsilon) = (q + 1) \hat{p}$. Формальная замена переменных $p \mapsto \hat{p}$ (и, соответственно, $\dd{p} \mapsto \dd{\hat{p}}$) в исходных уравнениях переформулирует их относительно $\hat{p}$.

\onlyinsubfile{
    \clearpage
    \phantomsection
    \addcontentsline{toc}{section}{Список литературы}
    \bibliographystyle{\docroot/../lib/doc/bib/utf8gosttu}
    \bibliography{\docroot/../lib/doc/bib/math,\docroot/../lib/doc/bib/physics}
}

\end{document}
