\providecommand{\docroot}{..}
\documentclass[\docroot/reports/draft/report.tex]{subfiles}

\begin{document}

\onlyinsubfile{\tableofcontents}

\subsection{Лагранжиан и тензор энергии-импульса идеальной жидкости}

    Классический подход к изучению термодинамики идеальной жидкости в СТО и ОТО связан с тензором энергии-импульса жидкости:
    %
    \begin{equation*}
        T^f_{ab} = (\varepsilon + p) u_a u_b + p g_{ab} .
    \end{equation*}
    %
    Здесь $\varepsilon$~--- плотность энергии жидкости, $p$~--- ее давление, $u^a$~--- 4-вектор скорости. В работе [\ref{bib:schutz-velocity}] представлен лагранжев подход. Работа базируется на представлении четырехмерной скорости $u^a$ шестью скалярными потенциалами: $(\mu,\phi,\alpha,\beta,\theta,S)$. Уравнения динамики жидкости получаются вариацией лагранжиана по этим потенциалам.

    Не все потенциалы, однако, являются существенными при изучении простейших моделей жидких сфер~--- потенциалы, связанные с тепловыми явлениями ($\theta$, $S$), а также потенциалы, описывающие вихри поля скоростей ($\alpha$, $\beta$), могут быть исключены из рассмотрения. Так, из оригинальных шести потенциалов следует оставить два: энтальпию $\mu$ и скалярный потенциал $\phi$ (который далее мы обозначаем как $\sigma$). В итоге 4-скорость $u^a$ в своих ковариантных компонентах представляется в виде
    %
    \begin{equation*}
        u_a = \mu^{-1} \sigma_{,a} = \mu^{-1} v_a .
    \end{equation*}
    %
    Векторное поле $v^a$, следуя Шютцу [\ref{bib:schutz-velocity}] и цитированным там работам будем называть потоком.

    Согласно [\ref{bib:schutz-velocity}], основой действия жидкости является ее давление в параметрах $(\mu,S)$. Имея в виду введенные ранее границы модели, запишем:
    %
    \begin{equation}
        S_f = \int \mathcal{L}_f \sqrt{-g} \dd[4]{x} , \quad
        \mathcal{L} = p(\mu) .
    \end{equation}

    Энтальпия $\mu$ задается соотношением
    %
    \begin{equation*}
        \mu\rho = \varepsilon + p ,
    \end{equation*}
    %
    где $\rho$~--- плотность вещества жидкости. Показывается, что
    %
    \begin{equation*}
        \rho = p' \equiv \dv{p}{\mu} .
    \end{equation*}

    Из соотношения $u^a u_a = -1$ следует выражение для $\mu$:
    %
    \begin{equation*}
        \mu = \sqrt{-g^{ab} v_a v_b} .
    \end{equation*}

    Вариация лагранжиана по компонентам метрики $g^{ab}$ действительно дает тензор энергии-импульса жидкости:
    %
    \begin{equation}\begin{aligned}
        T^f_{ab} = \frac{-2}{\sqrt{-g}} \fdv{\qty(\sqrt{-g} \mathcal{L}_f)}{g^{ab}} &=
            \frac{-2}{\sqrt{-g}} \pdv{\qty(\sqrt{-g} \mathcal{L}_f)}{g^{ab}} \\ &=
            \frac{-2}{\sqrt{-g}} \qty(
                - \frac{1}{2 \mu} \sqrt{-g} p'(\mu) v_a v_b - \frac{1}{2} g_{ab} \sqrt{-g} p(\mu)
            ) \\ &=
            \frac{1}{\mu} p'(\mu) v_a v_b + p(\mu) g_{ab} \\ &=
            (p + \varepsilon) u_a u_b + p g_{ab} .
    \end{aligned}\end{equation}
    %
    Здесь использовано равенство
    %
    \begin{equation*}
        \fdv{\sqrt{-g}}{g^{ab}} = - \frac{1}{2} g_{ab} \sqrt{-g} .
    \end{equation*}

    \vspace{1cm}

    \textbf{\Large{References}}:
    %
    \begin{enumerate}
        \item Wikipedia: \enquote{Stress–energy tensor}~--- \url{https://en.wikipedia.org/wiki/Stress%E2%80%93energy_tensor}
        \item E. Minguzzi. \enquote{Inclusion of a perfect fluid term into the Einstein-Hilbert action}~--- \url{https://arxiv.org/abs/1606.00082}
        \item \label{bib:schutz-velocity} Bernard F. Schutz, Jr. \enquote{Perfect Fluids in General Relativity: Velocity Potentials and a Variational Principle}~--- \url{https://journals.aps.org/prd/abstract/10.1103/PhysRevD.2.2762}.
    \end{enumerate}

\subsection{Уравнения динамики идеальной жидкости}

    Уравнения динамики можно получить исходя из тождества Гильберта:
    %
    \begin{equation}
        T^a_{b;a} = 0 ,
    \end{equation}
    %
    которое применительно к $T^f$ записывается в виде:
    %
    \begin{equation*}
        \qty(T^f)^a_{b;a} = t_{,a} u^a u_b + t (u^a_{;a} u_b + u^a u_{b;a}) + p_{,a} \delta^a_b = 0,
    \end{equation*}
    %
    где для краткости введено $t = p + \varepsilon$.

    Возьмем проекцию полученного выражения на $u^b$. Для чего следует иметь в виду тождества:
    %
    \begin{gather*}
        u_b u^b = -1 , \\
        0 = \qty(u_b u^b)_{;a} = u_{b;a} u^b + u_b u^b_{;a}
          = u_{b;a} u_s g^{sb} + u_b u^b_{;a}
          = u^s_{;a} u_s + u_b u^b_{;a}
          = 2 u_b u^b_{;a}
          = 0 .
    \end{gather*}
    %
    Получаем
    %
    \begin{equation*}\begin{aligned}
        u^b \qty(T^f)^a_{b;a} &= - t_{,a} u^a - t u^a_{;a} + p_{,a} u^a \\
                              &= - \varepsilon_{,a} u^a - t u^a_{;a} = 0 .
    \end{aligned}\end{equation*}
    %
    Отсюда заключаем, что имеет место равенство
    %
    \begin{equation}
        \nabla_u \varepsilon + t\ \nabla \cdot \vb{u} = 0 ,
    \end{equation}
    %
    подставляя которое (в одной из предыдущих форм) в тождество Гильберта получаем
    %
    \begin{equation*}\begin{aligned}
        \qty(T^f)^a_{b;a} &= t_{,a} u^a u_b - t_{,a} u^a u_b + p_{,a} u^a u_b + t u^a u_{b;a} + p_{,b} \\
                          &= p_{,a} u^a u_b + t u^a u_{b;a} + p_{,b} = 0 ,
    \end{aligned}\end{equation*}
    %
    или,
    %
    \begin{equation}
        \vb{u} \nabla_u p + t \nabla_u \vb{u} + \nabla p = 0 .
    \end{equation}

    \vspace{1cm}

    \textbf{\Large{References}}:
    %
    \begin{enumerate}
        \item Wikipedia: \enquote{Stress–energy tensor}~--- \url{https://en.wikipedia.org/wiki/Stress%E2%80%93energy_tensor}
        \item E. Minguzzi. \enquote{Inclusion of a perfect fluid term into the Einstein-Hilbert action}~--- \url{https://arxiv.org/abs/1606.00082}
        \item \label{bib:schutz-velocity} Bernard F. Schutz, Jr. \enquote{Perfect Fluids in General Relativity: Velocity Potentials and a Variational Principle}~--- \url{https://journals.aps.org/prd/abstract/10.1103/PhysRevD.2.2762}.
    \end{enumerate}

\subsection{Уравнения состояния идеальной жидкости}

    Локальная скорость звука $a$ определяется зависимостью $p(\varepsilon)$:
    %
    \begin{equation*}
        \frac{a^2}{c^2} \equiv q^{-1} = \dv{p}{\varepsilon} .
    \end{equation*}
    %
    Отсюда при $q = const$
    %
    \begin{equation*}
        \varepsilon = q p .
    \end{equation*}
    %
    Из общей теории известно, что для ультрарелятивистской жидкости $q = 3$.

    Полагая $q = const$, из выражения $\varepsilon = \mu\rho - p$ можно получить соотношение, связывающее $p$, $\mu$ и $\rho$:
    %
    \begin{equation}
        p = \frac{\mu\rho}{q + 1} .
    \end{equation}

    Ранее вариацией действия жидкости были получены четыре уравнения движения, а также пятое уравнение, являющееся следствием первых четырех. Данные уравнения могут служить для определения четырех компонент векторного поля $u^\alpha$. Еще одно уравнение получим вариацией действия по потенциалу $\sigma$:
    %
    \begin{equation*}
        0 = \fdv{\qty(\sqrt{-g} \mathcal{L}_f)}{\sigma} =
            \pdv{\qty(\sqrt{-g} \mathcal{L}_f)}{\sigma} -
            \qty(\pdv{\qty(\sqrt{-g} \mathcal{L}_f)}{v_k})_{,k} .
    \end{equation*}
    %
    Первое слагаемое обращается в нуль. Второе дает:
    %
    \begin{equation*}\begin{aligned}
        \qty(\pdv{\qty(\sqrt{-g} \mathcal{L}_f)}{v_k})_{,k} &=
        \qty(\sqrt{-g} \pdv{\mathcal{L}_f}{v_k})_{,k} =
        \qty(\sqrt{-g} \pdv{p(\mu)}{v_k})_{,k} =
        \sqrt{-g}^{-1} \qty(\pdv{p(\mu)}{v_k})_{;k} = 0
    \end{aligned}\end{equation*}
    %
    Производная $p$ по $v_k$ дает
    %
    \begin{equation*}
        \pdv{p(\mu)}{v_k} = p' \frac{1}{2 \mu} \qty(- 2 g^{k\alpha} v_\alpha) = - \rho u^k ,
    \end{equation*}
    %
    где использовано $\dv*{p}{\mu} = \rho$ и $v^k = \mu u^k$. Окончательно имеем
    %
    \begin{equation*}
        \qty(\rho u^k)_{;k} = 0 ,
    \end{equation*}
    %
    или
    %
    \begin{equation*}
        \rho_{,k} u^k + \rho u^k_{;k} = 0 .
    \end{equation*}
    %
    В ранее принятой векторной нотации
    %
    \begin{equation}
        \nabla_u \rho + \rho\ \nabla \cdot \vb{u} = 0 .
    \end{equation}

\subsection{Собираем все вместе}

    Выпишем все уравнения для жидкости, которые мы имеем на данный момент:
    %
    \begin{gather}
        \rho = \dv{p}{\mu} \\
        p = \frac{\mu\rho}{q + 1} , \quad q = const \\
        \nabla_u \rho + \rho\ \nabla \cdot \vb{u} = 0 \\
        \vb{u} \nabla_u p + (p + \varepsilon)\ \nabla_u \vb{u} + \nabla p = 0 \\
        \nabla_u \varepsilon + (p + \varepsilon)\ \nabla \cdot \vb{u} = 0 .
    \end{gather}
    %
    Покомпонентно это восемь уравнений, которые должны однозначно определить шесть переменных: четыре компоненты $u^\alpha$ и две скалярные функции, $\rho$ и $\mu$. Таким образом, два из этих уравнений являются \enquote{лишними}. Давление $p$ остается неопределенной независимой величиной. Приступим к формулировке окончательных уравнений.

    Из первых двух уравнений можно выразить $\rho$ и $\mu$ как функции $p$. Действительно:
    %
    \begin{equation*}
        \dv{p}{\mu} = \rho = (q + 1) \frac{p}{\mu} \quad\Longleftrightarrow\quad
        \frac{1}{q + 1} \frac{\dd{p}}{p} = \frac{\dd{\mu}}{\mu} ,
    \end{equation*}
    %
    откуда
    %
    \begin{equation*}
        \mu = p^{\frac{1}{q + 1}} , \quad \rho = (q + 1) p^{\chi} , \quad \chi = \frac{q}{q + 1} .
    \end{equation*}

    Остальные уравнения переформулируем относительно $p$:
    %
    \begin{align*}
        \nabla_u \rho + \rho\ \nabla \cdot \vb{u} = 0 \quad&\Longleftrightarrow\quad
        \frac{\nabla_u p}{p} + \frac{1}{\chi} \nabla \cdot \vb{u} = 0 \\
        \vb{u} \nabla_u p + (p + \varepsilon)\ \nabla_u \vb{u} + \nabla p = 0 \quad&\Longleftrightarrow\quad
        \vb{u} \nabla_u p + (p + \varepsilon)\ \nabla_u \vb{u} + \nabla p = 0 \\
        \nabla_u \varepsilon + (p + \varepsilon)\ \nabla \cdot \vb{u} = 0 \quad&\Longleftrightarrow\quad
        \frac{\nabla_u p}{p} + \frac{1}{\chi} \nabla \cdot \vb{u} = 0
    \end{align*}
    %
    Отсюда видно, что два из оставшихся уравнений совпадают.

    Мы помним, что осталось неопределенным лишь 4-векторное поле $u^\alpha$. Векторное уравнение позволяет определить его, однако переменные $u^\alpha$ сильно завязаны с $p$. Оставшееся вспомогательное уравнение решает эту проблему. Выразим $\nabla_u p$ из скалярного уравнения и подставим в векторное. После некоторых преобразований получим:
    %
    \begin{equation}
        \vb{u} \nabla \cdot \vb{u} - \frac{\chi}{1 - \chi} \nabla_u \vb{u} = \nabla \ln p^{\chi} .
    \end{equation}
    %
    Следует решать лишь уравнение
    %
    \begin{equation}\label{eq:u-of-p}
        \vb{u} \nabla \cdot \vb{u} - \frac{\chi}{1 - \chi} \nabla_u \vb{u} = \nabla f , \quad f = \ln p^{\chi} .
    \end{equation}

\subsection{Решение уравнения \autoref{eq:u-of-p}}

    Рассмотрим частный случай решения для несжимаемой жидкости. Такая жидкость характеризуется постоянной плотностью $\rho = const$, постоянной плотностью энергии $\varepsilon = const$, а также бесконечной скоростью звука $a = \infty$, откуда следует $q = 0$, $\chi = 0$. Тогда из \autoref{eq:u-of-p} остается лишь
    %
    \begin{equation*}
        \vb{u} \nabla \cdot \vb{u} = 0 \quad\Longleftrightarrow\quad
        \vb{u} = 0 \quad\text{или}\quad \nabla \cdot \vb{u} = 0 .
    \end{equation*}
    %
    Первое решение тривиально. Второе в более привычной форме:
    %
    \begin{equation}\label{eq:u-of-p-incompressible}
        \Div\vb{u} = 0 ,
    \end{equation}
    %
    решение которого может быть получено заменой $\vb{u} = \nabla \varphi$. Уравнение, определяемое \autoref{eq:u-of-p-incompressible}, тогда сводится к уравнению Лапласа:
    %
    \begin{equation}\label{eq:phi-of-p-incompressible}
        \Delta^{(4)} \varphi = 0 .
    \end{equation}
    %
    Если метрика пространства $g^{\alpha\beta}$ удовлетворяет соотношениям $g^{00} = 1$, $g^{0i} = 0$, \autoref{eq:phi-of-p-incompressible} переходит в трехмерное скалярное волновое уравнение.

    (Ранее мы полагали $\vb{v} = \nabla \sigma$. Принимая во внимание $\mu \vb{u} = \vb{v}$ и $\mu = const$, получаем $\vb{u} = \mu^{-1} \Grad \sigma = \Grad \qty(\mu^{-1} \sigma)$, откуда $\varphi = \mu^{-1} \sigma$.)

    В обсуждаемых уравнениях метрика $g^{\alpha\beta}$ вообще говоря должна определяться из вариационного принципа:
    %
    \begin{equation*}
        \fdv{(S_f + S_g)}{g^{\alpha\beta}} = 0 .
    \end{equation*}
    %
    Уравнение, определяемое \autoref{eq:phi-of-p-incompressible}, является дополнительным к уравнениям Эйнштейна и должно решаться в совокупности с ними. Можно, однако, положить, что жидкость и гравитационное поле не взаимодействуют, и решить уравнение в рамках СТО, взяв в качестве метрики пространства метрику Минковского.

    В сферических координатах при фиксированном радиусе $R$ сферической капли жидкости общее решение \autoref{eq:phi-of-p-incompressible} записывается через $\{l,m,n\}$-гармоники:
    %
    \begin{equation*}
        \varphi_{lmn}(r,\theta,\phi) =
            j_l(\omega_{ln} r) Y_{lm}(\theta,\phi) \exp(-i \omega_{ln} t) , \quad
        \omega_{ln} = \text{$n$-й корень}\ \{\ j_l(\omega R) = 0\ \} .
    \end{equation*}

\subsection{Комментарии}

    \begin{enumerate}
        \item E. Minguzzi. \enquote{Inclusion of a perfect fluid term into the Einstein-Hilbert action}~--- \url{https://arxiv.org/abs/1606.00082}.

        Описывается сам лагранжиан, вывод вариационных уравнений (тензора энергии-импульса). Не подводится глубокого смысла под сам лагранжиан и под процесс вывода окончательных уравнений.

        \item J. David Brown. \enquote{Action functionals for relativistic perfect fluids}~--- \url{https://arxiv.org/abs/gr-qc/9304026v1}~--- P.31--33, Sec.6 \enquote{Other action functionals}.

        Рассматриваются действия в разных формах (разных зависимых/независимых переменных). Демонстрируются связи между различными формами действия. Упоминаются авторы, ответственные за эти формы. Используемое здесь действие приводится в разделе прочих форм и не рассматривается подробно.

        Из статьи видно, что существует множество форм действия жидкости. Выбор той или иной формы~--- дело удобства.

        \item Bernard F. Schutz, Jr. \enquote{Perfect Fluids in General Relativity: Velocity Potentials and a Variational Principle}~--- \url{https://journals.aps.org/prd/abstract/10.1103/PhysRevD.2.2762}.

        \item Bernard F. Schutz, Jr. \enquote{Hamiltonian Theory of a Relativistic Perfect Fluid}~--- \url{https://journals.aps.org/prd/abstract/10.1103/PhysRevD.4.3559}.

        Эти две статьи формулируют динамические уравнения жидкости в лагранжевой и гамильтоновой форме через т.н. потенциалы скоростей. Формализм потенциалов скоростей в термодинамике не является новым. Статьи обобщают классический формализм на случай общей теории относительности. Для формулировки гамильтонова подхода к описанию динамики жидкости используется (3+1)-разложение пространства-времени.

        \item Бурланков Д.Е., Самочадин А.М. \enquote{Гравитационные волны в жидкой сфере}.

        Статья, на которой базируется данная работа. Содержит выжимку информации по динамике жидкости. Используется действие в форме Шютца. Рассмотрен простейший случай~--- из шести потенциалов скоростей оставлен единственный.
    \end{enumerate}

\onlyinsubfile{
    \nocite{*}
    \clearpage
    \phantomsection
    \addcontentsline{toc}{section}{Список литературы}
    \bibliographystyle{\docroot/../lib/doc/bib/utf8gosttu}
    \bibliography{\docroot/../lib/doc/bib/math,\docroot/../lib/doc/bib/physics}
}

\end{document}
