\def\docroot{..}
\documentclass[\docroot/reports/draft/report.tex]{subfiles}

\begin{document}

\onlyinsubfile{\tableofcontents}

\subsection{Метрика Шварцшильда как фоновая метрика}

    \paragraph{Лагранжиан}. В качестве фоновой метрики можно выбрать не только неискривленную метрику Минковского, но и искривленную метрику Шварцшильда (другую статическую сферически симметричную метрику). При этом лагранжиан пространства, выраженный через $E$ и $B$, по форме останется неизменным.

    \paragraph{Бироторное уравнение}. Основное вариационное бироторное уравнение также не должно измениться, поскольку оно получено без дополнительных предположений о виде метрики.

    \paragraph{Метрические коэффициенты}. Поскольку метрика выражена в сферических координатах, однородный вид метрики может формально использовать те же коэффициенты, что и в случае метрики Минковского. Хотя и для метрики Шварцшильда можно получить коэффициенты, обращающую форму $h^{ij}h_{ij}$ в $\sum_{ij} h^2_{ij}$, оказывается, что чисто сферические коэффициенты даже удобнее, поскольку не вносят дополнительных подкоренных выражений в уравнения.

    \paragraph{Векторы Киллинга}. Пространство с метрикой Шварцшильда обладает теми же векторами Киллинга, потому уравнение на угловые части не меняется~--- решения в метрике Минковского точно также справедливы. Сохраняются также и свойства операторов повышения и понижения.

    \paragraph{Четность}. Свойства оператора четности, и, как следствие, операторов $\Opsr$ и $\Opbr$, сохраняются: $\Opsr$ переводит четную моду в нечетную, оператор $\Opbr$ не меняет четности моды.

\subsection{Радиальные функции}

    Радиальные функции можно получать как непосредственно из лагранжиана, так и из бироторного уравнения. При этом для нечетной моды получается четыре уравнения, два из которых должны оказаться тривиальными.

    На практике же оказывается, что после применения операции $\Opbr$ получается не симметричный тензор. Условие для его симметрии требует обращение в нуль $f_{13}$ и $f_{23}$, что приводит к нулевому решению.

    Для четной моды условие симметричности добавляет дополнительную связь между всеми компонентами тензора.

\onlyinsubfile{
    \nocite{*}
    \clearpage
    \phantomsection
    \addcontentsline{toc}{section}{Список литературы}
    \bibliographystyle{\docroot/../lib/doc/bib/utf8gosttu}
    \bibliography{\docroot/../lib/doc/bib/math,\docroot/../lib/doc/bib/physics}
}

\end{document}
