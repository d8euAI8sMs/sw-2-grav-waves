\def\docroot{..}
\documentclass[\docroot/reports/draft/report.tex]{subfiles}

\begin{document}

\onlyinsubfile{\tableofcontents}

\subsection{Решения Шварцшильда для свободного пространства}

    Шварцшильдом было найдено статическое решение уравнений Эйнштейна для сферически симметричного пространства.

    Основные предположения следующие:
    %
    \begin{enumerate}
        \item Стационарность метрики: $\pdv*{g_{\mu\nu}}{t} = 0$. Инвариантность метрики относительно обращения времени.
        \item Свободное пространство (вакуум, отсутствие материи).
        \item Сферическая симметрия пространства. Инвариантность метрики относительно поворотов и отражений.
    \end{enumerate}
    %
    Эти положения приводят [\ref{bib:wiki-schwarzschild}] к следующему виду метрики (соглашение $({}-{}+{}+{}+{})$, индексация с нуля):
    %
    \begin{equation*}
        g_{\mu\nu} = \text{diag}_{\mu\nu}\ \{\ B(r),\ A(r),\ r^2,\ r^2 \sin\theta \ \}
    \end{equation*}
    %
    Компоненты $g_{\theta\theta}$ и $g_{\phi\phi}$ являются компонентами обычной двумерной сферической метрики.

    Уравнения Эйнштейна в сделанных предположениях принимают вид:
    %
    \begin{equation*}
        G_{\mu\nu} \equiv R_{\mu\nu} - \frac{1}{2} R g_{\mu\nu} = 0 ,
    \end{equation*}
    %
    где
    %
    \begin{equation*}
        R = g^{\mu\nu} R_{\mu\nu} , \quad
        R_{\mu\nu} = R^\rho_{\mu\rho\nu} , \quad
        R^\rho_{\mu\sigma\nu} = \Gamma^\rho_{\nu\sigma,\mu} - \Gamma^\rho_{\mu\sigma,\nu} + \Gamma^\rho_{\mu\lambda}\Gamma^\lambda_{\nu\sigma} - \Gamma^\rho_{\nu\lambda}\Gamma^\lambda_{\mu\sigma} .
    \end{equation*}
    %
    Процесс решения основывается на вычислении $\Gamma^\rho_{\mu\nu}$ по заданному $g_{\mu\nu}$ и подстановке результата в уравнение Эйнштейна. Результат данной подстановки~--- три нетривиальных дифференциальных уравнения на две неизвестные компоненты метрики.

    После алгебраических преобразований, разделения переменных и решения несложных дифференциальных уравнений можно заключить, что
    %
    \begin{equation*}
        B(r) = \frac{K}{A(r)} , \quad A(r) = \qty(1 + \frac{1}{r S})^{-1} ,
    \end{equation*}
    %
    где $K$ и $S$~--- некие константы. Они определяются из приближения слабого гравитационного поля (линеаризованной теории). [TODO: соответствующий вывод]. Их значения:
    %
    \begin{equation*}
        K = -c^2, \quad \frac{1}{S} = - \frac{2 G m}{c^2} .
    \end{equation*}
    %
    Здесь $G$~--- гравитационная постоянная, $m$~--- масса источника излучения. Величина $r_s = - S^{-1}$ называется Шварцшильдовским радиусом объекта массы $m$.

    Метрика принимает вид:
    %
    \begin{equation*}
        g_{\mu\nu} = \text{diag}_{\mu\nu}\ \{\
            -c^2 \qty(1 - \frac{r_s}{r}),\
            \qty(1 - \frac{r_s}{r})^{-1},\
            r^2,\
            r^2 \sin\theta
        \ \}
    \end{equation*}
    %
    Особенность при $r = r_s$ не является физической, а является лишь следствием выбора системы координат [\ref{bib:wiki-schwarzschild}].

    \vspace{1cm}

    \textbf{\Large{References}}:
    %
    \begin{enumerate}
        \item \label{bib:wiki-schwarzschild} Wikipedia: \enquote{Deriving the Schwarzschild solution}~--- \url{https://en.wikipedia.org/wiki/Deriving_the_Schwarzschild_solution}
    \end{enumerate}

\subsection{Метрика в экспонентах}

    Более элегантный вид метрики получается, если записать ее через экспоненты:
    %
    \begin{equation*}
        g_{\mu\nu} = \text{diag}_{\mu\nu}\ \{\ -e^{B(r)},\ e^{A(r)},\ r^2,\ r^2 \sin\theta \ \} .
    \end{equation*}
    %
    В таком варианте записи явно виден тип используемого соглашения (space-positive).

\subsection{Решения Шварцшильда для несжимаемой жидкости}

    Второе решение, которое удалось получить Шварцшильду~--- статическое сферически симметричное решение (внутреннее и внешнее) для жидкой сферы. Жидкость предполагается несжимаемой.

    Метрика $g_{\mu\nu}$ выбирается в том же виде, что и в вакуумной задаче:
    %
    \begin{equation*}
        g_{\mu\nu} = \text{diag}_{\mu\nu}\ \{\ -e^{A(r)},\ e^{B(r)},\ r^2,\ r^2 \sin\theta \ \} .
    \end{equation*}
    %
    Уравнения Эйнштейна записываются в полном виде:
    %
    \begin{equation*}
        G_{\mu\nu} \equiv R_{\mu\nu} - \frac{1}{2} R g_{\mu\nu} = \kappa T^f_{\mu\nu} ,
    \end{equation*}
    %
    где $T^f_{\mu\nu}$~--- тензор энергии-импульса жидкости:
    %
    \begin{equation*}
        T^f_{\mu\nu} = (\varepsilon + p) u_\mu u_\nu + p g_{\mu\nu} .
    \end{equation*}
    %
    Здесь $\varepsilon = const$~--- плотность энергии жидкости, $p(r)$~--- давление жидкости, $u^\mu$~--- 4-вектор скорости.

    Получим выражение для 4-вектора скорости, учитывая сферическую симметрию и статичность задачи.
    %
    \begin{equation*}
        x^\mu = \{\ ct,\ r,\ \theta,\ \phi \ \}^\mu , \quad
        u^\mu = \dv{x^\mu}{s} , \quad
        \dd{s} = \sqrt{ g_{\alpha\beta} \dd{x^\alpha} \dd{x^\beta} }
               = \sqrt{ g_{\alpha\beta} \pdv{x^\alpha}{x^0} \pdv{x^\beta}{x^0} } \dd{x^0} .
    \end{equation*}
    %
    В силу сферической симметрии задачи от $t$ может зависеть только $r$. Однако, поскольку задача статическая, $\pdv*{r}{t}$ также обращается в нуль. Получаем
    %
    \begin{equation*}
        \dd{s} = \sqrt{ g_{00} } \dd{x^0} , \quad
        u^\mu = \dv{x^\mu}{s}
              = \frac{\dd{x^\mu}}{\sqrt{g_{00}} \dd{x^0}}
              = \frac{1}{\sqrt{g_{00}}} \delta^\mu_0 .
    \end{equation*}
    %
    Отсюда
    %
    \begin{equation*}
        u_\nu = g_{\mu\nu} u^\mu = \frac{1}{\sqrt{g_{00}}} g_{\nu 0}
    \end{equation*}
    %
    и
    %
    \begin{equation*}
        T^f_{\mu\nu} = - (\varepsilon + p) \frac{g_{\mu 0} g_{\nu 0}}{g_{00}} + p g_{\mu\nu}
                     = - (\varepsilon + p) g_{00} \delta^0_\mu \delta^0_\nu + p g_{\mu\nu} .
    \end{equation*}
    %
    Минус перед первым слагаемым является следствием того, что $g_{00} < 0$, следовательно $\sqrt{g_{00}} = i \sqrt{-g_{00}}$.

    Первое уравнение получается из тождества $T^{\mu\nu}_{;\nu} = 0$:
    %
    \begin{equation}
        (\varepsilon + p) A' + 2 p' = 0 ,
    \end{equation}
    %
    решение которого:
    %
    \begin{equation}
        e^{A} = \frac{\gamma}{(\varepsilon + p)^2} , \quad \gamma = const .
    \end{equation}

    \begin{footnotesize}
        \textbf{Замечания}:
        %
        \begin{enumerate}
            \item Следует быть внимательным при вычислении $T^{\mu\nu}_{;\nu}$. Если сразу брать тензор энергии-импульса в его окончательной форме, применяя обычные правила дифференцирования для ковариантной производной, можно прийти к неверному результату. Действительно, в соотношении $u^\mu = \qty(\sqrt{g_{00}})^{-1} \delta^\mu_0$ величина слева является вектором, в то время как справа стоит тензор. Дифференцируя таким образом тензор энергии-импульса, получим нулевой вектор.

            Правильный подход заключается в том, чтобы сначала раскрыть ковариантную производную в форме, когда тензор $T^{\mu\nu}$ записан через $u^\mu$, а только потом подставить окончательный вид тензора энергии-импульса в получившееся выражение.

            \item Такое же уравнение можно было получить из ранее полученного общего векторного уравнения для идеальной жидкости. По сути, оно следует из того же тождества Гильберта, но приведено к более короткой форме.
        \end{enumerate}
    \end{footnotesize}

    Из уравнений Эйнштейна нетривиальными являются три. Первое записывается в виде:
    %
    \begin{equation}
        e^B (1 - \kappa \varepsilon r^2) + r B' - 1 = 0 ,
    \end{equation}
    %
    решение которого,
    %
    \begin{equation*}
        e^{B} = \frac{r - \flatfrac{1}{3} \kappa \varepsilon r^3 - const}{r} ,
    \end{equation*}
    %
    не является сингулярным в точке $r = 0$ только при $const = 0$. Итак,
    %
    \begin{equation*}
        e^{B} = 1 - \flatfrac{1}{3} \kappa \varepsilon r^2 .
    \end{equation*}
    %
    Второе уравнение при подстановке ранее полученных функций принимает вид:
    %
    \begin{equation}
        \frac{\kappa r (\varepsilon + 3 p)}{\kappa \varepsilon r^2 - 3} - \frac{2 p'}{\varepsilon + p} = 0 .
    \end{equation}
    %
    Его решение:
    %
    \begin{equation}
        p = \varepsilon \frac{1 - e^{2 \varepsilon C} \sqrt{\kappa \varepsilon r^2 - 3}}{e^{2 \varepsilon C} \sqrt{\kappa \varepsilon r^2 - 3} - 3} .
    \end{equation}
    %
    Последнее (третье) уравнение Эйнштейна после подстановки $A$, $B$, $p$ выполняется автоматически.

    Константа $C$ в последнем уравнении определяется из условия $p(R) = 0$. Окончательно,
    %
    \begin{equation}
        p = \varepsilon \frac{ \sqrt{\kappa \varepsilon R^2 - 3} - \sqrt{\kappa \varepsilon r^2 - 3}}{
            \sqrt{\kappa \varepsilon r^2 - 3} - 3 \sqrt{\kappa \varepsilon R^2 - 3}} .
    \end{equation}

    Константа $\gamma$ должна определиться из условий сшивки с метрикой свободного пространства на границе сферы (т.е. при $r = R$). Масса сферы определяется интегралом
    %
    \begin{equation*}
        m = \int\limits_0^R 4 \pi r^2 \rho \dd{r} = \qty( \rho = const ) = \frac{4}{3} \pi R^3 \rho .
    \end{equation*}
    %
    Сама же плотность жидкости $\rho = \flatfrac{\varepsilon}{c^2}$. [TODO: описать, почему так предполагаем].

    Отсюда заключаем, что
    %
    \begin{equation}
        \gamma = \frac{1}{3} c^2 \varepsilon^2 (3 - \kappa \varepsilon R^2) .
    \end{equation}

    Окончательно имеем:
    %
    \begin{equation}
        g_{\mu\nu} = \text{diag}_{\mu\nu}\ \{\
            \frac{c^2}{12} \qty(\sqrt{\kappa \varepsilon r^2 - 3} - 3 \sqrt{\kappa \varepsilon R^2 - 3})^2,\
            \frac{3}{3 - \kappa \varepsilon r^2},\
            r^2,\
            r^2 \sin\theta
        \ \}
    \end{equation}

    В таком виде решение не является, однако, математически однородным. Вынося $i$ из под корня, пользуясь определением Шварцшильдовского радиуса $r_s$, массы $m$, плотности жидкости $\rho$ и вводя параметр
    %
    \begin{equation*}
        \mathcal{R} = \frac{R^3}{r_s} ,
    \end{equation*}
    %
    можно придти к более компактному результату:
    %
    \begin{equation}
        g_{\mu\nu} = \text{diag}_{\mu\nu}\ \{\
            - \frac{c^2}{4} \qty(\sqrt{1 - \frac{r^2}{\mathcal{R}^2}} - 3 \sqrt{1 - \frac{R^2}{\mathcal{R}^2}})^2,\
            \qty(1 - \frac{r^2}{\mathcal{R}^2})^{-1},\
            r^2,\
            r^2 \sin\theta
        \ \}
    \end{equation}

\subsection{Решение Шварцшильда для свободного пространства через лагранжиан Гильберта}

    Выберем метрику в той же форме, что и ранее. В этой метрике плотность действия Гильберта $L = \flatfrac{1}{2 \kappa} R \sqrt{-g}$ (после исключения высших производных и деления на константу $\kappa$) запишется в виде:
    %
    \begin{equation*}
        L = e^{\frac{1}{2} (A - B)} \qty(e^B + r B' - 1) .
    \end{equation*}

    Вариации лагранжиана по $A$ и $B$ дают, соответственно,
    %
    \begin{equation*}
        e^B + r B' - 1 = 0 , \qquad e^B - r A' - 1 = 0 .
    \end{equation*}
    %
    Первое уравнение решается аналитически, для решения второго достаточно вычесть одно уравнение из другого. В итоге имеем пару уравнений,
    %
    \begin{equation*}
        \qquad A' + B' = 0 , \qquad e^B + r B' - 1 = 0
    \end{equation*}
    %
    решение которых:
    %
    \begin{equation*}
        A = C_1 - B , \qquad B = - \log{\frac{e^{C_2}}{r} + 1} .
    \end{equation*}

    Подставляя полученное решение в обратную метрику $g^{\mu\nu}$, получим решение с пока еще не определенными константами:
    %
    \begin{equation*}
        g^{\mu\nu} = \text{diag}^{\mu\nu}\ \{\
            -\frac{e^{-C_1} r}{e^{C_2} + r},\
            \frac{e^{C_2} + r}{r},\
            r^{-2},\
            \qty(r^2 \sin^2\theta)^{-1}
        \ \} .
    \end{equation*}
    %
    Константа $e^{C_1} = c^2$ определяется из требования $g^{\mu\nu} \to \gamma^{\mu\nu}$ при $r \to \infty$, где $\gamma^{\mu\nu}$~--- метрика Минковского. Теперь можно записать окончательный вид $g_{\mu\nu}$:
    %
    \begin{equation*}
        g_{\mu\nu} = \text{diag}_{\mu\nu}\ \{\
            - c^2 \qty(1 - \frac{r_s}{r}),\
            \qty(1 - \frac{r_s}{r})^{-1},\
            r^2,\
            r^2 \sin^2\theta
        \ \} , \qquad r_s = - e^{C_2} = \frac{2 G m}{c^2} .
    \end{equation*}
    %
    Здесь $r_s$~--- Шварцшильдовский радиус~--- определяется из приближения слабого гравитационного поля.

\subsection{Внутреннее решение Шварцшильда для жидкой несжимаемой сферы через лагранжиан Гильберта}

    Проделаем аналогичные предыдущему пункту шаги, однако теперь действие $\mathcal{L} = \mathcal{L}_g + \mathcal{L}_f$ состоит из двух слагаемых: лагранжиана пространства $\mathcal{L}_g = \flatfrac{1}{2 \kappa} R$ и лагранжиана идеальной жидкости $\mathcal{L}_f = p$.

    Ранее был получен краткий вид лагранжиана пространства в метрике Шварцшильда:
    %
    \begin{equation*}
        L_g = \frac{1}{\kappa} e^{\frac{1}{2} (A - B)} \qty(e^B + r B' - 1) .
    \end{equation*}
    %
    Лагранжиан жидкости получим из уравнения динамики жидкости
    %
    \begin{equation*}
        \vb{u} \nabla_u p + (\varepsilon + p) \nabla_u \vb{u} + \nabla p = 0 .
    \end{equation*}
    %
    Здесь мы, следуя Шварцшильду, полагаем жидкость несжимаемой, следовательно $\varepsilon = \rho c^2 = const$.
    %
    Выражение для $\vb{u}$ в шварцшильдовской метрике было получено ранее. Окончательно имеем уравнение
    %
    \begin{equation*}
        \frac{1}{2} (\varepsilon + p) A' + p' = 0 ,
    \end{equation*}
    %
    решение которого
    %
    \begin{equation*}
        p = \gamma e^{-\frac{A}{2}} - \varepsilon \equiv \mathcal{L}_f
    \end{equation*}
    %
    и является искомым лагранжианом жидкости.

    Вариации $L$ по неизвестным функциям $A$ и $B$ дают два уравнения, первое из которых,
    %
    \begin{equation*}
        e^B (1 - \varepsilon r^2 \kappa) + r B' - 1 = 0 ,
    \end{equation*}
    %
    решается независимо:
    %
    \begin{equation*}
        B = - \log\qty(1 - \frac{1}{3} \varepsilon r^2 \kappa - \frac{const}{r}) .
    \end{equation*}
    %
    Выбор константы $const = 0$ следует из того, что при $r \to 0$ сингулярность в решении должна отсутствовать. Окончательно,
    %
    \begin{equation*}
        B = - \log\qty(1 - \frac{1}{3} \varepsilon r^2 \kappa) .
    \end{equation*}

    Второе вариационное уравнение можно упростить, переформулировав его относительно $p$. Применив замену
    %
    \begin{equation*}
        A = 2 \log{\frac{\gamma}{\varepsilon + p}} ,
    \end{equation*}
    %
    получим уравнение
    %
    \begin{equation*}
        \kappa r (\varepsilon + p) (\varepsilon + 3 p) - 2 (\kappa \varepsilon r^2 - 3) p' = 0 ,
    \end{equation*}
    %
    которое также решается аналитически:
    %
    \begin{equation*}
        p = \varepsilon \frac{1 - e^{2 \varepsilon C} \sqrt{\kappa \varepsilon r^2 - 3}}{e^{2 \varepsilon C} \sqrt{\kappa \varepsilon r^2 - 3} - 3} .
    \end{equation*}

    Дальнейшая процедура определения констант совпадает с описанной ранее для решения через уравнения Эйнштейна.

\onlyinsubfile{
    \nocite{*}
    \clearpage
    \phantomsection
    \addcontentsline{toc}{section}{Список литературы}
    \bibliographystyle{\docroot/../lib/doc/bib/utf8gosttu}
    \bibliography{\docroot/../lib/doc/bib/math,\docroot/../lib/doc/bib/physics}
}

\end{document}
