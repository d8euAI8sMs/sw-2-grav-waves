\providecommand{\docroot}{..}
\documentclass[\docroot/reports/draft/report.tex]{subfiles}

\begin{document}

\onlyinsubfile{\tableofcontents}

\subsection{Достаточная часть действия Гильберта}

    Известно, что действие Гильберта~--- четырехмерная скалярная кривизна $R$:
    %
    \begin{equation*}
        S = \int_\Omega R \sqrt{-g} \dd[4]{x} , \quad R \sqrt{-g} \equiv L
    \end{equation*}
    %
    Можно показать [\ref{bib:se-r-decomposition}, \ref{bib:landau-t2} и др.], что плотность действия $L$ раскладывается на две части, одна из которых~--- полная дивергенция:
    %
    \begin{equation*}
        R = g^{\mu\nu} \qty(
            \Gamma^\lambda_{\lambda k}\Gamma^k_{\mu\nu} -
            \Gamma^\lambda_{\mu k}\Gamma^k_{\lambda\nu}
            ) + \frac{1}{\sqrt{-g}} \qty(
                \sqrt{-g} \qty(g^{\mu\nu}g^{\lambda k} - g^{\mu\lambda}g^{\nu k}) g_{\lambda k, \nu}
            )_{,\mu} = R_\text{quad} + \qty(\sqrt{-g})^{-1} \mathfrak{R}_\text{surf}.
    \end{equation*}
    %
    Отсюда
    %
    \begin{equation*}
        L = \sqrt{-g} R_\text{quad} + \mathfrak{R}_\text{surf} = L_\text{quad} + L_\text{surf} .
    \end{equation*}

    Величина $L_\text{quad}$ не является, однако, скаляром. Иначе, она зависит от системы координат. $L$ же является скаляром.

    Нерешенный вопрос, \textit{исчезает ли $L_\text{surf}$ при интегрировании в действии}. Наивные попытки выяснить это \enquote{на пальцах} завершились неудачей\footnotemark{}~--- ответ отрицательный. В [\ref{bib:se-r-decomposition}] автор также тщетно ищет ответ на этот вопрос. В [\ref{bib:landau-t2}] этот момент также не разъясняется. Из логических соображений можно заключить лишь, что формальное разложение возможно, но в конкретной системе координат следует использовать общее, не зависящее от системы координат выражение.

    \footnotetext{
        Численно автором было показано, что соотношение $R = R_\text{quad} + \qty(\sqrt{-g})^{-1} \mathfrak{R}_\text{surf}$ верно, однако в реальных задачах вариации $L_\text{surf}$ не нулевые, а вариации $L_\text{quad}$ не дают решения задачи.
    }

    \vspace{1cm}

    \textbf{\Large{References}}:
    %
    \begin{enumerate}
        \item \label{bib:se-r-decomposition} StackExchange: \enquote{Variation of the Einstein-Hilbert action in D dimensions without the Gibbons-Hawking-York term}~--- \url{https://physics.stackexchange.com/questions/218060}
        \item \label{bib:landau-t2} Ландау. Лифшиц: \enquote{Теоретическая физика. Т.2. Теория поля}
    \end{enumerate}

\onlyinsubfile{
    \nocite{*}
    \clearpage
    \phantomsection
    \addcontentsline{toc}{section}{Список литературы}
    \bibliographystyle{\docroot/../lib/doc/bib/utf8gosttu}
    \bibliography{\docroot/../lib/doc/bib/math,\docroot/../lib/doc/bib/physics}
}

\end{document}
