\providecommand{\docroot}{..}
\documentclass[\docroot/reports/draft/report.tex]{subfiles}

\begin{document}

\onlyinsubfile{\tableofcontents}

\subsection{Общий вид метрики в ТГВ}

    Общий вид метрики $g_{\mu\nu}$ в ТГВ схож с видом метрики в АДМ-представлении:
    %
    \begin{equation*}
        g_{\mu\nu} = \begin{pmatrix}
            1 - g_{ij} V^i V^j   &   g_{ij} V^i \\
            g_{ij} V^j           &   - g_{ij}   \\
        \end{pmatrix} , \quad
        g^{\mu\nu} = \begin{pmatrix}
            1   & V^j                \\
            V^i & V^i V^j - g^{ij}   \\
        \end{pmatrix} .
    \end{equation*}
    %
    Отличием является равенство единице функции хода. Выражение для 4-интервала:
    %
    \begin{equation*}
        \dd{s}^2 = \qty(1 - g_{ij} V^i V^j) \dd{t}^2 + 2 g_{ij} V^i \dd{x^j} \dd{t} - g_{ij} \dd{x^i}\dd{x^j} .
    \end{equation*}

\subsection{Решение ТГВ для свободного сферически симметричного пространства}

    Наиболее общий вид статической сферически симметричной метрики в ТГВ:
    %
    \begin{equation*}
        g_{ij} = \text{diag} \begin{pmatrix}B^{-1}, & r^2, & r^2 \sin\theta\end{pmatrix} , \quad
        V^i = \begin{pmatrix}V(r), & 0, & 0\end{pmatrix} .
    \end{equation*}
    %
    При этом 4-интервал имеет вид
    %
    \begin{equation*}
        \dd{s}^2 = \qty(1 - \frac{V^2}{B}) \dd{t}^2 + 2 \frac{V}{B} \dd{t} \dd{r} - \frac{1}{B^2} \dd{r}^2 - r^2 \dd{\Omega}^2 , \quad \dd{\Omega} = \dd{\theta}^2 + \sin^2{\theta} \dd{\phi}^2 .
    \end{equation*}

    Вид интервала математически существенно неоднороден. Также усложняется вид лагранжиана и вариационных уравнений. Можно привести интервал к симметричному виду
    %
    \begin{equation*}
        \dd{s}^2 = \qty(1 - V^2) \dd{t}^2 + 2 \frac{V}{B} \dd{t} \dd{r} - \frac{1}{B^2} \dd{r}^2 - r^2 \dd{\Omega}^2 , \quad \dd{\Omega} = \dd{\theta}^2 + \sin^2{\theta} \dd{\phi}^2
    \end{equation*}
    %
    с метрикой в виде
    %
    \begin{equation*}
        g_{ij} = \text{diag} \begin{pmatrix}B^{-2}, & r^2, & r^2 \sin\theta\end{pmatrix} , \quad
        V^i = \begin{pmatrix}B(r)V(r), & 0, & 0\end{pmatrix} .
    \end{equation*}

    Лагранжиан пространства $\mathcal{L} = R^{(4)} \sqrt{-g}$ после прямого вычисления оказывается довольно громоздким. Его можно упростить, устранив члены, не влияющие на уравнения динамики. Введем замену переменных
    %
    \begin{equation*}
        \qty(V^2(r) - 1) \mapsto U(r) , \quad U(r) \mapsto r^{-2} u(r) .
    \end{equation*}
    %
    После соответствующей замены лагранжиан примет вид
    %
    \begin{equation*}
        - \mathcal{L} = \frac{2}{B} + \frac{B' \qty(2 u + r u')}{r} + B u'' .
    \end{equation*}
    %
    Пользуясь тем, что к лагранжиану можно добавить полную производную, заменим последнее слагаемое на $\qty(B u')' - B'u'$. Применяя обратную замену $r^{-2} u \mapsto U$ и опуская член $\qty(B u')'$, получим окончательно
    %
    \begin{equation*}
        - \frac{1}{2} \mathcal{L} = \frac{1}{B} + r U B' .
    \end{equation*}

    Вариации лагранжиана по $B$ и $U$ дают уравнения:
    %
    \begin{equation*}
        B' = 0 , \quad 1 + B^2 (U + r U') = 0 .
    \end{equation*}
    %
    Из первого уравнения следует, что $B = const$. Второе дает:
    %
    \begin{equation*}
        U = \frac{C}{r} - \frac{1}{B^2} .
    \end{equation*}

    Константы $B$ и $C$ определятся из условий на бесконечности и условий сшивки с классической теорией. Первое заключается в том, что при $r \to \infty$ должно выполняться $g \to \gamma$, где $\gamma$~--- метрика Минковского. Отсюда получаем, что $B = 1$. Второе можно получить, усматривая эквивалентность движения тела в поле скоростей и его свободного падения в гравитационном потенциале. В обоих случаях движение тела будет одинаково, поэтому можно записать равенство кинетической и потенциальной энергий:
    %
    \begin{equation*}
        \frac{\qty|V|^2}{2} = \frac{G M}{r} ,
    \end{equation*}
    %
    $M$~--- масса источника поля, $G$~--- гравитационная постоянная Ньютона. Отсюда $C = \sqrt{2 G M}$.

    Окончательно имеем:
    %
    \begin{equation*}
        B = 1, \quad V^2 = \frac{2 G M}{r}
    \end{equation*}
    %
    \begin{equation*}
        g_{\mu\nu} = \begin{pmatrix}
            1 - \frac{2 G M}{r} & \sqrt{\frac{2 G M}{r}} & 0 & 0 \\
            \sqrt{\frac{2 G M}{r}} & -1 & 0 & 0 \\
            0 & 0 & -r^2 & 0 \\
            0 & 0 & 0 & r^2 \sin\theta \\
        \end{pmatrix} .
    \end{equation*}

\subsection{Решение ТГВ для идеальной жидкой самогравитирующей сферы}

    Будем рассматривать метрику в той же форме (но в этот раз противоположной сигнатуры). При этом будем учитывать взаимодействие гравитационного поля с идеальной жидкостью. Внешним решением будет выступать ранее полученное решение для свободного пространства. Внутреннее нам предстоит получить.

    Выражение для коэффициентов метрики:
    %
    \begin{equation*}
        g_{ij} = \text{diag} \begin{pmatrix}B^{-2}, & r^2, & r^2 \sin\theta\end{pmatrix} , \quad
        V^i = \begin{pmatrix}B(r)V(r), & 0, & 0\end{pmatrix} , \quad
        V(r) = \sqrt{1 - U(r)} .
    \end{equation*}

    Суммарный лагранжиан системы $\mathcal{L}$ будет складываться из ранее полученного лагранжиана пространства-времени $\mathcal{L}_g$ и лагранжиана идеальной жидкости $\mathcal{L}_f = p(\mu)$, где $p$~--- давление жидкости, $\mu$~--- энтальпия.

    Из уравнения состояния жидкости,
    %
    \begin{equation*}
        \vb{u} \nabla \cdot \vb{u} - \frac{\chi}{1 - \chi} \nabla_u \vb{u} = \nabla f , \quad f = \ln p^{\chi} ,
    \end{equation*}
    %
    где $\vb{u}$~--- 4-вектор скорости, $\chi = \flatfrac{q}{(q+1)}$, $q = \flatfrac{v^2}{c^2}$~--- скоростной фактор жидкости, можно получить выражение для $p$ через коэффициенты метрики.

    В силу статичности задачи и ее сферической симметрии $\vb{u} = \begin{pmatrix}u^0,&0,&0,&0\end{pmatrix}$:
    %
    \begin{gather*}
        \dd{\tau}^2 = - g_{\mu\nu} \dd{x^\mu}\dd{x^\nu} = - g_{\mu\nu} \dv{x^\mu}{t} \dv{x^\nu}{t} \dd{t}^2 = - g_{\mu\nu} \delta^\mu_0 \delta^\nu_0 \dd{t}^2 = - g_{00} \dd{t}^2 , \\
        u^\alpha = \dv{x^\alpha}{\tau} = \dv{x^\alpha}{t} \dv{t}{\tau} = \delta^\alpha_0 \dv{t}{\tau} = \frac{1}{\sqrt{- g_{00}}} \delta^\alpha_0 , \qquad - g_{00} = U .
    \end{gather*}

    Учитывая вышесказанное, получаем из уравнения состояния дифференциальное уравнение на $p$:
    %
    \begin{equation*}
        \frac{1}{2 (1 - \chi)} \frac{\dd{U}}{U} + \frac{\dd{p}}{p} = 0 ,
    \end{equation*}
    %
    решение которого при $q = 3$, что соответствует случаю ультрарелятивистского предела, имеет вид:
    %
    \begin{equation*}
        p = \frac{c_0}{U^2} .
    \end{equation*}
    %
    Ранее обсуждалось, что для учета ненулевой плотности энергии $\epsilon$ границы сферы достаточно заменить $p \mapsto P \equiv (p + \epsilon)$ в уравнении состояния, поэтому окончательно имеем
    %
    \begin{equation*}
        \mathcal{L}_f \equiv p = \frac{c_0}{U^2} - \epsilon .
    \end{equation*}

    Выбор $q = 3$ кажется наиболее реалистичным, поскольку при высоких давлениях (в нейтронных звездах и других массивных объектах) вещество должно переходить в ультрарелятивистский газ \cite{oppenheimer_volkoff,burlankov_new_phys} с $q = 3$ и $\epsilon = 0$.

    Неизвестные функции $U(r)$ и $B(r)$ определяются из вариационных уравнений:
    %
    \begin{align*}
        \fdv{(\mathcal{L}\sqrt{-g})}{B} = 0 \quad\Longrightarrow\quad&
            r \epsilon - \frac{1}{r} - \frac{r c_0}{U^2} + \frac{B^2 U}{r} + B^2 U' = 0 \\
        \fdv{(\mathcal{L}\sqrt{-g})}{U} = 0 \quad\Longrightarrow\quad&
            \frac{2 r c_0}{B U^3} + B' = 0
    \end{align*}

    Коэффициент метрики $g^{rr} = B^2 U$. В то же время в метрике свободного пространства $\gamma^{rr} = 1 - \flatfrac{2 G M}{r}$. Это подталкивает нас к замене $1 - \flatfrac{m(r)}{r} = B^2 U$, где $m(r)$~--- некоторая функция, значение которой при $r = R$ пропорционально массе жидкой сферы.

    Введением функций
    %
    \begin{equation*}
        m = r (1 - B^2 U) , \quad w = r^2 U^2
    \end{equation*}
    %
    полученные уравнения можно переформулировать:
    %
    \begin{align*}
        m'(r) &= r^2 \epsilon + \frac{3 r^4 c_0}{w(r)} , \\
        \frac{w'(r)}{2} &= \frac{r^4 c_0 + (1 - r^2 \epsilon) w(r)}{r - m(r)} .
    \end{align*}

    Проведем некоторый анализ полученного решения. Во-первых, обратная метрика в неизвестных $(B,m)$ имеет вид:
    %
    \begin{equation*}
        g^{0r} = - \sqrt{B^2(r) - 1 + \frac{m(r)}{r}} , \qquad
        g^{rr} = 1 - \frac{m(r)}{r} .
    \end{equation*}
    %
    Условия сшивки непосредственно определяют $B(R) = 1$.

    Во-вторых, условие обращения в нуль давления жидкости на границе сферы $p(R) = 0$ дает также
    %
    \begin{equation*}
        U(R) = \sqrt{\frac{c_0}{\epsilon}} .
    \end{equation*}
    %
    Отсюда, в частности, видно, что случай $\epsilon = 0$ несостоятелен: требования конечности массы $m(R) \sim M$, конечности $B(R) = 1$ и обращения в нуль давления $p(R) = 0$ (что влечет $U(R) \to \infty$) не могут быть удовлетворены одновременно. Таким образом, статическое решение для ультрарелятивисткого газа в принятой модели существовать может только в пределе.

\subsection{Решение ТГВ для идеальной шварцшильдовой жидкости}

    Выберем метрику ТГВ в том же виде, что и ранее. Однако будем полагать жидкость несжимаемой, т.е. $\varepsilon = const$, в то время как все еще $p = p(r)$.

    Лагранжиан пространства $\mathcal{L}_g$ останется тем же. Лагранжиан жидкости $\mathcal{L}_f$ получим из уравнения динамики жидкости:
    %
    \begin{equation*}
        \vb{u} \nabla_u p + (\varepsilon + p) \nabla_u \vb{u} + \nabla p = 0 ,
    \end{equation*}
    %
    откуда
    %
    \begin{equation*}
        p = \frac{\gamma}{\sqrt{V}} - \varepsilon \equiv \mathcal{L}_f .
    \end{equation*}

    Вариации суммарного лагранжиана дают два уравнения:
    %
    \begin{gather*}
        \varepsilon r - \frac{1}{r} - \frac{\gamma r}{\sqrt{V}} + \frac{B^2 V}{r} + B^2 V' = 0 , \\
        \frac{\gamma r}{2 B V \sqrt{V}} + B' = 0 .
    \end{gather*}

    Уравнения достаточно сложно завязаны. Связность их можно уменьшить с помощью замены переменных:
    %
    \begin{equation*}
        V = \frac{1}{r w} , \qquad B = \sqrt{b w} .
    \end{equation*}
    %
    Применяя замену в отношении обратной метрики $g^{\mu\nu}$, из необходимости $g^{rr} > 0$ можно усмотреть необходимость $b(r) > 0$. Требование действительности коэффициентов метрики также определяет $w(r) > 0$.

    Вариационные уравнения с применением указанной замены по отдельности остаются завязанными, однако их разность дает простое уравнение:
    %
    \begin{equation*}
        b' = 1 - \varepsilon r^2 ,
    \end{equation*}
    %
    откуда сразу же
    %
    \begin{equation*}
        b = r \qty(1 - \frac{\varepsilon r^2}{3}) + const .
    \end{equation*}
    %
    Константа $const = 0$, поскольку иначе нарушается требование $g^{rr} \neq \infty$ в центре сферы. Из необходимости $b(r) > 0$ вытекает важное условие на $r$:
    %
    \begin{equation*}
        r < \frac{\varepsilon}{\sqrt{3}} ,
    \end{equation*}
    %
    которое определяет при $r \to R$ критический радиус при заданной массе звезды (поскольку $\varepsilon = \flatfrac{\rho}{c^2} \sim M$).

    Подстановка решения $b(r)$ в вариационные уравнения дает уравнение на $w(r)$:
    %
    \begin{equation*}
        (1 - \varepsilon r^2) w + \gamma r^\frac{5}{2} w^\frac{3}{2} + \frac{r}{3} (3 - \varepsilon r^2) w' = 0 .
    \end{equation*}
    %
    Его решение:
    %
    \begin{equation*}
        w(r) = - \frac{4 \varepsilon^2}{
            r \qty(
                12 \sqrt{C} \varepsilon \gamma \sqrt{3 - \varepsilon r^2} - 4 C \varepsilon^2 (3 - \varepsilon r^2) - 9 \gamma^2
            )
        } .
    \end{equation*}
    %
    Константа $C$ сугубо положительна.

    Давление $p$ определяется лишь функцией $w(r)$. Имеем:
    %
    \begin{equation*}
        p(r) = \gamma \sqrt{r w} - \varepsilon .
    \end{equation*}
    %
    Условие $p(R) = 0$ определяет константу $\gamma$:
    %
    \begin{equation*}
        \gamma = 2 \sqrt{C} \varepsilon \sqrt{3 - \varepsilon R^2} .
    \end{equation*}
    %
    (Второе решение $\gamma_1 = \flatfrac{\gamma}{5}$ приводит к отрицательному давлению, его следует отбросить.)

    Дальнейшие выкладки существенно упрощаются, если ввести
    %
    \begin{align*}
        3 - \varepsilon r^2 &= 3 \sinh^2\chi , \\
        3 - \varepsilon R^2 &= 3 \sinh^2\xi .
    \end{align*}
    %
    Тогда
    %
    \begin{equation*}
        \gamma = 2 \sqrt{3} \sqrt{C} \varepsilon \sinh\xi ,
    \end{equation*}
    %
    а компонента $g^{rr}$~--- просто $\sinh^2\chi$.

    Константа $C$ определится из условий сшивки с метрикой свободного пространства: $g^{\mu\nu} = \gamma^{\mu\nu}$ на границе сферы. Компонента $\gamma^{0r}$ в новых переменных выглядит просто:
    %
    \begin{equation*}
        \gamma^{0r} = - \sqrt{1 - \sinh\chi} .
    \end{equation*}
    %
    Компонента $g^{0r}$:
    %
    \begin{equation*}
        g^{0r} = - \sqrt{
            \frac{1}{3 C (1 - 3 \cosh\chi \sinh\xi)^2} - \sinh\chi^2
        } .
    \end{equation*}
    %
    Сшивка дает $C = \flatfrac{1}{12}$. Тогда $\gamma = e \sinh\xi$.

    Учет констант определяет давление:
    %
    \begin{equation*}
        p(r) = \varepsilon \frac{\sinh\chi - \sinh\xi}{3\sinh\xi - \sinh\chi} .
    \end{equation*}
    %
    Следует отметить, что и числитель, и знаменатель полученного выражения всегда положительны, что обеспечивает положительность давления при $r < R$. Применяя обратную замену $(\chi,\xi) \mapsto (r,R)$, получаем
    %
    \begin{equation*}
        p(r) = \varepsilon \frac{\sqrt{3 - \varepsilon r^2} - \sqrt{3 - \varepsilon R^2}}{3 \sqrt{3 - \varepsilon R^2} - \sqrt{3 - \varepsilon R^2}} .
    \end{equation*}

    Мы пришли к такому же результату, что и при решении данной задачи в метрике Шварцшильда. Отличен лишь вид метрики, но не выражение для давления жидкости.

\subsection{Получение бироторных соотношений. Вариации $EB$-лагранжиана}

    Предметом многих перипетий является лагранжиан в трехмерной форме, выраженный через тензоры $E$, $B$ и поле скоростей $V$. Переход от \enquote{четырехмерного} квадратичного лагранжиана довольно сложный. Этой темы мы, возможно, коснемся в одной из последующих тем, если необходимость того встанет остро.

    \paragraph{Введение.}

    Сам лагранжиан окончательно дается следующим выражением:
    %
    \begin{equation}
        \mathcal{L} = \gamma^{ij}\gamma^{kl} (E_{ik}E_{jl} - E_{ij}E_{kl}) - B^i_j B_i^j .
    \end{equation}
    %
    Здесь все тензорные поля~--- трехмерные, в т.ч. фоновая метрика $\gamma$ есть ${}^{(3)}\gamma$, в отличие от ${}^{(4)}\gamma$:
    %
    \begin{equation}
        {}^{(4)}\gamma_{ij} = \begin{pmatrix}
            1 - {}^{(3)}\gamma_{ij} V^i V^j   & - {}^{(3)}\gamma_{ij} V^i \\
            - {}^{(3)}\gamma_{ij} V^j         & {}^{(3)}\gamma_{ij}
        \end{pmatrix} , \qquad
        {}^{(4)}\gamma^{ij} = \begin{pmatrix}
            - 1   & - V^j                           \\
            - V^i & {}^{(3)}\gamma^{ij} + V^i V^j
        \end{pmatrix} .
    \end{equation}
    %
    Отсюда видно, что ${}^{(3)}\gamma^{ij} \neq {}^{(4)}\gamma^{ij}$ (при $i,j > 0$).

    В приведенном выше лагранжиане поля $E_{ij}$ и $B^i_j$ задаются через поле возмущений
    %
    \begin{equation}
        {}^{(4)}h_{ij} = \begin{pmatrix}
            0    & - v_j          \\
            -v_i & {}^{(3)}h_{ij}
        \end{pmatrix} .
    \end{equation}
    %
    Через поле ${}^{(3)}h_{ij}$, а также поле возмущений абсолютных скоростей $v^i$, поля $E_{ij}$ и $B^i_j$ выражаются следующим образом:
    %
    \begin{equation}
        E_{ij} = D_t{h}_{ij} + v_{i;j} + v_{j;i}, \qquad
        B^i_j = \Opsr(h)^i_j = \varepsilon^{ikl} h_{jk;l} ,
    \end{equation}
    %
    где $\varepsilon^{ijk} = 0,\, \pm \flatfrac{1}{\sqrt{\gamma}}$, а под $D_t{h}_{ij}$ понимается инвариантная производная по времени:
    %
    \begin{gather}
        D_t{h}_{ij} =
            \dot{h}_{ij} - \delta_{V} h_{ij}\\
        -\delta_{V} h_{ij} =
            V^s h_{ij,s} + V^s_{,i} h_{sj} + V^s_{,j} h_{is} =
            V^s h_{ij;s} + V^s_{;i} h_{sj} + V^s_{;j} h_{is} .
    \end{gather}

    Поскольку поле $v^i$ является чисто калибровочным, далее будем работать в глобальной калибровке, где
    %
    \begin{equation}
        E_{ij} = D_t{h}_{ij} .
    \end{equation}

    \paragraph{Вариационные уравнения.}

    Имея это под рукой, приступим к выводу в общем виде вариационных уравнений. С математической точки зрения запись
    %
    \begin{equation*}
        0 = \fdv{\sqrt{\gamma}\mathcal{L}}{h_{\alpha\beta}} =
            \pdv{\sqrt{\gamma}\mathcal{L}}{h_{\alpha\beta}} -
            \qty(\pdv{\sqrt{\gamma}\mathcal{L}}{h_{\alpha\beta,\chi}})_{,\chi} -
            \qty(\pdv{\sqrt{\gamma}\mathcal{L}}{\dot{h}_{\alpha\beta}})_{,t}
    \end{equation*}
    %
    эквивалентна записи
    %
    \begin{equation*}
        0 =
            \pdv{\mathcal{L}}{h_{\alpha\beta}} -
            \qty(\pdv{\mathcal{L}}{h_{\alpha\beta;\chi}})_{;\chi} -
            \qty(\pdv{\mathcal{L}}{\dot{h}_{\alpha\beta}})_{,t} ,
    \end{equation*}
    %
    лишь только теперь $h_{\alpha\beta}$ и $h_{\alpha\beta;\chi}$ считаются независимыми переменными: $\pdv*{h_{\alpha\beta;\chi}}{h_{\alpha\beta}} = 0$. Нам остается лишь найти частные производные различных входящих сюда полей по $h_{\alpha\beta}$ и его производным.

    \begin{gather*}
        \pdv{E_{ij}}{\dot{h}_{\alpha\beta}} =
            \pdv{\qty(\dot{h}_{ij} + \dots)}{\dot{h}_{\alpha\beta}} =
            \delta_i^\alpha \delta_j^\beta + 0 , \qquad
        \pdv{B^i_j}{\dot{h}_{\alpha\beta}} = 0 , \\
        \pdv{E_{ij}}{h_{\alpha\beta;\chi}} =
            \pdv{\qty(V^s h_{ij;s} + \dots)}{h_{\alpha\beta;\chi}} =
            V^s \delta_i^\alpha \delta_j^\beta \delta_s^\chi , \qquad
        \pdv{B^i_j}{h_{\alpha\beta;\chi}} =
            \pdv{\qty(\varepsilon^{ipq}h_{jp;q})}{h_{\alpha\beta;\chi}} =
            \varepsilon^{ipq} \delta_j^\alpha \delta_p^\beta \delta_q^\chi , \\
        \pdv{E_{ij}}{h_{\alpha\beta}} =
            \pdv{\qty(V^s_{;i}h_{sj} + V^s_{;j}h_{is} + \dots)}{h_{\alpha\beta}} =
            V^s_{;i} \delta_s^\alpha \delta_j^\beta +
                V^s_{;j} \delta_i^\alpha \delta_s^\beta, \quad
        \pdv{B_{ij}}{h_{\alpha\beta}} = 0 .
    \end{gather*}

    Таким образом, окончательно вариация лагранжиана собирается из компонент:
    %
    \begin{gather*}
        \frac{1}{2}\pdv{\mathcal{L}}{h_{\alpha\beta}} =
            \gamma^{ij}\gamma^{kl} \qty[
                \qty(V^\alpha_{;i} \delta_k^\beta + V^\beta_{;k} \delta_i^\alpha) E_{jl} -
                \qty(V^\alpha_{;i} \delta_j^\beta + V^\beta_{;j} \delta_i^\alpha) E_{kl}
            ] \\ =
                E^{\alpha s} V^\beta_{_;s} + E^{\beta s} V^\alpha_{_;s} -
                \qty(V^{\alpha;\beta} + V^{\beta;\alpha}) E, \\
        \frac{1}{2}\qty(\pdv{\mathcal{L}}{\dot{h}_{\alpha\beta}})_{,t} =
            \qty(\gamma^{ij}\gamma^{kl} \qty[
                \delta_i^\alpha \delta_k^\beta E_{jl} -
                \delta_i^\alpha \delta_j^\beta E_{kl}
            ] )_{,t} =
            \qty(
                \gamma^{\alpha j}\gamma^{\beta l} E_{jl} -
                \gamma^{\alpha \beta}\gamma^{kl} E_{kl}
            )_{,t} \\ =
            \qty(
                E^{\alpha\beta} - \gamma^{\alpha\beta} E
            )_{,t} , \\
        \frac{1}{2}\qty(\pdv{\mathcal{L}_B}{h_{\alpha\beta;\chi}})_{;\chi} =
            -\qty(
                \varepsilon^{ipq} \delta_j^\alpha \delta_p^\beta \delta_q^\chi B^j_i
            )_{;\chi} =
            -\qty(
                \varepsilon^{i\beta\chi} B^\alpha_i
            )_{;\chi} =
            -\varepsilon^{i\beta\chi} B^\alpha_{i;\chi} =
            \varepsilon^{\beta i\chi} B^\alpha_{i;\chi} \\ =
            \gamma^{\alpha m} \varepsilon^{\beta i\chi} B_{mi;\chi} =
            \gamma^{\alpha m} \Opsr(B)^\beta_m =
            \Opsr(B)^{\beta\alpha} , \\
        \frac{1}{2}\qty(\pdv{\mathcal{L}_E}{h_{\alpha\beta;\chi}})_{;\chi} =
            \gamma^{ij}\gamma^{kl} \qty(
                V^s \delta_i^\alpha \delta_k^\beta \delta_s^\chi E_{jl} -
                V^s \delta_i^\alpha \delta_j^\beta \delta_s^\chi E_{kl}
            )_{;\chi} \\ =
            \qty(V^\chi \qty[
                E^{\alpha\beta} - \gamma^{\alpha\beta} E
            ])_{;\chi} =
            V^\chi_{;\chi} \qty(E^{\alpha\beta} - \gamma^{\alpha\beta} E) +
                V^\chi \qty(E^{\alpha\beta} - \gamma^{\alpha\beta} E)_{;\chi}.
    \end{gather*}

    Введем поле
    %
    \begin{equation}
        e_{ij} = E_{ij} - \gamma_{ij} \tr E .
    \end{equation}
    %
    Тогда в конечном счете получаем:
    %
    \begin{equation*}
        0 = \frac{1}{2}\fdv{\sqrt{\gamma}\mathcal{L}}{h_{\alpha\beta}} =
            \qty(e^{\alpha s} V^\beta_{_;s} + e^{\beta s} V^\alpha_{_;s})
                - \dot{e}^{\alpha\beta}
                - \Opsr(B)^{\beta\alpha}
                - \qty(e^{\alpha\beta} V^s_{;s} + V^s e^{\alpha\beta}_{;s}) = 0 .
    \end{equation*}
    %
    Распишем инвариантную производную по времени для $e^{ij}$
    %
    \begin{equation}
        D_t{e^{ij}} = \dot{e}^{ij} - \delta_V e^{ij} =
            \dot{e}^{ij} + V^s e^{ij}_{;s} - V^i_{;s} e^{sj} - V^j_{;s} e^{is} .
    \end{equation}
    %
    Отсюда
    %
    \begin{equation}\label{eq:breq_eBV}
        D_t{e^{\alpha\beta}} + e^{\alpha\beta} \Opdiv{V} + \Opsr(B)^{\alpha\beta} = 0 .
    \end{equation}

    К сожалению, $D_t{\tau^{\alpha\beta}} \neq \gamma^{\alpha i}\gamma^{\beta j} D_t{\tau_{ij}}$ (операторы $D_t$ и $\gamma^{ij}$ не коммутируют: это является следствием того, что $\delta_V{\gamma^{ij}} \neq 0$), поэтому мы не можем напрямую переписать полученное выражение в обозначениях, где $e_{\alpha\beta}$ фигурировали бы исключительно с нижними индексами. Потому мы не можем записать тождество $D_t{e^{\alpha\beta}} = D^2_t{h^{\alpha\beta}}$. Иными словами, наиболее компактный явный вид полученного уравнения выглядит так:
    %
    \begin{equation}
        \gamma_{\alpha i}\gamma_{\beta j} D_t\qty(\gamma^{ik}\gamma^{jl} D_t{h_{kl}}) + D_t{h_{\alpha\beta}} \Opdiv{V} + \Opbr(h)_{\alpha\beta} = 0 .
    \end{equation}

    \paragraph{Случай $V^i = 0$.}

    Случай, когда фоновая метрика является метрикой Минковского, т.е. когда $V^i = 0$, является наиболее простым. Инвариантная производная по времени переходит в обычную, поскольку $\delta_V{h_{ij}} = 0$. Дивергенция $\Opdiv{V}$ исчезает. Уравнение (\ref{eq:breq_eBV}) записывается в простейшей форме
    %
    \begin{equation}
        \dot{e}_{\alpha\beta} + \Opsr(B)_{\alpha\beta} = 0 .
    \end{equation}

    Рассмотрим, какие требования это уравнение накладывает на тензоры $h_{ij}$, $e_{ij}$ и $B_{ij}$.

    Во-первых, если $a = \Opsr(b)$, то для $a$ верно:
    %
    \begin{enumerate}[nosep]
        \item $\tr a = 0$,
        \item $\Opsdiv(a) = 0$,
        \item $a_{ij} = a_{ji}$, если также $\Opsdiv(b) = 0$.
    \end{enumerate}

    При выполнении всех трех условий тензор $a$ будем называть симтензором. Поскольку $B_{ij}$ сам является полуротором, для тензора $\Opsr(B)_{ij}$ справедливы все три свойства. Отсюда тривиально следует, что $\dot{e}_{ij}$ также должен быть симтензором:
    %
    \begin{align*}
        \tr \dot{e} + \tr \Opsr(B) = 0
            \quad&\Longrightarrow\quad \qty(\tr e)_{,t} = 0 \\
        \Opsdiv(\dot{e})_i + \Opsdiv(\Opsr(B))_i = 0
            \quad&\Longrightarrow\quad \qty(\Opsr(e)_i)_{,t} = 0
    \end{align*}
    %
    Поскольку зависимость от времени для всех компонент $h_{ij}$ мы считаем определенной и одинаковой, данные условия выражают требования равенства нулю $\tr e$ и $\Opsdiv e$.

    Рассмотрим более подробно, что означают эти условия для исходного тензора $E_{ij}$:
    %
    \begin{equation*}
        e^{ij} = E^{ij} - \gamma^{ij} E , \quad
        \tr e = \tr E - 3 \tr E = -2 \tr E \quad\Longrightarrow\quad
        0 = e = -2E \quad\Longrightarrow\quad
        E = 0
    \end{equation*}
    %
    Отсюда $E_{ij} = e_{ij}$.

    Распишем $\Opsdiv(E)_i$:
    %
    \begin{equation*}
        \Opsdiv(E)_i
            = \gamma^{jk} E_{ij;k}
            = \gamma^{jk} \dot{h}_{ij;k}
            = \qty(\Opsdiv(h)_i)_{,t} ,
    \end{equation*}
    %
    откуда, следуя сделанному ранее замечанию, заключаем, что $h_{ij}$ также является симтензором.

    Итого, лишь из структуры уравнения (\ref{eq:breq_eBV}) мы получили, что все введенные поля необходимо должны быть симтензорами.

    Уравнение (\ref{eq:breq_eBV}) мы легко можем переписать теперь относительно $h_{ij}$:
    %
    \begin{equation}
        \ddot{h}_{\alpha\beta} + \Opbr(h)_{\alpha\beta} = 0 .
    \end{equation}
    %
    Представляя $h_{ij} \to h_{ij} \exp(-it)$, получаем простейшее уравнение вида
    %
    \begin{equation}
        \ddot{h} = \Opbr(h) .
    \end{equation}

    \paragraph{Условия на тензоры $h_{ij}$, $e_{ij}$, $B_{ij}$ в общем случае.}

    Воспользуемся той же методикой, которой мы следовали в предыдущем пункте:
    %
    \begin{gather*}
        \tr D_t{e^{\alpha\beta}} + e \Opdiv{V} = 0 \\
        \tr D_t{e^{ij}} = \gamma_{ij} \qty(
            \dot{e}_{ij} + V^s e^{ij}_{;s} - V^i_{;s} e^{sj} - V^j_{;s} e^{is}
        ) = \dot{e} + V^s e_{;s} - 2 V^{i;j} e_{ij} \quad\Longrightarrow\\
        \dot{e} + \nabla_V e + e \Opdiv{V} = 2 V^{i;j} e_{ij} .
    \end{gather*}
    %
    Это выражение является сравнительно более сложным, чем в случае $V^i = 0$, и даже априорное требование $e = 0$ не обращает его в тождество. В лучшем случае мы имеем требование $V^{i;j} e_{ij} = 0$, явный смысл которого мы указать не в силах.

    Взятием полуротора от основного уравнения можно прийти к еще более громоздкому результату, из которого явно также не следует, что $e_{ij}$ необходимо должен быть симтензором. Априорное предположение этого также не облегчает положения.

    Интуитивно $h_{ij}$ также следует искать в виде симтензора, на что намекает следующее обстоятельство.

    Операция $\Opsr(h)$ переводит нечетные гармоники в четные (и наоборот). Для нечетной гармоники верно $\tr h^{(o)} = 0$. Двойное применение оператора $\Opsr(h)$, т.е. оператор $\Opbr(h)$, не меняет четности гармоники. Иными словами, решение уравнение (\ref{eq:breq_eBV}) можно искать сначала в классе нечетных решений, а затем генерировать четные применением $\Opsr(h)$.

    Однако сгенерированное таким образом решение также будет иметь нулевой след. Для случая $V^i = 0$ условие $\tr h = 0$ также гарантировало $\tr h^{(e)} = 0$. Здесь такое условие не является следствием вариационных уравнений, а значит формально допустимы такие четные моды, для которых $\tr h^{(e)} = 0$.

    Более того, априорное предположение $\tr h = 0$ наравне с $\tr E = 0$ также влечет за собой нетривиальное условие $V^{i;j} h_{ij} = 0$, которое в общем случае может и не выполняться.

    По всей видимости, данные трудности можно было бы несколько сгладить подходящим выбором калибровочного поля скоростей $v^i \neq 0$.

\onlyinsubfile{
    \nocite{*}
    \clearpage
    \phantomsection
    \addcontentsline{toc}{section}{Список литературы}
    \bibliographystyle{\docroot/../lib/doc/bib/utf8gosttu}
    \bibliography{\docroot/../lib/doc/bib/math,\docroot/../lib/doc/bib/physics}
}

\end{document}
