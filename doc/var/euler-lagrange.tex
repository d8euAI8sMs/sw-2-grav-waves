\providecommand{\docroot}{..}
\documentclass[\docroot/reports/draft/report.tex]{subfiles}

\begin{document}

\onlyinsubfile{\tableofcontents}

\subsection{Традиционное уравнение Эйлера-Лагранжа}

    Имеем действие вида
    %
    \begin{equation}
        S = \int_\Omega \mathcal{L}(\hat{x}_n, \hat{f}_m(\hat{x}_n), \hat{f}'_m(\hat{x}_n)) \dd[n]{x} ,
    \end{equation}
    %
    где через $\hat{x}_n$ обозначен набор $n$ переменных.

    Минимум действия достигается на функциях $\hat{f}_m(\hat{x}_n)$, определяемых из уравнения Эйлера-Лагранжа:
    %
    \begin{equation}
        \pdv{\mathcal{L}}{f_m} - \sum\limits_n \partial_n{\pdv{\mathcal{L}}{(\partial_n{f_m})}} = 0 .
    \end{equation}

    Применительно к тензорным полям $h_{ij}(t,r,\theta,\varphi)$ действие будет иметь вид
    %
    \begin{equation}
        S = \int_\Omega \mathcal{L}(x^i, h_{ij}, h_{ij,k}) \sqrt{-\gamma} \dd[n]{x} ,
    \end{equation}
    %
    где $\gamma$~--- метрика пространства.

    Уравнения переписываются:
    %
    \begin{equation}
        \fdv{S}{h_{ij}} = T^{ij} = \pdv{\sqrt{-\gamma}\mathcal{L}}{h_{ij}} - \qty(\pdv{\sqrt{-\gamma}\mathcal{L}}{h_{ij,k}})_{,k} = 0 .
    \end{equation}

    Возникает естественный вопрос, \textit{возможно ли заменить частные производные на ковариантные?}. Если ответ положительный, многие выкладки общего характера можно будет упростить.

    \vspace{1cm}

    \textbf{\Large{References}}:
    %
    \begin{enumerate}
        \item Wikipedia: \enquote{Euler-Lagrange equation}~--- \url{https://en.wikipedia.org/wiki/Euler%E2%80%93Lagrange_equation}
    \end{enumerate}

\subsection{\foreignlanguage{english}{Gauge-invarient Euler-Lagrange equation}}

    Оказывается, ответ положительный. Преобразуем последнее уравнение к виду, имея в виду, что операторы $\pdv*{h_{ij}}$ и $\pdv*{h_{ij,k}}$ коммутируют с $\sqrt{-\gamma}$:
    %
    \begin{equation}
        \sqrt{-\gamma}\pdv{\mathcal{L}}{h_{ij}} - \qty(\sqrt{-\gamma}\pdv{\mathcal{L}}{h_{ij,k}})_{,k} = 0 .
    \end{equation}
    %
    Теперь поделим все на детерминант метрики:
    %
    \begin{equation}
        \pdv{\mathcal{L}}{h_{ij}} - \frac{1}{\sqrt{-\gamma}}\qty(\sqrt{-\gamma}\pdv{\mathcal{L}}{h_{ij,k}})_{,k} = 0 .
    \end{equation}

    Известно, что $\nabla_\mu = {\sqrt{-\gamma}}^{-1}(\sqrt{-\gamma}\ \partial_\mu)$, поэтому уравнение упрощается до
    %
    \begin{equation}
        \pdv{\mathcal{L}}{h_{ij}} - \qty(\pdv{\mathcal{L}}{h_{ij,k}})_{;k} = 0 .
    \end{equation}

    Осталось \enquote{разделаться} с частной производной $h_{ij,k}$. Здесь следует отметить, что поскольку $h_{ij;k} = h_{ij,k} + f(\Gamma, h)$, а $f$ не зависит от $h_{ij,k}$, в предыдущем выражении можно заменить $h_{ij,k}$ на $h_{ij;k}$.

    Окончательно,
    %
    \begin{equation}
        \pdv{\mathcal{L}}{h_{ij}} - \qty(\pdv{\mathcal{L}}{h_{ij;k}})_{;k} = 0 .
    \end{equation}

    \vspace{1cm}

    \textbf{\Large{References}}:
    %
    \begin{enumerate}
        \item Wikipedia: \enquote{Euler-Lagrange equation}~--- \url{https://en.wikipedia.org/wiki/Euler%E2%80%93Lagrange_equation}
        \item Wikipedia: \enquote{Gauge covariant derivative}~--- \url{https://en.wikipedia.org/wiki/Gauge_covariant_derivative}
        \item Wikipedia: \enquote{Covariant derivatives of tensors}~--- \url{https://en.wikipedia.org/wiki/Christoffel_symbols}
        \item Clinton L. Lewis. \enquote{Explicit gauge covariant Euler–Lagrange equation}~--- \url{http://www.stat.physik.uni-potsdam.de/~pikovsky/teaching/stud_seminar/eulerlagrange.pdf}
    \end{enumerate}

\onlyinsubfile{
    \nocite{*}
    \clearpage
    \phantomsection
    \addcontentsline{toc}{section}{Список литературы}
    \bibliographystyle{\docroot/../lib/doc/bib/utf8gosttu}
    \bibliography{\docroot/../lib/doc/bib/math,\docroot/../lib/doc/bib/physics}
}

\end{document}
