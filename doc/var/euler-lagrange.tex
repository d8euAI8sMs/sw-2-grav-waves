\providecommand{\docroot}{..}
\documentclass[\docroot/reports/draft/report.tex]{subfiles}

\begin{document}

\onlyinsubfile{\tableofcontents}

\subsection{Традиционное уравнение Эйлера-Лагранжа}

    Имеем действие вида
    %
    \begin{equation}
        S = \int_\Omega \mathcal{L}(\hat{x}_n, \hat{f}_m(\hat{x}_n), \hat{f}'_m(\hat{x}_n)) \dd[n]{x} ,
    \end{equation}
    %
    где через $\hat{x}_n$ обозначен набор $n$ переменных.

    Минимум действия достигается на функциях $\hat{f}_m(\hat{x}_n)$, определяемых из уравнения Эйлера-Лагранжа:
    %
    \begin{equation}
        \pdv{\mathcal{L}}{f_m} - \sum\limits_n \partial_n{\pdv{\mathcal{L}}{(\partial_n{f_m})}} = 0 .
    \end{equation}

    Применительно к тензорным полям $h_{ij}(t,r,\theta,\varphi)$ действие будет иметь вид
    %
    \begin{equation}
        S = \int_\Omega \mathcal{L}(x^i, h_{ij}, h_{ij,k}) \sqrt{-\gamma} \dd[n]{x} ,
    \end{equation}
    %
    где $\gamma$~--- метрика пространства.

    Уравнения переписываются:
    %
    \begin{equation}
        \fdv{S}{h_{ij}} = T^{ij} = \pdv{\sqrt{-\gamma}\mathcal{L}}{h_{ij}} - \qty(\pdv{\sqrt{-\gamma}\mathcal{L}}{h_{ij,k}})_{,k} = 0 .
    \end{equation}

    Возникает естественный вопрос, \textit{возможно ли заменить частные производные на ковариантные?}. Если ответ положительный, многие выкладки общего характера можно будет упростить.

    \vspace{1cm}

    \textbf{\Large{References}}:
    %
    \begin{enumerate}
        \item Wikipedia: \enquote{Euler-Lagrange equation}~--- \url{https://en.wikipedia.org/wiki/Euler%E2%80%93Lagrange_equation}
    \end{enumerate}

\subsection{\foreignlanguage{english}{Gauge-invarient Euler-Lagrange equation}}

    Оказывается, ответ положительный. Преобразуем последнее уравнение к виду, имея в виду, что операторы $\pdv*{h_{ij}}$ и $\pdv*{h_{ij,k}}$ коммутируют с $\sqrt{-\gamma}$:
    %
    \begin{equation}
        \sqrt{-\gamma}\pdv{\mathcal{L}}{h_{ij}} - \qty(\sqrt{-\gamma}\pdv{\mathcal{L}}{h_{ij,k}})_{,k} = 0 .
    \end{equation}
    %
    Теперь поделим все на детерминант метрики:
    %
    \begin{equation}
        \pdv{\mathcal{L}}{h_{ij}} - \frac{1}{\sqrt{-\gamma}}\qty(\sqrt{-\gamma}\pdv{\mathcal{L}}{h_{ij,k}})_{,k} = 0 .
    \end{equation}

    Известно\footnote{
        К сожалению, \enquote{очевидность} этого утверждения не выдержала непосредственной проверки. В одном из источников утверждение доказывается для \textit{скалярного} поля $\phi$. Разумеется, $\phi_{,s} = \phi_{;s}$. Для тензорного поля же формулировка лагранжевых уравнений требует непосредственного вывода.
    }\footnote{
        В первом случае независимыми переменными выступают $h_{ij}$ и $h_{ij,k}$. Во втором~--- $h_{ij}$ и $h_{ij;k}$. Данное замечание существенно, потому как считая $\pdv*{h_{ij;k}}{h_{lm}} \neq 0$, можно придти к некорректному результату.
    }, что $\nabla_\mu = {\sqrt{-\gamma}}^{-1}(\sqrt{-\gamma}\ \partial_\mu)$, поэтому уравнение упрощается до
    %
    \begin{equation}
        \pdv{\mathcal{L}}{h_{ij}} - \qty(\pdv{\mathcal{L}}{h_{ij,k}})_{;k} = 0 .
    \end{equation}

    Осталось \enquote{разделаться} с частной производной $h_{ij,k}$. Здесь следует отметить, что поскольку $h_{ij;k} = h_{ij,k} + f(\Gamma, h)$, а $f$ не зависит от $h_{ij,k}$, в предыдущем выражении можно заменить $h_{ij,k}$ на $h_{ij;k}$.

    Окончательно,
    %
    \begin{equation}
        \pdv{\mathcal{L}}{h_{ij}} - \qty(\pdv{\mathcal{L}}{h_{ij;k}})_{;k} = 0 .
    \end{equation}

\subsection{Уравнения Эйлера-Лагранжа в ковариантной форме}

    Попробуем доказать, что
    %
    \begin{equation*}
        \pdv{\mathcal{L}}{h_{ij}} - \qty(\pdv{\mathcal{L}}{h_{ij;k}})_{;k} = 0
    \end{equation*}
    %
    действительно выражает уравнение Эйлера-лагранжа для лагранжиана $\mathcal{L}$, если полагать переменные $h_{ij}$ и $h_{ij;k}$ независимыми.

    Пусть изначально задан лагранжиан вида $\mathcal{L} = \mathcal{L}(h_{ij}, h_{ij,k})$. Его вариации на метрике $h_{\alpha\beta}$ имеют вид:
    %
    \begin{gather*}
        \fdv{\sqrt{\gamma}\mathcal{L}}{h_{\alpha\beta}} =
            \pdv{\sqrt{\gamma}\mathcal{L}}{h_{\alpha\beta}} - \qty(\pdv{\sqrt{\gamma}\mathcal{L}}{h_{\alpha\beta,\chi}})_{,\chi} = \\
            \sqrt{\gamma}\pdv{\mathcal{L}}{h_{\alpha\beta}} - \qty(\sqrt{\gamma}\pdv{\mathcal{L}}{h_{\alpha\beta,\chi}})_{,\chi}
        .
    \end{gather*}

    В то же время мы можем записать $\mathcal{L} = \Phi(h_{ij}, h_{ij;k})$. Подставляя $\Phi$ в выражение для вариации, получим:
    %
    \begin{equation*}
        \sqrt{\gamma}^{-1}\fdv{\sqrt{\gamma}\mathcal{L}}{h_{\alpha\beta}} = \pdv{\Phi}{h_{\alpha\beta}} - \sqrt{\gamma}^{-1}\qty(\sqrt{\gamma}\pdv{\Phi}{h_{\alpha\beta,\chi}})_{,\chi} .
    \end{equation*}
    %
    Пользуясь тем, что
    %
    \begin{equation*}
        h_{ij;k} = h_{ij,k} - \Gamma^s_{ik} h_{sj} - \Gamma^s_{jk} h_{is},
    \end{equation*}
    %
    запишем
    %
    \begin{gather*}
        \pdv{\Phi(h_{ij}, h_{ij;k})}{h_{\alpha\beta}} =
            \pdv{\Phi}{h_{ij}} \pdv{h_{ij}}{h_{\alpha\beta}} +
            \pdv{\Phi}{h_{ij;k}} \pdv{h_{ij;k}}{h_{\alpha\beta}} \\
        \pdv{\Phi(h_{ij}, h_{ij;k})}{h_{\alpha\beta,\chi}} =
            \pdv{\Phi}{h_{ij;k}} \pdv{h_{ij;k}}{h_{\alpha\beta,\chi}} .
    \end{gather*}
    %
    Частная производная $\pdv*{h_{ij;k}}{h_{\alpha\beta,\chi}}$, очевидно, есть $\delta_i^\alpha \delta_j^\beta \delta_k^\chi$. Другая частная производная также не сложно вычисляется:
    %
    \begin{equation*}
        \pdv{h_{ij;k}}{h_{\alpha\beta}} =
            - \Gamma^s_{ik} \delta_s^\alpha \delta_j^\beta - \Gamma^s_{jk} \delta_i^\alpha \delta_s^\beta =
            - \Gamma^\alpha_{ik} \delta_j^\beta - \Gamma^\beta_{jk} \delta_i^\alpha .
    \end{equation*}

    Введем обозначения:
    %
    \begin{equation*}
        \tau^{ijk} \equiv \pdv{\Phi}{h_{ij;k}} , \qquad
        \varkappa^{ij} \equiv \pdv{\Phi}{h_{ij}}
    \end{equation*}
    %
    Выпишем для наглядности имеющийся на текущий момент результат:
    %
    \begin{equation*}
        \sqrt{\gamma}^{-1}\fdv{\sqrt{\gamma}\mathcal{L}}{h_{\alpha\beta}} =
            \qty(
                \varkappa^{\alpha\beta}
                - \Gamma^\alpha_{ik} \tau^{\beta i k}
                - \Gamma^\beta_{ik} \tau^{\alpha i k}
            )
            - \sqrt{\gamma}^{-1}\qty(
                \sqrt{\gamma} \tau^{\alpha\beta\chi}
            )_{,\chi} .
    \end{equation*}
    %
    Здесь произведены небольшие преобразования (правило подстановки для $\delta^i_j$, перестановка симметричных индексов и т.д.).

    Осталось преобразовать вторую часть. Раскроем скобки и получим:
    %
    \begin{equation*}
        \qty(\sqrt{\gamma} \tau^{\alpha\beta\chi})_{,\chi} =
            \sqrt{\gamma}_{,\chi} \tau^{\alpha\beta\chi} +
            \sqrt{\gamma} \tau^{\alpha\beta\chi}_{,\chi} =
        \sqrt{\gamma} \Gamma^s_{s \chi} \tau^{\alpha\beta\chi} +
            \sqrt{\gamma} \tau^{\alpha\beta\chi}_{,\chi}
    \end{equation*}
    %
    Также, поскольку:
    %
    \begin{equation*}
        \tau^{\alpha\beta\chi}_{;\rho}
            = \tau^{\alpha\beta\chi}_{,\rho}
            + \Gamma^\alpha_{s \rho} \tau^{s \beta\chi}
            + \Gamma^\beta_{s \rho} \tau^{\alpha s \chi}
            + \Gamma^\chi_{s \rho} \tau^{\alpha\beta s} ,
    \end{equation*}
    %
    имеем
    %
    \begin{equation*}
        \tau^{\alpha\beta\chi}_{,\chi}
            = \tau^{\alpha\beta\chi}_{;\chi}
            - \Gamma^\alpha_{s \chi} \tau^{s \beta\chi}
            - \Gamma^\beta_{s \chi} \tau^{\alpha s \chi}
            - \Gamma^\chi_{s \chi} \tau^{\alpha\beta s} .
    \end{equation*}
    %
    Замечаем, что слагаемые вида $\Gamma^s_{s \chi} \tau^{\alpha\beta\chi}$ взаимно уничтожаются.

    Наконец, имеем:
    %
    \begin{gather*}
        \sqrt{\gamma}^{-1}\fdv{\sqrt{\gamma}\mathcal{L}}{h_{\alpha\beta}} =
            \varkappa^{\alpha\beta}
                - \Gamma^\alpha_{ik} \tau^{\beta ik}
                - \Gamma^\beta_{ik} \tau^{\alpha ik}
            - \tau^{\alpha\beta k}_{; k}
                + \Gamma^\alpha_{ik} \tau^{i \beta k}
                + \Gamma^\beta_{ik} \tau^{\alpha ik} \\
            = \varkappa^{\alpha\beta} - \tau^{\alpha\beta k}_{; k}
            = \pdv{\Phi}{h_{ij}} - \qty(\pdv{\Phi}{h_{ij;k}})_{;k}.
    \end{gather*}

    \vspace{1cm}

    \textbf{\Large{References}}:
    %
    \begin{enumerate}
        \item Wikipedia: \enquote{Euler-Lagrange equation}~--- \url{https://en.wikipedia.org/wiki/Euler%E2%80%93Lagrange_equation}
        \item Wikipedia: \enquote{Gauge covariant derivative}~--- \url{https://en.wikipedia.org/wiki/Gauge_covariant_derivative}
        \item Wikipedia: \enquote{Covariant derivatives of tensors}~--- \url{https://en.wikipedia.org/wiki/Christoffel_symbols}
        \item Clinton L. Lewis. \enquote{Explicit gauge covariant Euler–Lagrange equation}~--- \url{http://www.stat.physik.uni-potsdam.de/~pikovsky/teaching/stud_seminar/eulerlagrange.pdf}
    \end{enumerate}

\onlyinsubfile{
    \nocite{*}
    \clearpage
    \phantomsection
    \addcontentsline{toc}{section}{Список литературы}
    \bibliographystyle{\docroot/../lib/doc/bib/utf8gosttu}
    \bibliography{\docroot/../lib/doc/bib/math,\docroot/../lib/doc/bib/physics}
}

\end{document}
