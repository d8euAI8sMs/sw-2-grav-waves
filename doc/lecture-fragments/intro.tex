\def\docroot{..}
\documentclass[\docroot/reports/lectures-draft/report.tex]{subfiles}

\begin{document}

\onlyinsubfile{\tableofcontents}

При изучении сложных явлений всегда проще начинать с каких-либо аналогий. Это касается и, в частности, волновых явлений. Так, перед изучением механических волн обычно демонстрируются простейшие опыты с волнами на поверхности жидкости, хотя, строго говоря, это отдельный класс волн.

Перед изучением гравитационных волн читателю будет полезно вспомнить (или изучить) основные моменты, касающиеся волн как таковых: понятие волны, плотности и потока энергии. Этой цели служит секция об электромагнитных волнах.

Для простоты мы не будем углубляться в волны в среде, а будем рассматривать лишь решения, справедливые в вакууме. Это позволит нам с одной стороны ввести все необходимые понятия, с другой же не переусложнить введение.

\onlyinsubfile{
    \clearpage
    \phantomsection
    \addcontentsline{toc}{section}{Список литературы}
    \bibliographystyle{\docroot/../lib/doc/bib/utf8gosttu}
    \bibliography{\docroot/../lib/doc/bib/math,\docroot/../lib/doc/bib/physics}
}

\end{document}
