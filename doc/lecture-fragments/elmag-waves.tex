\def\docroot{..}
\documentclass[\docroot/reports/lectures-draft/report.tex]{subfiles}

\begin{document}

\onlyinsubfile{\tableofcontents}

\subsection{Энергия и работа электрического поля}

    Рассмотрим два точечных заряда, $q_1$ и $q_2$, расположенных на расстоянии $R$ друг от друга. На первый заряд со стороны второго заряда согласно закону Кулона будет действовать сила (в Гауссовой системе единиц)
    %
    \begin{equation*}
        F = \frac{q_1 q_2}{R^2} .
    \end{equation*}
    %
    Направление этой силы совпадает с радиус-вектором, соединяющим центры этих зарядов, поэтому можно написать
    %
    \begin{equation*}
        \vb{F} = \frac{q_1 q_2}{R^2} \frac{\vb{R}}{R} ,
    \end{equation*}
    %
    имея в виду под $\vb{R}$ радиус-вектор, проведенный из центра первого заряда в центр второго.

    Если теперь начать двигать второй заряд, тем самым придавая $\vb{R}$ произвольные значения $\vb{r}$, сила $\vb{F}$ станет функцией $\vb{r}$, именно $\vb{F} = \vb{F}(\vb{r})$. Будем смотреть на $q_2$ как на некоторый \textit{пробный заряд}:
    %
    \begin{equation*}
        \vb{F}(\vb{r}) = q_2 \frac{q_1}{r^2} \frac{\vb{r}}{r} = q_2 \vb{E}(\vb{r}) .
    \end{equation*}
    %
    Величина $\vb{E}$, называемая \textit{напряженностью электрического поля}, есть \textit{характеристика электрического поля} заряда $q_1$. Она показывает, как действует заряд $q_1$ на единичный пробный заряд $q$, находящийся в точке с радиус-вектором $\vb{r}$.

    Если теперь рассмотреть систему зарядов $\qty{q_i}$, то на некоторый заряд $q_0$ со стороны системы зарядов по аналогии будет действовать сила
    %
    \begin{equation*}
        \vb{F}(\vb{r}) = \sum\limits_i \frac{q_0 q_i}{r_{0i}} \frac{\vb{r}_{0i}}{r_{0i}} = q_0 \vb{E}(\vb{r}) , \quad \vb{r}_{0i} = \vb{r}_0 - \vb{r}_i .
    \end{equation*}
    %
    Отсюда видно, что поле $\vb{E}$ может быть довольно сложной конфигурации.

    Элементарная работа против сил поля по перемещению заряда $q$ в некотором поле $\vb{E}$ из точки $\vb{r}$ в бесконечно близкую точку $\vb{r} + \dd{\vb{r}}$, в которой поле $\vb{E}$ можно считать таким же, как в точке $\vb{r}$, как известно, определяется произведением силы $\vb{F}$ на элементарное перемещение $\dd{\vb{r}}$:
    %
    \begin{equation*}
        \dd{A} = \vb{F} \dd{\vb{r}} = q \vb{E} \dd{\vb{r}} = q \dd{\varphi} .
    \end{equation*}
    %
    Работа при перемещении заряда между точками $(1)$ и $(2)$ найдется интегрированием предыдущего выражения:
    %
    \begin{equation*}
        A_{12} = \int\limits_{(1)}^{(2)} q \dd{\varphi} = q (\varphi_2 - \varphi_1) .
    \end{equation*}
    %
    Этим вводится новая величина $\varphi(\vb{r})$, называемая \textit{электрическим потенциалом}. Формально деля в последнем равенстве выражения для элементарной работы на $\dd{\vb{r}}$, получаем выражение $\vb{E}$ через $\varphi$:
    %
    \begin{equation*}
        \vb{E} = \dv{\varphi}{\vb{r}} = \Grad \varphi .
    \end{equation*}

    Из выражения для $A_{12}$ видно, что работа определяется лишь разностью потенциалов. Иначе его можно переписать в виде
    %
    \begin{equation*}
        A_{12} = q \varphi_2 - q \varphi_1 = W_2 - W_1 = \Delta_{12} W .
    \end{equation*}
    %
    Здесь введена величина $W = q \varphi$, которую можно интерпретировать как \textit{электрическую энергию}. Работа есть изменение этой энергии при переходе из одного состояния в другое.

    Рассмотрим, с чем связано изменение энергии $W$. Для этого распишем дифференциал $\dd{W}$:
    %
    \begin{equation*}
        \dd{W} = q \dd{\varphi} + \varphi \dd{q} .
    \end{equation*}
    %
    Первое знакомое слагаемое связано с изменением потенциала при переходе из точки $\vb{r}$ в бесконечно близкую точку $\vb{r} + \dd{\vb{r}}$. Второе же связано с изменением величины заряда $q$. Подойдем к его обоснованию постепенно, приняв без доказательства следующую теорему:
    %
    \begin{equation*}
        \oint\limits_S \vb{E} \dd{\vb{S}} = 4 \pi q_\text{int} .
    \end{equation*}
    %
    Данная теорема носит название теоремы Гаусса: она связывает поле на границе поверхности $S$ с зарядом $q_\text{int}$, заключенным внутри этой поверхности. Прямым вычислением можно показать, что она выполняется для одиночного заряда $q$ и окружающей его поверхности $S$ в виде сферы некоторого радиуса с центром в точке стояния заряда. Можно показать справедливость теоремы для случая любой конфигурации поля и любой поверхности $S$.

    Рассмотрим процесс переноса бесконечно малого заряда $\dd{q} = \dd{S} \dd{\sigma}$ с плоской площадки $\dd{S}$ на такую же площадку $\dd{S}$ на расстоянии $\dd{l}$, причем полагаем, что $\dd{l}$~--- бесконечно малая более высокого порядка, чем $\dd{S}$, так что $\dd{S}$ можно считать бесконечной плоскостью, пренебрегая краевыми эффектами, а поле направленным перпендикулярно к ней. Работа против сил поля связана с изменением энергии $W$:
    %
    \begin{equation*}
        \dd{A} = \dd{W} = \varphi \dd{q} = (E \dd{l}) (\dd{S} \dd{\sigma}) = E \dd{V} \dd{\sigma} .
    \end{equation*}
    %
    Изменение поля между площадками по теореме Гаусса связано с добавлением заряда $\dd{q}$ к одной из них. Окружая эту площадку поверхностью в виде параллелепипеда и проводя тривиальное интегрирование, имеем:
    %
    \begin{equation*}\begin{aligned}
        (E \dd{S})_{(2)} - (E \dd{S})_{(1)} &= \dd{S} \dd{E} \\ &= 4 \pi \qty[(Q + \dd{q}) - Q] = 4 \pi \dd{q} = 4 \pi \dd{S} \dd{\sigma} .
    \end{aligned}\end{equation*}
    %
    Подставляя $\dd{\sigma}$ из последнего выражения в предыдущее, получим
    %
    \begin{equation*}
        \dd{W} = \frac{1}{4 \pi} E \dd{E} \dd{V} ,
    \end{equation*}
    %
    Деление на $\dd{V}$ и интегрирование по $E$ дает
    %
    \begin{equation*}
        w = \frac{1}{8 \pi} \vb{E}^2 .
    \end{equation*}
    %
    Величина $w$ называется \textit{объемной плотностью электрической энергии} и является функцией пространственных координат. Говорят, что энергия поля \textit{локализована в пространстве}~--- в самом поле. Интегрирование по всему объему дает полную энергию поля $W$:
    %
    \begin{equation*}
        W = \frac{1}{8 \pi} \int \vb{E}^2 \dd{V} .
    \end{equation*}

\subsection{Уравнения Максвелла}

    Изучению волновых явлений предшествует изучение электрических и магнитных статических явлений. Мы же опустим здесь рассмотрение магнитостатики, полагая, что читатель знаком с понятием магнитного поля. Статические явления во многих случаях могут рассматриваться независимо: только электрические явления и только магнитные явления. В динамике же между магнитными и электрическими явлениями появляется связь. Эту связь выражают уравнения Максвелла.

    Уравнения Максвелла для векторов напряженности электрического $\vb{E}$ и магнитного $\vb{B}$ полей записываются следующим образом:
    %
    \begin{alignat}{3}
        \Div \vb{E} &= 4 \pi \rho , &\quad \Rot \vb{E} &= - \frac{1}{c} \pdv{\vb{B}}{t} , \\
        \Div \vb{B} &= 0 , &\quad \Rot \vb{B} &= \frac{4 \pi}{c} \vb{j} + \frac{1}{c} \pdv{\vb{E}}{t} .
    \end{alignat}
    %
    Для рассмотрения простейших волновых явлений достаточно изучать пространство без сторонних зарядов и токов ($\rho = 0$, $\vb{j} = 0$). Тогда эволюция электромагнитного поля определяется лишь самим этим полем:
    %
    \begin{alignat}{4}\label{eq:maxwell-free-space}
        \Div \vb{E} &= 0 , &\quad \Rot \vb{E} &= - &{}\frac{1}{c} \pdv{\vb{B}}{t} , \\
        \Div \vb{B} &= 0 , &\quad \Rot \vb{B} &= &{}\frac{1}{c} \pdv{\vb{E}}{t} .
    \end{alignat}

\subsection{Плотность и поток энергии электромагнитного поля}

    По аналогии с плотностью энергии электрического поля, можно ввести и плотность энергии магнитного поля:
    %
    \begin{equation*}
        w_B = \frac{1}{8 \pi} \vb{B}^2 .
    \end{equation*}
    %
    Полная плотность энергии тогда будет состоять из двух слагаемых:
    %
    \begin{equation*}
        w = w_E + w_B = \frac{1}{8 \pi} (\vb{E}^2 + \vb{B}^2) .
    \end{equation*}

    Выясним, как меняется $w$ со временем. Для этого продифференцируем $w$ по $t$:
    %
    \begin{equation*}
        \pdv{w}{t} = \frac{1}{4 \pi} (\vb{E}\vb{\dot{E}} + \vb{B}\vb{\dot{B}}) .
    \end{equation*}
    %
    Выразим производные от полей по времени из уравнений Максвелла и подставим их в полученное выражение:
    %
    \begin{equation*}
        \pdv{w}{t} = \frac{c}{4 \pi} (\vb{E} \Rot \vb{B} - \vb{B} \Rot \vb{E}) .
    \end{equation*}
    %
    Пользуясь векторным тождеством
    %
    \begin{equation*}
        \vb{E} \Rot \vb{B} - \vb{B} \Rot \vb{E} = - \Div \qty(\vb{E} \times \vb{B})
    \end{equation*}
    %
    и вводя векторную величину
    %
    \begin{equation*}
        \vb{S} = \frac{c}{4 \pi} \qty(\vb{E} \times \vb{B}) ,
    \end{equation*}
    %
    получим окончательно
    %
    \begin{equation*}
        \pdv{w}{t} + \Div \vb{S} = 0 .
    \end{equation*}
    %
    Вектор $\vb{S}$ называется \textit{вектором Пойнтинга} (Умова-Пойнтинга), или \textit{плотностью потока энергии}.

    Для выяснения физического смысла полученного результата проинтегрируем его по конечному объему $V$:
    %
    \begin{equation*}
        \pdv{t} \int\limits_V w \dd{V} = - \oint\limits_{\partial V} \vb{S} \dd{\vb{\Omega}} .
    \end{equation*}
    %
    Здесь объемный интеграл от $\Div \vb{S}$ был заменен на интеграл от $\vb{S}$ по поверхности $\partial V$, ограничивающей объем $V$. Первый интеграл~--- суть полная энергия поля, сосредоточенная в объеме $V$. Ее изменение во времени связано с потоком энергии через поверхность, ограничивающую этот объем. Таким образом, полученное соотношение выражает не что иное как \textit{закон сохранения энергии}.

\subsection{Волновое уравнение}

    Записанные выше уравнения Максвелла линейны по $\vb{E}$ и по $\vb{B}$~--- справедлив принцип суперпозиции. Это означает, что любую сложную волну можно рассматривать как набор отдельных монохроматических волн, зависящих от времени через множитель вида $\cos(\omega (t - t_0))$ с различными частотами $\omega$. Итак,
    %
    \begin{equation*}
        \vb{E}(\vb{r},t) = \vb{E}(\vb{r}) \cos(\omega (t - t_0)) .
    \end{equation*}
    %
    Часто удобнее пользоваться комплексным представлением монохроматической волны:
    %
    \begin{equation*}
        \vb{E}(\vb{r},t) = \Re{\vb{\tilde{E}}(\vb{r},t)} = \Re{\vb{\tilde{E}}(\vb{r}) \exp(i \omega (t - t_0))} .
    \end{equation*}
    %
    Здесь $\vb{\tilde{E}}(\vb{r})$~--- т.н. комплексная амплитуда волны. Далее для краткости обозначений значок $\tilde{\ }$ будет опускаться, равно как и добавленное для большей общности фазовое слагаемое $\omega t_0$.

    Подстановка $\vb{E}(\vb{r},t) = \vb{E}(\vb{r}) \exp(i \omega t)$, $\vb{B}(\vb{r},t) = \vb{B}(\vb{r}) \exp(i \omega t)$ в два последних уравнения Максвелла (\autoref{eq:maxwell-free-space}) дает:
    %
    \begin{equation*}
        \Rot \vb{E} = - \frac{i \omega}{c} \vb{B} , \quad
        \Rot \vb{B} = \frac{i \omega}{c} \vb{E} .
    \end{equation*}
    %
    Выражая $\vb{B}$ из первого уравнения и подставляя его во второе, получаем окончательно волновое уравнение для вектора $\vb{E}$:
    %
    \begin{equation}\label{eq:waweeq-rot}
        \Rot\Rot \vb{E} = \frac{\omega^2}{c^2} \vb{E} .
    \end{equation}
    %
    Аналогичное уравнение получается и для вектора $\vb{B}$. Волновое уравнение можно упростить, используя тождество $\Rot\Rot = \Grad\Div - \Delta$ и имея в виду первое из уравнений Максвелла ($\Div \vb{E} = 0$):
    %
    \begin{equation}\label{eq:waweeq-lap}
        \Delta \vb{E} = - \frac{\omega^2}{c^2} \vb{E} .
    \end{equation}

\subsection{Плоские электромагнитные волны}

    Самый простой вид электромагнитных волн~--- плоская монохроматическая волна:
    %
    \begin{equation*}
        \vb{E} = \vb{E}_0 \exp(i \omega t - i \vb{k}\vb{r}) .
    \end{equation*}
    %
    Она является элементарным решением волнового уравнения в декартовых координатах (второе решение получается формальной заменой $\vb{k}$ на $-\vb{k}$ в экспоненте и представляет собой волну, распространяющуюся в противоположном первой направлении). Здесь $\vb{k}$~--- волновой вектор, определяющий направление распространения волны: $\vb{n} = \flatfrac{\vb{k}}{\qty|\vb{k}|}$, причем существует связь между волновым вектором $\vb{k}$ и частотой $\omega$, выражаемая соотношением $\qty|\vb{k}| = \flatfrac{\omega}{c}$. Плоские волны физически являются волнами от удаленного источника.

    Несложно убедиться, что вектора $\vb{E}$ и $\vb{B}$ в плоской волне ортогональны друг другу и вектору $\vb{k}$. Действительно, т.к. $\Rot \vb{E} = \nabla \times \vb{E} = - i \vb{k} \times \vb{E}$, то из одного из последних двух уравнений Максвелла
    %
    \begin{equation*}
        \vb{k} \times \vb{E} = \frac{\omega}{c} \vb{B} .
    \end{equation*}
    %
    Вектор $\vb{B}$ коллинеарен вектору $\vb{k} \times \vb{E}$, ортогональному как $\vb{E}$, так и $\vb{k}$.

    Таким образом, физическая картина плоских волн следующая: напряженности электрического и магнитного поля колеблются в перпендикулярных друг другу плоскостях, причем обе плоскости перпендикулярны направлению распространения, задаваемому вектором $\vb{k}$. В силу последнего говорят, что плоские электромагнитные волны~--- поперечные. Пучности в волне распределены в пространстве по гармоническому закону, причем они сдвигаются со временем в направлении $\vb{k}$.

    Из последнего соотношения легко видеть, что для плоских волн $\qty|\vb{E}| = \qty|\vb{B}|$. Действительно, поскольку $\vb{k} \perp \vb{E}$, модуль векторного произведения $\vb{k} \times \vb{E}$ равен произведению модулей сомножителей. Принимая во внимание, что $\qty|\vb{k}| = \flatfrac{\omega}{c}$, получаем искомый результат. Отсюда плотность электромагнитной энергии имеет вид:
    %
    \begin{equation*}
        w = \frac{1}{8 \pi} \qty(\vb{E}^2 + \vb{B}^2) = \frac{1}{4 \pi} \vb{E}^2 = \frac{1}{4 \pi} \vb{B}^2 .
    \end{equation*}
    %
    Плотность потока энергии ожидаемо совпадает по направлению с вектором $\vb{k}$:
    %
    \begin{equation*}
        \vb{S} = \frac{c}{4 \pi} \qty(\vb{E} \times \vb{B}) = \frac{c}{4 \pi} \qty(\vb{E} \times \vb{k} \times \vb{E}) = \frac{c}{4 \pi} \vb{k} \vb{E}^2 = \frac{c}{4 \pi} \vb{k} \vb{B}^2 .
    \end{equation*}
    %
    Здесь использовано тождество $\vb{a} \times \vb{b} \times \vb{c} = \vb{b}\ \qty(\vb{a} \vb{c}) - \vb{c}\ \qty(\vb{a} \vb{b})$ и тот факт, что $\vb{k} \vb{E} = 0$ в силу их ортогональности.

\onlyinsubfile{
    \clearpage
    \phantomsection
    \addcontentsline{toc}{section}{Список литературы}
    \nocite{sivuhin_elect}
    \bibliographystyle{\docroot/../lib/doc/bib/utf8gosttu}
    \bibliography{\docroot/../lib/doc/bib/math,\docroot/../lib/doc/bib/physics}
}

\end{document}
