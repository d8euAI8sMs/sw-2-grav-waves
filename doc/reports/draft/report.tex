\documentclass[12pt,a4paper]{article}

\providecommand{\docroot}{../..}

\input{\docroot/../lib/doc/pkg/common}
\input{\docroot/../lib/doc/math/operators}

\geometry{left=0.5cm,right=2.5cm,top=2cm,bottom=2cm}
\usepackage[textwidth=2cm,textsize=tiny,linecolor=OliveGreen,bordercolor=OliveGreen,backgroundcolor=OliveGreen!25,colorinlistoftodos]{todonotes}

\title{Сферические гравитационные волны}
\author{Василевский А.В.}

\begin{document}

    \makedocroot

    \maketitle
    \tableofcontents

    \section{Введение}
    \subfile{\docroot/fragments/intro.tex}

    \section{Постановка проблемы}
    \subfile{\docroot/fragments/problem.tex}

    \section{Математические инструменты описания гравитационных волн}
    \subfile{\docroot/fragments/grav-generic.tex}

    \section{Сферические гравитационные волны}
    \subfile{\docroot/fragments/grav-spher.tex}

    \section{Заключение}
    \subfile{\docroot/fragments/outro.tex}

    \begin{appendix}

        \section{Общая информация}
        \subfile{\docroot/var/general.tex}

        \section{Уравнение Эйлера-Лагранжа}
        \subfile{\docroot/var/euler-lagrange.tex}

        \section{Тензор энергии-импульса}
        \subfile{\docroot/var/energy-impulse.tex}

        \section{Идеальная жидкость}
        \subfile{\docroot/var/perfect-fluid.tex}

        \section{Решения Шварцшильда}
        \subfile{\docroot/var/schwarzschild.tex}

    \end{appendix}

    \todo[inline]{Унифицировать нумерацию}
    \todototoc\listoftodos

    \clearpage

    \phantomsection
    \addcontentsline{toc}{section}{Список литературы}
    \bibliographystyle{\docroot/../lib/doc/bib/utf8gosttu}
    \bibliography{\docroot/../lib/doc/bib/math,\docroot/../lib/doc/bib/physics}

\end{document}
