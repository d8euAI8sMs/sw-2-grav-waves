\providecommand{\docroot}{../..}
\documentclass[\docroot/reports/draft/report.tex]{subfiles}

\begin{document}

\onlyinsubfile{\tableofcontents}

Идеальная жидкость является одной из наиболее простых (наряду с пылевидной материей) моделей вещества в общей теории относительности (ОТО). Большинство реальных звезд достаточно неплохо описывается в приближении идеальной жидкости. Анализ моделей жидкости и методов описания взаимодействия жидкости с гравитационным полем может помочь с решением проблемы учета источников излучения в задаче о гравитационных волнах.

В работе исследуются модели и методы описания идеальной жидкости в рамках ОТО и ТГВ (теории глобального времени, \cite{burlankov_space_dynamics,burlankov_grav_waves,burlankov_new_phys}). Делается упор на лагранжев подход, впервые предложенный Б.~Шютцем \cite{schutz_vel_pot}. Из лагранжиана жидкости выводятся динамические уравнения, справедливые для любой идеальной безвихревой жидкости. Температурные зависимости не рассматриваются.

Статические сферически симметричные модели решаются аналитически в рамках ОТО и ТГВ: воспроизводится внутреннее решение Шварцшильда для несжимаемой жидкости, которое сопоставляется с аналогичным решением в метрике ТГВ, впервые полученным в данной работе. В рамках ТГВ также рассматривается модель сжимаемой ультрарелятивистской жидкости. Усложнение модели приводит к усложнению уравнений~--- простое аналитическое решение их отсутствует.

\subsection{Постановка задачи}

Целью работы является исследование аппарата описания идеальной жидкости в рамках ОТО и ТГВ. Рассматриваются статические сферически симметричные модели.

\subsection{Общая теория относительности (ОТО)}
\subfile{\docroot/fragments/gr-intro.tex}

\subsection{Проблемы ОТО и теория глобального времени}
\subfile{\docroot/fragments/gtt-intro.tex}

\subsection{Применяемые обозначения}
\subfile{\docroot/fragments/notation.tex}

\onlyinsubfile{
    \clearpage
    \phantomsection
    \addcontentsline{toc}{section}{Список литературы}
    \bibliographystyle{\docroot/../lib/doc/bib/utf8gosttu}
    \bibliography{\docroot/../lib/doc/bib/math,\docroot/../lib/doc/bib/physics}
}

\end{document}
