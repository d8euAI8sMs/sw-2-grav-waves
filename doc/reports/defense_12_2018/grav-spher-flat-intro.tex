\providecommand{\docroot}{../..}
\documentclass[\docroot/reports/draft/report.tex]{subfiles}

\begin{document}

Для изучения гравитационных волн вблизи сферически симметричного источника наиболее естественным видом фоновой метрики является метрика Шварцшильда. Это, однако, вносит определенные вычислительные трудности, сопряженные с необходимостью учитывать искривленность самог\'{о} пространства сравнения. В приближенной теории кривизной пространства сравнения на достаточном удалении от источника можно пренебречь.

В данном разделе в качестве пространства сравнения выступает пространство Минковского. Фоновая метрика~--- метрика плоской сферической системы координат. Выводы \autoref{sec:grav-spher} же справедливы для любого выбора фоновой метрики, лишь бы она была сферически симметричной. В частности, они справедливы и в пространстве сравнения с метрикой Шварцшильда (Пенлеве).

\end{document}
