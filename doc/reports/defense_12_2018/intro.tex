\providecommand{\docroot}{../..}
\documentclass[\docroot/reports/draft/report.tex]{subfiles}

\begin{document}

\onlyinsubfile{\tableofcontents}

Предметом изучения в данной работе являются сферические гравитационные волны от точечных источников. Проблемы классического эйнштейновского описания гравитационных волн известны. Камнем преткновения теории является принцип общей ковариантности, согласно которому пространство и время равноправны: нельзя отличить временн\'{у}ю эволюцию системы от простой замены координат.

Данная работа строится на основе теории глобального времени, полагающей время выделенной переменной, тем самым отказываясь от принципа обшей ковариантности. Как мы увидим далее, описанный подход является достаточно плодотворным.

Основными результатами работы являются описанные подходы к описанию сферических гравитационных волн в линейном приближении. Получены аналитические решения, найдены и описаны плотность и поток энергии.

\end{document}
