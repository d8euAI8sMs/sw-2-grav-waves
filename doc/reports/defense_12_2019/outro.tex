\def\docroot{../..}
\documentclass[\docroot/reports/draft/report.tex]{subfiles}

\begin{document}

\onlyinsubfile{\tableofcontents}

1. В работе описан подход к построению теории гравитации на принципах глобального времени. Отказ от общековариантного подхода и переход к чисто лагранжеву формализму позволяют упростить получение линеаризованных уравнений динамики. Сферическая симметрия задачи и явно выделенное глобальное время позволяют эффективно применить метод Ли-генерации сферических гармоник.

2. Данная работа является продолжением трудов \cite{burlankov_space_dynamics} и \cite{burlankov_grav_waves}. Здесь впервые получены и исследованы уравнения динамики в самом общем их виде.

3. В отличие от более ранней работы \cite{Vas2018b}, рассматривающей волны в плоском пространстве, фактических аналитических решений уравнений динамики получено не было. Это связано с более сложной структурой самих уравнений. О плотности и потоке энергии, а также диаграммах направленности излучения говорить пока преждевременно. Без сомнения, однако, что угловые части сферических мод естественным образом переносятся и на шварцшильдовский фон: угловые уравнения не меняются.

\onlyinsubfile{
    \clearpage
    \phantomsection
    \addcontentsline{toc}{section}{Список литературы}
    \bibliographystyle{\docroot/../lib/doc/bib/utf8gosttu}
    \bibliography{\docroot/../lib/doc/bib/math,\docroot/../lib/doc/bib/physics}
}

\end{document}
