\def\docroot{../..}
\documentclass[\docroot/reports/draft/report.tex]{subfiles}

\begin{document}

\onlyinsubfile{\tableofcontents}

Изучение распространения гравитационных волн от точечных источников в пространстве является одной из наиболее важных задач в теоретической и экспериментальной части общей теории относительности. Со времен становления ОТО подобные задачи различной степени общности ставились и изучались как родоначальниками теории, так и их последователями. В свое время еще А.~Эйнштейн \cite{einstein_grav_waves} указал на тот факт, что гравитационные волны очень слабы: для их изучения достаточно лишь линеаризованных уравнений динамики~--- решение ищется в виде малых поправок к некоторой фиксированной фоновой метрике. Многочисленные работы, венцом которых стала статья К.~Торна \cite{thorne_multipole}, старались с различных сторон подойти к этому вопросу.

К.~Шварцшильдом в 1916 г. \cite{schwarzschild_free_space_rus} было найдено статическое решение уравнений Эйнштейна для сферически симметричного пространства. В 1923 г. Дж.Д.~Биркгоф показал, что любая статическая сферически симметричная метрика эквивалентна метрике Шварцшильда. Каково бы ни было решение уравнений Эйнштейна внутри сферической тяготеющей массы (звезды), оно должно сшиваться с внешним решением Шварцшильда. Таким образом, возникает идея, развиваемая, в частности, в работе Т.~Редже и Дж.~Уилера \cite{regge_wheeler_1957}, описания линейных сферических гравитационных волн на фоне наиболее естественной Шварцшильдовской метрики.

В то время как ОТО придерживается принципа общей ковариантности~--- равноправия пространственных и временной координат~--- теория глобального времени, разрабатываемая в работах Д.Е.~Бурланкова (см. напр. \cite{burlankov_space_dynamics,burlankov_grav_waves}), вводит концепцию глобального времени, явно выделяя временн\'{у}ю координату, в попытке избавиться от проблем ОТО, таких как требование обращения полной энергии системы в нуль. ТГВ сужает класс возможных метрик, полагая одну из компонент метрического тензора, а именно $g^{00} \equiv g^{tt}$, фиксированной (равной единице).

В данной и предшествующих работах (\hspace{1sp}\cite{Vas2018b,Vas2019a}) сферические гармоники гравитационных волн рассматриваются в рамках ТГВ и сопоставляются с соответствующими решениями ОТО. Первая работа цикла \cite{Vas2018b} на базе разработанного в ТГВ аппарата представила мультипольное разложение гравитационных волн на фоне метрике Минковского. Поскольку на бесконечности метрика Шварцшильда естественным образом переходит в метрику Минковского, на больших расстояниях от источника поля (звезды) кривизной пространства можно пренебречь. В работе \cite{Vas2019a} были получены уравнения, описывающие внутреннее решение для жидкой самогравитирующей сферы. Там же (и цитированных там работах, в частности, все той же \cite{burlankov_space_dynamics}) показывается, что в рамках ТГВ метрика Шварцшильда переходит в (эквивалентную) метрику Пенлеве.

Данная работа является следующим этапом уже намеченной программы. Она рассматривает сферические гравитационные волны на фоне метрики Пенлеве (Шварцшильда) в рамках теории глобального времени. Основным результатом работы является сформулированное уравнение динамики, обобщение полученного в \cite{Vas2018b} результата.

Изучение гравитационных волн в рамках ТГВ может пролить свет на проблему их детектирования.

\subsection{Постановка задачи}

Целью работы является разработка аппарата описания сферических гравитационных волн на фоне метрики Шварцшильда.

\subsection{Общая теория относительности (ОТО)}
\subfile{\docroot/fragments/gr-intro.tex}

\subsection{Проблемы ОТО и теория глобального времени (ТГВ)}
\subfile{\docroot/fragments/gtt-intro.tex}

\subsection{Применяемые обозначения}
\subfile{\docroot/fragments/notation.tex}

\onlyinsubfile{
    \clearpage
    \phantomsection
    \addcontentsline{toc}{section}{Список литературы}
    \bibliographystyle{\docroot/../lib/doc/bib/utf8gosttu}
    \bibliography{\docroot/../lib/doc/bib/math,\docroot/../lib/doc/bib/physics}
}

\end{document}
