\def\docroot{../..}
\documentclass[\docroot/reports/draft/report.tex]{subfiles}

\begin{document}

\subsection{Угловые части сферических гармоник}

    Поскольку фоновая 3-метрика одинакова как в плоском пространстве, так и в пространстве Шварцшильда, вектора Киллинга, а значит и угловые части сферических гармоник (в силу \autoref{sec:killing} и \autoref{sec:lie-generation}) не меняются при переходе от плоского к искривленному фону. Угловые части сферических гармоник на фоне плоского пространства были получены в работе \cite{Vas2018b} и здесь приводятся без изменений.

    Для основной нечетной гармоники (с $m = 0$)
    %
    \begin{equation}\label{eq:hl0e}
        (h_{l0}^{(o)})_{ij} = \begin{pmatrix}
            0&0&f^l_{13}(r)u^l_{13}(\theta)\sin\theta\\
            0&0&r f^l_{23}(r)u^l_{23}(\theta)\sin\theta\\
            f^l_{13}(r)u^l_{13}(\theta)\sin\theta &
                r f^l_{23}(r)u^l_{23}(\theta)\sin\theta & 0
        \end{pmatrix} \exp(-i \omega t).
    \end{equation}
    %
    угловое уравнение (\ref{eq:killeq-hl0}) определяет функции $u^l_{13}(\theta)$ и $u^l_{13}(\theta)$:
    %
    \begin{equation*}\begin{aligned}
        u''_{13} + \cot\theta u'_{13} + (l(l+1) - \cosec^2\theta) u_{13} &= 0, \\
        u''_{23} + \cot\theta u'_{23} + (l(l+1) - 4 \cosec^2\theta) u_{23} &= 0,
    \end{aligned}\end{equation*}
    %
    откуда
    %
    \begin{equation}\label{hl0o-u}
        u^l_{13} = P_l^1(\cos\theta), \quad u^l_{23} = P_l^2(\cos\theta),
    \end{equation}
    %
    где $P_l^m$~--- $m$-й присоединенный полином Лежандра первого рода.

    Неосновные гармоники (с $m \neq 0$) получаются из основной применением операторов повышения и понижения. Четные (угловые) гармоники получаются из нечетных (угловых) гармоник применением операции полуротора. Относительно радиальных частей, однако, в общем случае шварцшильдовского фона этого сказать нельзя (\autoref{sec:eq-analyze}).

\subsection{Анализ основного уравнения}\label{sec:eq-analyze}

    Для наглядности снова приведем основное вариационное уравнение в форме (\ref{eq:lagr2var}):
    %
    \begin{equation*}\begin{aligned}
        D_t{e^{ij}} + e^{ij} \Opdiv{V} + \Opsr(B)^{ij} = 0, \qquad e^{ij} = E^{ij} - \gamma^{ij} \tr E .
    \end{aligned}\tag{\ref{eq:lagr2var}*}\end{equation*}
    %
    В предыдущей работе \cite{Vas2018b} был сформулирован аналогичный результат при более сильных (и более благоприятных) условиях на фоновую метрику. ${}^{(4)}\gamma_{ij}$ задавалась в виде метрики Минковского. Несложно показать, какие следствия влечет за собой это предположение. Достаточно обратиться к тому факту, что у плоской метрики отсутствует поле скоростей: $V^i = 0$. Дополнительным упрощающим условием в работе было применение калибровки $v^i = 0$ (см. далее \autoref{sec:gauge-invariance}), чем мы также воспользуемся и здесь.

    При $V^i = 0$ инвариантная производная по времени переходит в обычную: $D_t \to \pdv*{t}$. Этого достаточно, чтобы переписать \autoref{eq:lagr2var} в виде
    %
    \begin{equation*}
        \dot{e}^{ij} + \Opsr(B)^{ij} = 0 .
    \end{equation*}
    %
    Оказывается, что в глобальной калибровке $v^i = 0$ лишь только на основании структуры уравнения можно судить о характере всех входящих в него тензоров: $B_{ij}$, $E_{ij}$ и, как следствие, $h_{ij}$ должны быть симтензорами, т.е. симметричными тензорами с нулевыми следом $\tr$ и симдивергенцией $\Opsdiv$ (это легко видеть: $\Opsr(B) = \Opbr(h)$ уже является симтензором в силу свойств оператора $\Opsr$, достаточно лишь затребовать те же свойства для оставшейся части уравнения). Имея это в виду, можно переписать полученное соотношение сразу относительно тензора возмущений $h_{ij}$:
    %
    \begin{equation}\label{eq:lagr2var-noV}
        \ddot{h} + \Opbr(h) = 0 .
    \end{equation}
    %
    Данное уравнение и было названо \textit{основным вариационным уравнением} для плоского пространства.

    В случае с исходным \autoref{eq:lagr2var} мы ничего не можем сказать относительно входящих в него величин (кроме симметричности всех тензоров, следующей из симметричности самог\'{о} $h_{ij}$): взятие $\tr$ или $\Opsdiv$ от левой части уравнения уничтожает полуроторную компоненту, однако оставляет нас с довольно нетривиальными условиями, о выполнимости которых в общем случае довольно сложно судить. Это хорошо демонстрируется следующими выкладками на примере взятия следа:
    %
    \begin{gather*}
        D_t{e^{ij}} + e^{ij} \Opdiv{V} + \Opsr(B)^{ij} = 0 \quad |\ \tr \\
        \tr D_t{e^{\alpha\beta}} + e \Opdiv{V} = 0 \\
        \tr D_t{e^{ij}} = \gamma_{ij} \qty(
            \dot{e}_{ij} + V^s e^{ij}_{;s} - V^i_{;s} e^{sj} - V^j_{;s} e^{is}
        ) = \dot{e} + V^s e_{;s} - 2 V^{i;j} e_{ij} \quad\Longrightarrow\\
        \dot{e} + \nabla_V e + e \Opdiv{V} = 2 V^{i;j} e_{ij} .
    \end{gather*}
    %
    Здесь применено сокращение $\tr(e^{ij}) = e$. Полученное выражение является сравнительно более сложным, чем в случае $V^i = 0$, и даже априорное требование $e = 0$, которое хотелось бы наложить, оглядываясь на случай плоского пространства, не обращает его в тождество. В лучшем случае мы имеем требование $V^{i;j} e_{ij} = 0$, явный смысл которого, в частности, выполнимость его в общем случае, мы указать не в силах.

    Взятием полуротора от основного уравнения можно прийти к еще более громоздкому результату, из которого явно также не следует, что $e_{ij}$ необходимо должен быть симтензором. Априорное предположение этого также не облегчает положения.

    Интуитивно $h_{ij}$ также следует искать в виде симтензора, на что намекает следующее обстоятельство.

    Операция $\Opsr(h)$ переводит нечетные гармоники в четные (и наоборот). Для нечетной гармоники верно $\tr h^{(o)} = 0$. Двойное применение оператора $\Opsr(h)$, т.е. оператор $\Opbr(h)$, не меняет четности гармоники. Иными словами, решение уравнения (\ref{eq:lagr2var}) можно искать сначала в классе нечетных решений, а затем генерировать четные применением $\Opsr(h)$, как это уже объяснялось в предыдущих разделах.

    Однако сгенерированное таким образом решение также будет иметь нулевой след. Для случая $V^i = 0$ условие $\tr h = 0$ также гарантировало $\tr h^{(e)} = 0$. Здесь такое условие не является следствием вариационных уравнений, а значит формально допустимы такие четные моды, для которых $\tr h^{(e)} \neq 0$.

    Более того, априорное предположение $\tr h = 0$ наравне с $\tr E = 0$ также влечет за собой нетривиальное условие $V^{i;j} h_{ij} = 0$, которое в общем случае может и не выполняться.

    По всей видимости, данные трудности можно было бы несколько сгладить подходящим выбором калибровочного поля скоростей $v^i \neq 0$. Верность этого предположения можно было бы углядеть из сопоставления полученных здесь результатов с результатами Редже и Уилера \cite{regge_wheeler_1957} путем приведения их решений к глобальному времени.

\subsection{Решение для нечетной моды}

    Уже не раз отмечалось, что решение $h^{(o)}$ имеет наименьшее число компонент, наиболее простой вид, а также нулевой след. Отсутствие симдивергенции справедливо, правда, лишь для плоского пространства. Угловая часть базовой нечетной моды была определена ранее. Радиальная часть определится подстановкой общего вида решения в основное вариационное уравнение.

    Оператор основного динамического уравнения является пространственно-временным дифференциальным оператором второго порядка. Действуя на основную нечетную гармонику, он приводит к системе двух дифференциальных уравнений второго порядка относительно двух неизвестных функций $f_{13}(r)$ и $f_{23}(r)$ вида
    %
    \begin{equation*}\begin{aligned}
        a_1(r) f_{13}(r) + b_1(r) f_{13}'(r) + c_1(r) f_{13}''(r) +
        d_1(r) f_{23}(r) + e_1(r) f_{23}'(r) = 0 \\
        b_2(r) f_{13}'(r) + d_2(r) f_{23}(r) +
        e_2(r) f_{23}'(r) + f_2(r) f_{23}''(r) = 0
    \end{aligned}\end{equation*}
    %
    с комплексными коэффициентами $a_i(r) \dots f_i(r)$. Данная система имеет довольно сложный вид и не сводится простыми преобразованиями к разрешимой системе уравнений. Ее решение сопряжено с большими трудностями, если вообще возможно. Возможные пути смягчения указанной сложности были намечены в \autoref{sec:eq-analyze}.

    Следует отметить, что для нечетной моды величина $\tr e$ обращается в нуль, чего нельзя сказать о $\Opsdiv(e)$.

\end{document}
