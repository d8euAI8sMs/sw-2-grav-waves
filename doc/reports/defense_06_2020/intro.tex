\def\docroot{../..}
\documentclass[\docroot/reports/draft/report.tex]{subfiles}

\begin{document}

\onlyinsubfile{\tableofcontents}

Одной из наиболее фундаментальных метрик в общей теории относительности является метрика Шварцшильда, возникающая естественным образом как решение уравнений Эйнштейна для свободного статического сферически симметричного пространства. Данная метрика была впервые получена К.~Шварцшильдом в 1916~г. \cite{schwarzschild_free_space_rus} на заре становления ОТО.

Немало работ, в частности знаменитая статья Т.~Редже и Дж.~Уилера 1957~г. \cite{regge_wheeler_1957}, посвящены анализу шварцшильдовой геометрии как таковой, линеаризованной гравитации на фоне обсуждаемой метрики и разложению линейных гравитационных волн по сферическим гармоникам, исследованию взаимодействия гравитационного поля с электромагнитным и т.д.

Автору, однако, неизвестны статьи, сколь либо подробно разбирающие сугубо скалярные или сугубо векторные (в частности электромагнитное) поля в метрике Шварцшильда или какой-либо другой близкой по геометрии метрике в отсутствие гравитации.

Хотя данная задача имеет чисто академический интерес, потому как не рассматривает взаимодействие скалярного или векторного поля с гравитационным, соответственно не нуждается в решении уравнений Эйнштейна как таковых, она все же позволяет улучшить понимание механизмов распространения волн, в т.ч. потенциально гравитационных, в искривленных (римановых) пространствах.

В основе лагранжева подхода к решению физических задач лежит принцип наименьшего действия с функцией Лагранжа под интегралом действия. Лагнаржев подход легко обобщается на любые римановы пространства, в частности на пространства с геометрией Шварцшильда. Известны функции Лагранжа для скалярных, векторных (электромагнитных) и тензорных (гравитационных) волн. В частности, вариацией лагранжиана Гильберта в ОТО можно получить уравнения Эйнштейна, а вариация лагранжиана электромагнитных волн дает уравнения Максвелла. Изучение механизма распространения волн меньшей тензорной размерности и их математического описания может помочь в понимании механизмов распространения гравитационных волн.

\subsection{Постановка задачи}

Целью работы является исследование скалярных и векторных волн (сферических гармоник, плотности и потока энергии) в геометрии Шварцшильда, сравнение их с волнами в плоском пространстве (пространстве Минковского).

\subsection{Применяемые обозначения}\label{seq:notation}
\subfile{\docroot/fragments/notation.tex}

\onlyinsubfile{
    \clearpage
    \phantomsection
    \addcontentsline{toc}{section}{Список литературы}
    \bibliographystyle{\docroot/../lib/doc/bib/utf8gosttu}
    \bibliography{\docroot/../lib/doc/bib/math,\docroot/../lib/doc/bib/physics}
}

\end{document}
