\def\docroot{../..}
\documentclass[\docroot/reports/draft/report.tex]{subfiles}

\begin{document}

\onlyinsubfile{\tableofcontents}

В данной работе построены сферические гармоники скалярных и векторных волн в метрике Шварцшильда, обозначены особенности данных волн относительно волн в плоском пространстве. Волны отличаются по многим параметрам, главный из которых~--- наличие в искривленном пространстве областей с разными типами волн. Внутри сферы гравитационного радиуса волны являются почти стандартными бегущими волнами. В то же время вне этой сферы существуют лишь стоячие волны (с небольшой долей бегущей волны, наиболее сильно выраженной вблизи гравитационного радиуса).

Полученные решения, включая такие характеристики решений, как энергия и поток энергии, были сопоставлены с решениями в неискривленном (плоском) пространстве. Оказалось, что за гравитационным радиусом волны существенно напоминают стоячие волны в плоском пространстве, хотя от таковых и отличаются.

Энергетические характеристики векторных стоячих волн в плоском и искривленном пространстве далеко за гравитационным радиусом существенно ближе друг к другу, чем аналогичные скалярных волн. Из этого едва ли можно сделать вывод, что с увеличением количества компонент поля влияние внешней геометрии на решения снижается, однако данный факт является интересным и требует более глубокого исследования.

\onlyinsubfile{
    \clearpage
    \phantomsection
    \addcontentsline{toc}{section}{Список литературы}
    \bibliographystyle{\docroot/../lib/doc/bib/utf8gosttu}
    \bibliography{\docroot/../lib/doc/bib/math,\docroot/../lib/doc/bib/physics}
}

\end{document}
