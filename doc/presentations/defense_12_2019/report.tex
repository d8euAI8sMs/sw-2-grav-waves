\documentclass[compress]{beamer}

\def\docroot{../..}

\input{\docroot/../lib/doc/pkg/beamer}
\input{\docroot/../lib/doc/pkg/beamer_flat_style}
\input{\docroot/../lib/doc/math/operators}

\usepackage{array} %% for fixed-width columns in `tabular`

\graphicspath{{\docroot/img/}}

\deftranslation[to=russian]{Theorem}{Теорема}
\deftranslation[to=russian]{theorem}{Теорема}

\title{Гравитационные волны на фоне метрики Шварцшильда}
\author[Василевский~А.В.]{
    Василевский~А.В. \\[\baselineskip]
    {\footnotesize Научный руководитель: Бурланков~Д.Е.}
}
\institute[ННГУ]{Нижегородский университет им. Н.И.~Лобачевского}
\date{2019}

\begin{document}

    \frame[plain]{\titlepage}

    \begin{frame}\frametitle{Введение}

        \begin{itemize}
            \item Гравитационные волны очень слабы~--- достаточно линеаризованных уравнений относительно возмущений некоторой фоновой метрики.
            \item Наиболее простая фоновая метрика~--- плоская метрика Минковского.
            \item Наиболее естественная фоновая метрика для статической сферически симметричной задачи~--- метрика Шварцшильда (внешняя метрика тяготеющей массы~--- звезды).
        \end{itemize}

    \end{frame}

    \begin{frame}\frametitle{Введение}

        \begin{itemize}
            \item Сферические гармоники гравитационных волн на фоне метрики Шварцшильда в общем виде были получены в работе Т.~Редже и Дж.~Уилера \cite{regge_wheeler_1957}.
            \item Работа \cite{regge_wheeler_1957} (и проч., напр. \cite{thorne_multipole}) исходят из ОТО, данная работа базируется на ТГВ (теории глобального времени) \cite{burlankov_space_dynamics}, где время~--- выделенная переменная.
        \end{itemize}

    \end{frame}

    \begin{frame}\frametitle{Постановка задачи}

        Исследование аппарата описания сферических гармоник гравитационных волн на фоне неплоской метрики (Шварцшильда) в рамках ТГВ.

    \end{frame}

    \begin{frame}\frametitle{Общий вид метрики}

        \begin{equation*}
            {}^{(4)}g^{ij}_\text{тгв} = \begin{pmatrix}
                - 1   & - V^j                           \\
                - V^i & {}^{(3)}g^{ij} + V^i V^j
            \end{pmatrix} .
        \end{equation*}

        В ТГВ ${}^{(4)}g^{00} = -1$~--- следствие теории (глобального времени).

    \end{frame}

    \begin{frame}\frametitle{Лагранжиан Гильберта}

        Для слабых ГВ истинное решение~--- фон плюс малые поправки:
        %
        \begin{equation*}
            {}^{(4)}g_{ij} = {}^{(4)}\gamma_{ij} + {}^{(4)}h_{ij}, \qquad
            {}^{(4)}h_{ij} = \begin{pmatrix}
                0    & - v_j          \\
                -v_i & {}^{(3)}h_{ij}
            \end{pmatrix}
        \end{equation*}

        Для получения линейных вариационных уравнений достаточно квадратичной вырезки из лагранжиана Гильберта:
        %
        \begin{equation}
            \mathcal{L} = \gamma^{ij}\gamma^{kl} (E_{ik}E_{jl} - E_{ij}E_{kl}) - B^i_j B_i^j .
        \end{equation}

    \end{frame}

    \begin{frame}\frametitle{Основное вариационное уравнение}

        Вариацией $\sqrt{\gamma} \mathcal{L}$ по $h_{ij}$ можно прийти к вариационному уравнению:
        %
        \begin{equation}
            D_t{e^{\alpha\beta}} + e^{\alpha\beta} \Opdiv{V} + \Opsr(B)^{\alpha\beta} = 0,
        \end{equation}
        \begin{gather*}
            e^{ij} = E^{ij} - \gamma^{ij} \tr E, \quad
            E_{ij} = D_t{h_{ij}} + v_{i;j} + v_{j;i}, \\
            B^i_j = \Opsr(h)^i_j = \varepsilon^{ikl} h_{jk;l} \\
            D_t{h_{ij}} = \dot{h}_{ij} - \delta_V h_{ij} ,
        \end{gather*}
        %
        $\delta_V h_{ij}$~--- Ли-вариация тензора $h_{ij}$. Поле $v^i$~--- калибровочное: полагаем $v^i = 0$.

    \end{frame}

    \begin{frame}\frametitle{В плоском пространстве}

        Метрика Минковского: $V^i = 0$. Приходим к упрощенному варианту
        %
        \begin{equation}
            \ddot{h} + \Opsr^2(h) = 0 .
        \end{equation}

        Метрика Шварцшильда в глобальном времени~--- метрика Пенлеве с $V^i = \qty{ V^r(r), 0, 0 }$. Упростить уравнение не удается.

    \end{frame}

    \begin{frame}\frametitle{Угловые и радиальные части гармоник}

        \begin{equation*}
            h_{ij}(t,r,\theta,\phi) = f^r_{ij}(r) f^\Omega_{ij}(\theta,\phi) \exp(-i \omega t) .
        \end{equation*}

        \begin{enumerate}
            \item Угловые части $f^\Omega_{ij}(\theta,\phi)$ те же, что и для плоского фона (получение методом Ли-генерации).
            \item Для радиальных частей $f^r_{ij}(r)$ не найдено аналитического выражения (сложная система ДУ 2-го порядка).
        \end{enumerate}

    \end{frame}

    \begin{frame}\frametitle{Заключение}

        \begin{itemize}
            \item Исследован аппарат описания сферических гармоник на фоне метрики Шварцшильда.
            \item Показано, что угловые части мод не меняются при переходе от плоского фона к шварцшильдовскому.
            \item Радиальные части мод удовлетворяют сложным уравнениям, для которых аналитического решения не найдено. Ищется возможность использования $v^i \neq 0$.
        \end{itemize}

    \end{frame}

    \begin{frame}\frametitle{Литература}

        {\tiny{
        \bibliographystyle{\docroot/../lib/doc/bib/utf8gosttu}
        \bibliography{\docroot/../lib/doc/bib/math,\docroot/../lib/doc/bib/physics}
        }}

    \end{frame}

\end{document}
