\documentclass[compress]{beamer}

\providecommand{\docroot}{../..}

\input{\docroot/../lib/doc/pkg/beamer}
\input{\docroot/../lib/doc/pkg/beamer_flat_style}
\input{\docroot/../lib/doc/math/operators}

\usepackage{array} %% for fixed-width columns in `tabular`

\graphicspath{{\docroot/img/}}

\deftranslation[to=russian]{Theorem}{Теорема}
\deftranslation[to=russian]{theorem}{Теорема}

\title{Идеальная жидкость в ОТО и ТГВ}
\author[Василевский~А.В.]{
    Василевский~А.В. \\[\baselineskip]
    {\footnotesize Научный руководитель: Бурланков~Д.Е.}
}
\institute[ННГУ]{Нижегородский университет им. Н.И.~Лобачевского}
\date{2019}

\begin{document}

    %
    %
    %
    %%%%%%%%%%%%%%%%%%%%%%%%%%%%%%%%%%%%%%%%%%%%%%%%%%%%%%%%%%%%%%%%%%%%%%%
    %                            FRAME                                    %
    %%%%%%%%%%%%%%%%%%%%%%%%%%%%%%%%%%%%%%%%%%%%%%%%%%%%%%%%%%%%%%%%%%%%%%%
    %
    %
    %

    \frame[plain]{\titlepage}

    %
    %
    %
    %%%%%%%%%%%%%%%%%%%%%%%%%%%%%%%%%%%%%%%%%%%%%%%%%%%%%%%%%%%%%%%%%%%%%%%
    %                            FRAME                                    %
    %%%%%%%%%%%%%%%%%%%%%%%%%%%%%%%%%%%%%%%%%%%%%%%%%%%%%%%%%%%%%%%%%%%%%%%
    %
    %
    %

    \begin{frame}\frametitle{Введение}

        \begin{itemize}
            \item Идеальная жидкость~--- одна из наиболее простых моделей вещества в ОТО.
            \item Большинство реальных звезд неплохо описывается в приближении идеальной жидкости.
            \item Квадрупольные колебания идеальной жидкой сферы~--- по-видимому простейший источник гравитационных волн.
        \end{itemize}

    \end{frame}

    %
    %
    %
    %%%%%%%%%%%%%%%%%%%%%%%%%%%%%%%%%%%%%%%%%%%%%%%%%%%%%%%%%%%%%%%%%%%%%%%
    %                            FRAME                                    %
    %%%%%%%%%%%%%%%%%%%%%%%%%%%%%%%%%%%%%%%%%%%%%%%%%%%%%%%%%%%%%%%%%%%%%%%
    %
    %
    %

    \begin{frame}\frametitle{Постановка задачи}

        \begin{itemize}
            \item Исследование аппарата описания идеальной жидкости в ОТО и ТГВ.
            \item Исследование взаимодействия гравитационного поля с жидкостью на примере некоторых статических сферически симметричных задач.
        \end{itemize}

    \end{frame}

    %
    %
    %
    %%%%%%%%%%%%%%%%%%%%%%%%%%%%%%%%%%%%%%%%%%%%%%%%%%%%%%%%%%%%%%%%%%%%%%%
    %                            FRAME                                    %
    %%%%%%%%%%%%%%%%%%%%%%%%%%%%%%%%%%%%%%%%%%%%%%%%%%%%%%%%%%%%%%%%%%%%%%%
    %
    %
    %

    \begin{frame}\frametitle{Лагранжиан жидкости}

        Лагранжиан жидкости (Б.Шютц \cite{schutz_vel_pot}):
        %
        \begin{equation*}
            S_f = \int p(\mu) \sqrt{-g} \dd[4]{x} , \quad
            \mu = \sqrt{- g^{ab} \phi_{,a} \phi_{,b}} ,
        \end{equation*}
        %
        $p$~--- давление жидкости, $\mu$~--- энтальпия, $u_a = \mu^{-1} \phi_{,a}$~--- \mbox{4-скорость}, $\rho$~--- плотность вещества, $\varepsilon$~--- плотность энергии.

        \begin{align*}
            \fdv{S_f}{\phi} = 0 \quad&\Longrightarrow\quad \nabla_u \rho + \rho\ \nabla \cdot \vb{u} = 0 \quad\text{(ур-е неразрывности)}, \\
            \fdv{S_f}{g^{ab}} = \mathfrak{T}^f_{ab} \quad&\Longrightarrow\quad T^f_{ab} = (p + \varepsilon) u_a u_b + p g_{ab} ,
        \end{align*}

    \end{frame}

    %
    %
    %
    %%%%%%%%%%%%%%%%%%%%%%%%%%%%%%%%%%%%%%%%%%%%%%%%%%%%%%%%%%%%%%%%%%%%%%%
    %                            FRAME                                    %
    %%%%%%%%%%%%%%%%%%%%%%%%%%%%%%%%%%%%%%%%%%%%%%%%%%%%%%%%%%%%%%%%%%%%%%%
    %
    %
    %

    \begin{frame}\frametitle{Скорость звука}

        Скорость звука $a$ в жидкости определяется зависимостью $p(\varepsilon)$:
        %
        \begin{equation*}
            \frac{a^2}{c^2} \equiv q^{-1} = \dv{p}{\varepsilon} .
        \end{equation*}

        \begin{itemize}
            \item Две крайности: $a = \infty$~--- несжимаемая жидкость, $q = 3$~--- ультрарелятивистская жидкость.
            \item Первая противоречит принципу относительности ($a < c$), однако это самая простая модель.
            \item Вторая~--- предел, к которому стремится жидкость при высоких температуре и давлении (например, в звезде \cite{oppenheimer_volkoff,burlankov_new_phys}).
        \end{itemize}

    \end{frame}

    %
    %
    %
    %%%%%%%%%%%%%%%%%%%%%%%%%%%%%%%%%%%%%%%%%%%%%%%%%%%%%%%%%%%%%%%%%%%%%%%
    %                            FRAME                                    %
    %%%%%%%%%%%%%%%%%%%%%%%%%%%%%%%%%%%%%%%%%%%%%%%%%%%%%%%%%%%%%%%%%%%%%%%
    %
    %
    %

    \begin{frame}\frametitle{Лагранжиан пространства-материи}

        Действие Гильберта:
        %
        \begin{equation*}
            S_g = \int R^{(4)} \sqrt{-g} \dd[4]{x} .
        \end{equation*}

        Суммарное действие:
        %
        \begin{equation*}
            S = S_f + S_f = \int \qty(R + p) \sqrt{-g} \dd[4]{x} .
        \end{equation*}

        \begin{equation*}
            \fdv{S}{g_{ab}} = 0 \quad\Longrightarrow\quad \text{неизвестные коэффициенты метрики}
        \end{equation*}

    \end{frame}

    %
    %
    %
    %%%%%%%%%%%%%%%%%%%%%%%%%%%%%%%%%%%%%%%%%%%%%%%%%%%%%%%%%%%%%%%%%%%%%%%
    %                            FRAME                                    %
    %%%%%%%%%%%%%%%%%%%%%%%%%%%%%%%%%%%%%%%%%%%%%%%%%%%%%%%%%%%%%%%%%%%%%%%
    %
    %
    %

    \begin{frame}\frametitle{Статика жидкой сферы}

        Наиболее общий вид диагональной сферически симметричной статической метрики~--- метрика Шварцшильда:
        %
        \begin{equation*}
            g_{\mu\nu} = \text{diag} \begin{pmatrix}
                -P(r),& Q(r),& r^2,& r^2 \sin\theta
            \end{pmatrix} .
        \end{equation*}

        В ТГВ метрика в общем случае не диагональна:
        %
        \begin{equation*}
            {}^{(4)} g_{\mu\nu} = \begin{pmatrix}
                1 - V_s V^s  & -V_k   \\
                -V_i         & {}^{(3)} g_{ik} \\
            \end{pmatrix} .
        \end{equation*}

        Сферически симметричная метрика в ТГВ:
        %
        \begin{equation*}
            g_{ij} = \text{diag} \begin{pmatrix}B^{-1}, & r^2, & r^2 \sin\theta\end{pmatrix} , \quad
            V^i = \begin{pmatrix}V(r), & 0, & 0\end{pmatrix} .
        \end{equation*}

    \end{frame}

    %
    %
    %
    %%%%%%%%%%%%%%%%%%%%%%%%%%%%%%%%%%%%%%%%%%%%%%%%%%%%%%%%%%%%%%%%%%%%%%%
    %                            FRAME                                    %
    %%%%%%%%%%%%%%%%%%%%%%%%%%%%%%%%%%%%%%%%%%%%%%%%%%%%%%%%%%%%%%%%%%%%%%%
    %
    %
    %

    \begin{frame}\frametitle{Решения}

        {\footnotesize{
        \begin{tabular}{|m{4em}|m{13em}|m{13em}|}\hline
            {} & Шварцшильд & ТГВ \\\hline\hline
            Вакуум & К.Шварцшильд, 1916 г. \cite{schwarzschild_free_space_rus} & Д.Е.Бурланков, 2006 г. \cite{burlankov_space_dynamics,burlankov_new_phys} \\\hline
            Несж.ж. & К.Шварцшильд, 1916 г. \cite{schwarzschild_fluid} & Наст. работа\footnotemark{}, 2019 г. \\\hline
            Ультр.ж. & Попытки Толмана (1934 г.), Оппенгеймера (1939 г.) и др. \cite{tolman,oppenheimer_volkoff,oppenheimer_snyder} & Попытки Д.Е.Бурланкова, 2006 г. \cite{burlankov_space_dynamics,burlankov_new_phys} \\\hline
        \end{tabular}
        }}

        \footnotetext{
            Из первых принципов, а также через преобразование метрик (см. далее).
        }

        \vfill

        Рассмотрим далее одно из решений (несжимаемая жидкость в метрике Шварцшильда; выглядит компактнее).

    \end{frame}

    %
    %
    %
    %%%%%%%%%%%%%%%%%%%%%%%%%%%%%%%%%%%%%%%%%%%%%%%%%%%%%%%%%%%%%%%%%%%%%%%
    %                            FRAME                                    %
    %%%%%%%%%%%%%%%%%%%%%%%%%%%%%%%%%%%%%%%%%%%%%%%%%%%%%%%%%%%%%%%%%%%%%%%
    %
    %
    %

    \begin{frame}\frametitle{Несжимаемая жидкость}

        В шварцшильдовой метрике (внутреннее решение):
        %
        \begin{gather}
            \begin{aligned}
                g_{\mu\nu} = \text{diag} \left(
                    - \frac{1}{4} \qty(\sqrt{1 - \frac{r^2}{\mathcal{R}^2}} - 3 \sqrt{1 - \frac{R^2}{\mathcal{R}^2}})^2\right.&, \\
                    \left.\qty(1 - \frac{r^2}{\mathcal{R}^2})^{-1},\,
                    r^2,\,
                    r^2 \sin\theta
                \right)& ,
            \end{aligned}\\
            p = \varepsilon \frac{\sqrt{\varepsilon R^2 - 3} - \sqrt{\varepsilon r^2 - 3}}{
                \sqrt{\varepsilon r^2 - 3} - 3 \sqrt{\varepsilon R^2 - 3}
            } .
        \end{gather}
        %
        Давление $p(r)$ уменьшается к границе сферы (радиуса $R$; константа $\mathcal{R}\sim R^3/M$).

    \end{frame}

    %
    %
    %
    %%%%%%%%%%%%%%%%%%%%%%%%%%%%%%%%%%%%%%%%%%%%%%%%%%%%%%%%%%%%%%%%%%%%%%%
    %                            FRAME                                    %
    %%%%%%%%%%%%%%%%%%%%%%%%%%%%%%%%%%%%%%%%%%%%%%%%%%%%%%%%%%%%%%%%%%%%%%%
    %
    %
    %

    \begin{frame}\frametitle{Координатные преобразования}

        Преобразованием дифференциала времени по формуле
        %
        \begin{equation*}
            \dd{t}_\text{шв} = \dd{t}_\text{гв} + w(r) \dd{r}, \quad
            w(r) = \frac{V(r)}{B(r) - V^2(r)}
        \end{equation*}
        %
        можно перейти от сферически-симметричной метрики в глобальном времени к метрике Шварцшильда (и наоборот).

        Давление жидкости $p(r)$ не затронуто преобразованием дифференциала времени и не должно отличаться по форме в разных метриках.

        Вопрос выбора метрики с математической точки зрения~--- вопрос удобства. С физической~--- принципиальный.

    \end{frame}

    %
    %
    %
    %%%%%%%%%%%%%%%%%%%%%%%%%%%%%%%%%%%%%%%%%%%%%%%%%%%%%%%%%%%%%%%%%%%%%%%
    %                            FRAME                                    %
    %%%%%%%%%%%%%%%%%%%%%%%%%%%%%%%%%%%%%%%%%%%%%%%%%%%%%%%%%%%%%%%%%%%%%%%
    %
    %
    %

    \begin{frame}\frametitle{Заключение}

        \begin{itemize}
            \item Исследован и практически применен лагранжев подход к описанию динамики жидкости в ОТО и ТГВ.
            \item Исследованы некоторые модельные статические задачи.
            \item В рамках ТГВ получено решение новой задачи о равновесии несжимаемой жидкой сферы.
        \end{itemize}

    \end{frame}

    %
    %
    %
    %%%%%%%%%%%%%%%%%%%%%%%%%%%%%%%%%%%%%%%%%%%%%%%%%%%%%%%%%%%%%%%%%%%%%%%
    %                            FRAME                                    %
    %%%%%%%%%%%%%%%%%%%%%%%%%%%%%%%%%%%%%%%%%%%%%%%%%%%%%%%%%%%%%%%%%%%%%%%
    %
    %
    %

    \begin{frame}\frametitle{Литература}

        {\tiny{
        \bibliographystyle{\docroot/../lib/doc/bib/utf8gosttu}
        \bibliography{\docroot/../lib/doc/bib/math,\docroot/../lib/doc/bib/physics}
        }}

    \end{frame}

\end{document}
